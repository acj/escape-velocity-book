\documentclass[12pt]{book}
\usepackage[paperwidth=5.25in,paperheight=8in,hmargin=0.75in,vmargin=0.5in]{geometry}

% To skip blank pages that make new chapters start on a right-side page
%\makeatletter\@openrightfalse

\begin{document}

\chapter{}

``---escape velocity?  It's too much---'' the woman's voice had said.

Jesse stopped, wheeling around to find the voice, but it was lost in the groggy brook of sounds and suitcoats moving toward him, away from him, and breaking around him.  He had just been thinking about his job and remembering his mother's advice.  \emph{When you're stuck but ambition pulls you up, use it, you're already at escape velocity}\dots.  A short passerby bumped into him, sending a momentary pain through his neck and shoulders.  The stranger apologized and tipped his hat, quickly regaining his brisk walking pace.

\emph{Nobody else's mother teaches life lessons with physics metaphors}, Jesse thought, allowing himself a wry smile.  He looked across Bedford Avenue, took a deep breath, and stepped back into the stream of Brooklyners.

\chapter{}

The City University of New York Office of Internal Affairs was cold this morning.  The calendar above Sascha Greene's desk would lead one to believe that summer raged on, but the brisk morning air left no doubt that late September was a better guess.

``Sascha, the top feed is ready,'' called a voice near the window.  The Office of Internal Affairs, despite the lofty title, was a cramped, stuffy office that contained five desks, eighteen computers and other office machines, and could comfortably fit three bodies, provided that the bodies were reasonably proportioned and were not given to clumsiness.  Sascha sat facing the north wall, preferring not to be distracted by the constant movement of his colleagues.  Jill, the office technology specialist, sat near the windows where she could keep a watchful eye on the myriad devices that kept the room warm.  Sean, the interdepartmental liason (the External Internal Affairs Spokesman, Sascha called him), worked at a stand-up desk that was mounted on the west wall between the two oat doors.  The remainder of the 600 square foot space was a labyrinth of machines, cables clamped to the floor with black duct tape, stacks of official university records, and the occasional abandoned coffee mug.  The abundance of machines meant that one window stayed propped open year-around.

``Got it,'' Sascha said.  He stretched his shoulders and called up the day's list of business items that had been assembled by the administrative assistants on the floor above.  \emph{Amazing that these lists ever get out the door\emph{, he thought.  Sascha knew that despite the polished look of the top feed, it was the result of hours of messy negotiations, hearsay, and petty politics.  }If the students only knew what they were paying for}.

The top feed was a document that laid out, in gory and often career-damaging detail, each instance of academic dishonesty, professional misconduct, inter-departmental dispute, and plenty of other violations of laws or university policies.  The name of each involved party, whether accuser or accused, was listed in bold print alongside a description of the incident and various recommendations that the faculty senate or (in severe cases) the board of directors had made for resolving the issue at hand.  A copy was emailed to the president of the university, the ombudsman, the vice president of Internal Affairs (Sascha's superior), and the chief of university police.  Several delicate, high-profile cases had been exposed when details from the top feed had made their way into local newspapers, placing the university in an awkward situation.

Scanning through the list, Sascha spotted a familiar name.  He took a drink from his coffee mug, taking a moment to process the news.  He absently poked the rubber Gumby figure sitting next to his monitor, making it jiggle.

``Jill, did you see that Winter is on the list again?'' he said, his eyes still on the computer screen.

``Hadn't noticed.  What's the claim?'' Jill asked.

``Espionage and destruction of university property,'' Sasched said, his voice laced with disbelief.

Sascha heard the steady rhythm of Jill's typing stop abruptly.  He turned to face her.  ``This isn't his first time on the feed, but it sounds like it could be his last.  And I'd bet my left nut that he's innocent.''

Jill ignored the comment.  ``I'll admit that espionage is over the top, even for this group of suits.  Winter's first time was inconclusive, though, and you have to admit that it was fishy.''

``Fishy or not, the guy has a good moral compass.  He's innocent.''

The rhythmic typing resumed, and Sascha turned back to his computer.  He scrawled a note on his legal pad, downed the last of his coffee, and left the office.

\chapter{}
								
Jesse swallowed hard, wondering how he had gotten here.  The university's prosecutor had just painted him as a screwoff that, even though he should never have been hired, \emph{most definitely} should not be allowed to handle anything more than a press release for an obscure bit of undergraduate research.  Jesse had admired the bastard's poise and confidence as he delivered one gently stretched fact after another, building to a rousing conclusion that even Jesse found convincing.  But it simply didn't agree with reality.  \emph{Never let the facts get in the way}, Jesse thought to himself.

Three weeks earlier, Jesse had just finished writing a report for his superiors that evaluated the success of a new program at CUNY that paired journalism students with investigative reporters from major news organizations and private investigative agencies.  William Tweedie, the current president of CUNY, had made it his personal mission to restore investigative journalism to its rightful place as a tool that feeds the beating heart of democracy.  In addition to the partnership with private agencies, the vision stated, students would conduct an internship overseas in an area of strategic interest to the United States.  Military zones and several unfriendly nations were understandably excluded, but this did nothing to blunt the flood of criticism that the Tweedie, and the university as a whole, endured in the months that followed.

One prominent liberal blog bemoaned the arrogance and gall of a university president who would put his students at risk in foreign political landscapes where the local newspapers were as likely to carry a story about a reporter who had been jailed and executed as a spy as they were to discuss news about the London Stock Exchange.  Conservative hawks were just as eager to pounce, declaring investigative journalism a dead art that shouldn't be put on life support.  The interference of reporters in military and diplomatic affairs, they went on, had a detrimental effect on combat effectiveness and encouraged commanders and politicians to withdraw from public inquiry and debate for fear of being mischaracterized in the press.

Tweedie refused to pull the plug on the program that had made him a household name.  Perhaps by taking public figures to task and embracing modern technology as a means of tracking and eliminating corruption, he argued, the young public can wrest control of its future from the cash-heavy organizations that have a key vested interest in keeping everyone in the dark.  It took a toll on him, which was evidenced most clearly by the private security detail that he hired to keep a few eyes on his family's small house.  His efforts to promote investigative journalism and to encourage students to take up the fight inspired many, especially students in France and elsewhere in the United States, to form impromptu groups of vigilante reporters.  There was a steady stream of reports of students being arrested or harassed by police after they had been found trespassing or had been accused of stalking.  While the actions of these student reporters departed from what Tweedie had intended, it told him something important: the will to peel back the layers of secrecy and deceit that crippled the public's understanding of major events, long thought to be lost on the young generations of the 21st century, was alive and well.

At the time of his first trial, Jesse Winter was 29.  A graduate of University of Colorado at Boulder, he held a dual degree in public policy and mathematics, with a concentration in information technology policy.  He had also taken a minor in poetry.  Jesse had served as the president of the campus debate club, had a senior internship at the Electronic Frontier Foundation, and wrote a technology column for a Boulder newspaper.

Jesse joined CUNY fresh out of college.  He had been offered positions at the Independent newspaper in the United Kingdom and a handful of private firms in Germany and the northeastern United States.  Making the decision to stay in the U.S. had been difficult, but the job paid well enough that Jesse could afford to travel and satisfy his interest in foreign politics.  His official title was Special Deputy to the Dean, working in the Office of the Ombudsman.  While this was misleading--his duties frequently involved international travel to undisclosed locations and conferring with individuals that the university could not publicly recognize as partners--much of his work was in the vein of settling disputes.

When William Tweedie had announced his pet program for reinvigorating investigative journalism, he had personally notified Jesse that his job description would be changing.  If he felt comfortable with the plan, then he was welcome to stay.  Jesse was intrigued by the promise of hush-hush work, additional travel, and doing his part to promote an old component of journalism that he had found lacking.  He signed a non-disclosure agreement, agreed to the revised terms, and came to work the next week to find that his desk and computer had been removed.  In its place was a note from the dean and a plane ticket without a marked destination that left from a terminal at JFK that Jesse had  never heard of.  ``This should be educational,'' he said, turning on his heel and heading toward the elevator.

``You're going to eastern Europe,'' the dean had told him.  ``The flight plan has been filed, and your pilot knows what to do.  Here's some background for you to read over while you're over the Atlantic.''  He handed Jesse a thin tablet computer whose only marking was a serial number etched into the upper right corner.

``What's the scoop?'' Jesse asked, running his hand over the tablet.

``You'll find everything on there,'' said the dean.  ``This project is the president's baby, and you probably know more than I do.  When you land, use the switch on the side of the tablet to securely erase its flash memory.  None of the information on it is illegal, but it's sensitive and we don't want you getting tied up in a rumble with immigration.''

``All right,'' Jesse said.  He noted the location of the secure erasure switch and looked back at the dean.  ``Should I pay an extra month's rent, or is this a visit for tea?''

The dean laughed.  ``See you in two weeks, son.''

\chapter{}

The flight was uneventful.  The Astra Jets 1125 was cozy and quiet, carrying only three men.  Jesse had met the pilot, Lars Millsen, at a recent faculty meeting.  Jesse's nervous finger tapping caught the attention of Milssen on one occasion, prompting him to whisper something to his first officer.  A moment later, he was sitting next to Jesse in the front row of the cabin and placing a hot drink on each of their tray tables.  Jesse instinctively steadied the mug, but Milssen put him at ease.

``Don't worry, the bottoms are magnetic,'' Milssen said, gently tipping his own mug from side to side to show the resistance of the magnet.  ``One of the perks of these corporate birds.''

Jesse smiled and took a drink, trying to ease the knot in his stomach.  Green tea.  One of his favorites.  Jesse had already slogged through the 65 pages of minutiae related to the trip, undoubtedly written by the same smiling administrative assistants who handled all of the university's laundry.  He was bound for Warsaw, Poland, a place that he had visited twice for pleasure.  The third visit would be a tense one.

``Ready for duty?'' Milssen asked with a smile.

``Not a chance,'' Jesse said, ``but I'll do what I can.  These kids really stepped in it.''

``The bits of news that have made it to my ear haven't been good.  Be careful, and we'll see you Thursday evening.  Give me a call when you're packed up.  We'll be starting our descent in about twenty.''

Milssen handed Jesse a black prepaid mobile phone, shook his hand, and walked back to the cockpit.

Jesse knew that tomorrow, Wednesday, would be a tense affair.  He kept his nerves hidden, but his bowels never got the memo.  \emph{I won't be doing this until I retire, unless they let me go at 32}, he thought, looking out the tiny window at the hills of central Europe.

The kids had \emph{really} stepped in it.  One student, the son of Polish immigrants, had enrolled in the new investigative journalism program at CUNY with a personal mission to prove that his father's failed bid for a seat on the Warsaw City Council in 1986 had been illegally sabotaged.  The family was convinced that an opponent had floated a rumor about his father's marital fidelity shortly before the election, and that another political rival had abused her position of power at town meetings as a soapbox for making inaccurate statements about the family's intentions.  The student, Sam Rzeznik, had recruited two of his classmates to join him in Poland for a month to investigate the case.

The true story, however, had only come to light after the team of students was arrested by Interpol for breaking into the offices of a local political party and installing recording devices.  The official application for funds had stated that the team would be investigating instances of polling irregularities at a handful of small voting precincts outside of Warsaw.  The low-key nature of the trip, and the delicate issue at hand, had won over the review board and brought praise from William Tweedie.  The team had been awarded a travel and research grant for \$8,000, enough to keep them fed and mobile for 30 days.

It was the voice recorder in the men's room that led to their arrest.  The team had installed surveillance equipment in two conference rooms, the office of a senior staffer, and the men's room.  The argument for putting microphones near the toilets was simple: plenty of informal chit-chat and joking happens when people are washing their hands.  If they could soak up the official line from the board room and grab the raw, unalloyed truth from the loo, then the team had an odds-on chance of landing a good story.

A week after the team had installed the equipment and left the building without raising an alarm, a pipe in the ceiling above the men's room had burst, soaking the voice recorder and flooding two rooms.  The office hired a pair of plumbers to repair the pipes and assess the damage.  The workers had no reason to suspect foul play when they found a waterlogged voice recorder, and they threw it into a strap pile with everything else that had shown water damage.  When the office manager came around to get an assessment from the men and learned about the voice recorder, he immediately dialed the chief of police.  After pulling the tapes from the office's own video surveillance system, it took less than an hour to work out what had happened.

Legally, Sam Rzeznik needed a miracle.  The police and Polish immigration authorities were able to match the images on the video tapes to photos on the students' visa paperwork, making the investigation quick and decisive.  If his case went to trial, the prospects were grim.  The evidence against him was devastating.  His academic career would be over for violating the terms of his admission to the investigative journalism program.  Even if he was acquitted, his extended family who still lived in Poland could face harassment, physical threats, or worse.  The other students on the team were in a better position to negotiate a reduced sentence, but they would still face jail time if convicted.

\chapter{}
									
Jesse tipped his taxi driver and stepped out of the cab onto a busy street in Warsaw.  Opening his umbrella, he looked up at the massive stone columns that flanked the entrance to the historic courthouse.  To his right was a series of shops that included a barber shop, a Greenpeace branch office, and a small coffee shop that had a cupcake protruding from the wall above its traditional-looking iron door.  Checking his impression against the description written on the napkin in his hand, he convinced himself that it was the right venue.  Jesse had no idea what to expect for the next two hours, but William Tweedie had sent him for a reason.

The cafe was busy but had a very European charm.  Jesse unbuttoned his overcoat, walking slowly toward the rear of the cafe as his senses took in the environment.  At the bar that lined the left side of the cafe, a woman was animatedly telling the bartender something about a bracelet that she was wearing.  Next to her, a bearded man in a suit turned to stare at Jesse as he walked past.  A waitress brushed against Jesse's arm, startling him and prompting terse apology.  On the other side of the cafe was a set of staggered four-seat tables, and along the back wall was a pair of high tables with two seats each.  At one of the high tables was a young-looking man with wiry hair pulled back into a ponytail who was reading a newspaper.  He was dressed in an expensive dress shirt and had draped his sport jacket over the back of his chair.  As Jesse approached, the man glanced up from his newspaper and looked him over.

``Jesse Winter?'' the man said.  His voice was clear despite an accent that Jesse couldn't place, but the accent didn't concern him as much as the lack of friendliness.  Jesse reminded himself that he shouldn't expect to make friends today.  \emph{Time to go to work}.

``Pleased to meet you.  Mr...?'' Jesse said, offering his hand.

The man stood up, shaking Jesse's hand.  ``Jasna.  Marcel Jasna.  You may call me Marcel.  May I call you Jesse?''

``Sure,'' Jesse said.

``Shall we sit, then?'' said Marcel, gesturing to the empty chair across the small table.

Jesse pulled off his overcoat, draping it over the back of his chair, and sat down.  Immediately, a waiter was beside him to take his drink preference.  Recalling that Marcel had a glass of scotch on the table, he ordered a gin and tonic.  Stretching his shoulders, Jesse turned back to Marcel and offered a disarming smile.

``Well, Jesse, it's no secret why we come here today,'' Marcel said simply.  ``The manager of my brother's political office caught your students in his building.  It was clear to us that they were spying on our meetings and attempting to sabotage our campaign.  We have video and audio evidence from the night when they entered the building and left recording devices.  This is, as you say, an open and shut case, don't you think?''

Jesse opened his mouth to respond, but Marcel had paused only for rhetorical effect.  He went on: ``My brother has no interest in ruining the lives of young people.  But he finds your program that encourages students to become foreign spies for the United States to be distasteful and shameful.  If students go to jail for a time in order to bring the proper shame on your program, very few Polish citizens here will pay attention.  Your program is unpopular in your own country, yes?''

This time Marcel stopped to wait for a response, but Jesse sensed a trap.  ``The program enjoys strong support from many quarters,'' he offered.

``Yes, there are always those who see public servants as corrupt by definition.  Have you noticed that foreign politicians are always corrupt, but your government does not look at its own face in a mirror?''  Marcel took a drink from his glass, then cleared his throat.  ``This is not so simple as you think, Jesse.  You are thinking of your students, their future, their families, your president, perhaps your job.  We understand that it is far deeper.  Each time our politicians are painted as corrupt, Poland suffers.  We have suffered under the heel of such injustice for most of a century.''

Jesse shifted in his seat.  ``Marcel, these students lied to us.  I understand your position on spying, and I know that from your vantage point we seem to be training a new crop of young spies.  You need to know that these students were not authorized to conduct themselves in this way.  They lied on the application, and they lied to the review committee.''

``This does not change substance of the situation.''

``That's correct.  But it changes the details of the situation that you and I care about.  We're not here to debate state philosophies on espionage.  This is about the welfare of a group of kids who, while they showed themselves to be overzealous, have passion and talent that \emph{must} be put to productive uses.  Do you see?''

Marcel responded immediately.  ``With respect, the fate of the students is beside the point.  This is an issue of national and political sovereignty, and we intend to treat it as such.  I must admit that I do not think we will reach a settlement.  We must not appear weak in the face of interference from America, even if the face is that of a young spy ring.''

The waiter returned with Jesse's drink, taking the unspoken hint that the men were not interested in ordering a meal.

``Our university is prepared to work with you, sir,'' Jesse continued.  ``We have no interest in inflaming the relationship between the U.S. and a friendly nation such as Poland.  We know that despite your feelings about espionage, it would not bode well for your campaign to be associated with the arrest of a Polish student.''  Turning his palms upward in a conciliatory gesture, Jesse added, ``What options remain on the table?''

Marcel leaned back in his seat and tilted his head to the right, clearly wondering what Jesse had been in Jesse's mind as he spoke those words.  It was a moment before he spoke again.  ``You offer us money, Jesse?''

``We are prepared to cover the costs of the repairs to your brother's office, as well as any legal costs that you've absorbed.  Beyond that, well . . . this is a negotiation.''

``Please excuse me for a moment, Jesse,'' Marcel said, rising to his feet and walking toward the back door of the cafe.

\emph{Time for a side-conference during the powwow}, Jesse thought to himself.  He took the moment to enjoy a taste of the gin and tonic in front of him that had so far been ignored.  As the strong flavor filled his senses, Jesse talked himself through the next few minutes of his conversation.  The man in the other chair suspected that Jesse was out of chips in the negotiation.  Jesse was certain of that.  Jesse also knew that his own case was weak and that offering money for repairs and cooperation was small potatoes to these guys.  Someone who cares about national and political sovereignty wasn't going to let a group of foreign kids with an agenda go without making a point or making some demands to make the American politians squirm.  And Jesse was just a pawn in all of this.  He knew that Marcel felt no personal sympathy for him.

As long as the guy didn't pull a gun, Jesse could handle this.  Several years of dealing with stuffy, arrogant university administrators had given him a thick skin.  He took another drink.  The cafe had grown louder as the lunch crowd began to assemble.  Still, the footsteps on the hardwood floor were distinct.  The conference was over.

``My apologies, Jesse,'' Marcel said as he took his seat.  ``My brother can be very detailed.''

``Not to worry.  How is he doing?''

``Very busy these days.  This issue with the students has taken a lot of his time, and he is eager to wash his hands of them.''

\emph{Didn't expect that}, Jesse thought.  \emph{But he's not finished}.

``However,'' Marcel said, leaning back in his seat and slowly turning his glass, ``a detail has come to his attention today.  One of the students, Sam Rzeznik.  You know of him?''

Jesse nodded.

``Yes,'' Marcel continued.  ``It seems that his family has a history.  A history that complicates things for us.  What do you know about this?''

``I know that Sam is of Polish descent,'' Jesse said.  ``He mentioned in his interview with our review board that his grandfather was a public servant, but no serious problems came up during the proceedings.  Everything that Sam told us was verified before he was allowed to start his research project.''

That much was true, but Jesse wasn't revealing his hand just yet.  The political problems had not involved Sam's grandfather, who had been involved in municipal government for most of his life.  Sam's father, on the other hand, had found himself the victim of circumstances entirely beyond his control.  The review board that CUNY had convened to consider the applications submitted by the investigative journalism students had learned about Jesse's father's political past after conversing with Polish authorities, but the board had given its stamp of approval after reviewing the facts.

``Yes, his grandfather served in the city government with my grandfather.  Those were very good days for Poland, despite the problems of the world.  It was the next generation that began the divisive era of our politics that we still see today.  The dishonesty.  The posturing.''  Marcel paused, looking pointedly at Jesse.  ``And the \emph{spying}.  It is this generation that Sam's father helped to assemble.

``My father was in his third year on the city council when Sam's father declared himself a candidate for the same seat.  My father had worked for ten years to clean up this city, and to restore pride to Warsaw citizens.  He had done every task to match the letter of the law.  His constituents trusted him.  Sent him letters of thanks each year.  He knew the spirit of the people, and knew how to bring money and business to Warsaw.  He understood the real purpose of a public life.''

Marcel took a deep breath, clearly incensed by the memories that he was bringing to mind.  ``When Sam's father declared himself a candidate, he . . . he knew nothing of politics.  He knew nothing of campaigns.  His only public achievement was to manage the local library, and poorly at that.  The library was never lower on funds.  He began to attend city council meetings, asking impossible questions.  There was never a more irritating man.  He knew nothing of the business of government, but he made it his mission to lecture everyone on what he called the `proper order of the gathering'.

``It was known that he saw visited women other than his wife.  After he declared his candidacy, this secret became known to the public.  It is not clear how this information became known.  He never recovered from the stain on his reputation, and he lost the election.  It was soon after that his family left the country, and we see now that he reaches through his son in an attempt to blame us for his past mistakes.''

Marcel paused to finish his scotch.  ``There will be no deal, Jesse.''

``Are you familiar with the phrase `catch-22', Marcel?''

``Yes, of course.  It is a situation in which every available course of action carries negative consequences.  Why do you ask this?''

``Since you are familiar with your father's political story, you might remember an incident involving an old woman in June of 1986.''

Marcel's face clouded, and he shifted in his seat.  ``Yes,'' he allowed, ``I am familiar.''

``If I told you that we had an voice record of that conversation, would this change the substance of our discussion?''

Jesse could sense the man's tension, and he didn't admire the mental quandary that he was inflicting on someone he had just met.  He probably believed honestly that the popular account of his father's involvement in the story about the old woman was a deliberate smear.  It was also clear that he had not expected Jesse to have the ace, much less to throw it now.

Marcel considered Jesse for a long moment.  ``You attempt to deceive me.''

``It is not your claims, sir, but the way that you abuse the trust of the people with one hand while smoothing their feathers with the other that make me doubt you,'' Jesse said, making it clear that he was uttering a quotation.  The old woman had spoken with an east Warsaw accent.

Marcel stared at him.  He would have remembered that decades-old quote.  Just when Jesse was sure that he had collected his thoughts and had worked out a stinging train of logic, he took a deep breath and pulled a cash clip from his pocket.  He placed three Euro notes on the table before replacing the clip in his trousers.  Then he met Jesse's eyes.

``Your students will meet you at the airport tomorrow, Mr. Winter,'' Marcel said.

Marcel was out of his seat and halfway to the door of the cafe before Jesse fully processed what had happened.  The students were absolved of their legal worries.  An impossible situation had been turned on its head.  But there was no way that this was the end of the story.  These guys didn't quit that easily.  Jesse's trump card had ended the contest, but he knew that he would hear from Marcel again before he retired.  \emph{Unless I retire at 32}.


The next day at the airport, Jesse stepped out of his taxi and was ambushed by a phalanx of relieved CUNY students.  It was obvious that they had all been crying.  Jesse reassured them that the university was concerned about their behavior but had committed to do everything it its power to protect them from unfair litigation.  That would be much easier once they were home, and Jesse wasted no time in moving them through a series of security checkpoints and onto the private jet.

The students were asleep before the jet had taxied to the runway.  Lars and his first officer looked over the exhausted students before nodding to Jesse and settling into the cockpit.  The jet would need to refuel in Iceland, he was told, and they would be home by morning.  Then the real questions would begin.  The review board would be reconvened to look for any clues that someone knew more about Sam Rzeznik's family than what was discussed.  The press would be there to get the scoop on the students and take their stories.  The FBI and the State Department would want to know everything.

Jesse had a feeling that this would be the best sleep that he would have for a few days.

\chapter{}

In the nine months that followed the Warsaw trip, Jesse heard little about the case once the furor of the press moved on.  The much-anticipated renewed public backlash against the investigative journalism program never arrived.  The CUNY administration was visibly relieved, but they all had to wonder when the other shoe might fall.  Were Marcel and his brother working behind the scenes to put together a counter-spying team?  Was anyone from CUNY already being followed?  It seemed prudent to be watchful and to keep an eye on the brothers' campaign in Poland, but no one wanted to discuss the matter, much less do anything about it.

One Thursday morning, a uniformed university police officer knocked on the door of Jesse's office.

``Mr. Jesse Winter?'' the officer asked, glancing at a sheet of paper in his hand and looking back at Jesse.

``Yes?'' Jesse replied.

``I'm detective Morse from the CUNY police department, I'd like to ask you a few questions, if that's all right.''

``Certainly.  Please come in.''  \emph{This should be good}, Jesse thought, feeling his pulse quicken.

The officer stood near the door, refusing Jesse's offer of a chair.  ``Mr. Winter,'' he said, ``one of the members of the board of directors has launched an investigation into your involvement in the Warsaw incident from June.  Are you familiar with that case?''

``Yes, I traveled to Poland in June to negotiate on behalf of the university.  What's this about?''

``Would you mind coming down to the station to talk about it?  We can be sure of our privacy there.''

``Am I under arrest, detective?''

``No, sir.  We just want you to discuss a few things with us.''

``This office is private.  Why not stay here?''  Jesse felt his old privacy hackles starting to rise.  \emph{My office is certainly more private than an interrogation room}.

``With all due respect, Jesse, given your involvement in this case you should know better than to assume something like that.''


Jesse's visit to the CUNY police department was short and terrifying.  He learned that one of the board members, Stewart Rheingold, was investigating him for professional misconduct related to the Warsaw negotiations.  It was improper and against the law, the claim went, that Jesse made implicit legal threats against foreign nationals in order to secure the release of students who had broken Polish law.  Rheingold had been a vocal opponent of the investigative journalism program from the outset, and his position on the board of directors made him privy to the internal operations of the program, including Jesse's job description and the details of his work.

Still, Rheingold was not in the loop regarding what happened in Warsaw.  Only Jesse's immediate superior, the Dean, had known what transpired.  The Dean was a trusted associate who knew the high stakes for that trip.  There had been other instances when reporters had been circling like sharks, looking for a juicy story of undercover deals that they could turn into an exclusive.  The Dean had been gracious but firm, refusing to share details or to implicate Jesse.  The information had leaked another way, but it would be days before Jesse understood how.

The detective was as helpful as he could have been without breaching protocol.  He knew that Rheingold had a personal axe to grind.  When Jesse asked what the next few weeks would bring as the case proceeded, he was told that the department wouldn't continue with the investigation unless Rheingold could produce more evidence than a hunch.  Detective Morse couldn't share the details of the charges that had been filed, but he made it clear that this would be an uphill battle for the plaintiff.


When the trial dates were finally set, the details of the case against Jesse were foggier than ever.  Due to the confidential nature of the case, the evidence submitted by the university's prosecutor was sealed in an envelope for the judge to view privately with both attorneys.  Jesse had hired an attorney from a local firm that specialized in privacy and first amendment litigation.  At 50, the guy had been around the block more than once, and he seemed the type to be cool under pressure.

The case against Jesse was surprisingly weak, but the university prosecutor had constructed an elaborate argument that had a chance of succeeding.  If he could convince the judge that Jesse had issued a threat, even an implied one, then Jesse could be found guilty of obstruction of justice.  His attorney seemed confident that the judge would reject the argument, but its plausibility made Jesse nervous.  If the key question was whether Jesse issued a threat, then he was in a difficult position.  He \emph{had} issued a threat, although it was not spoken.  Anyone who had been privy to the conversation in Warsaw would have understood that what Jesse said was a threat.  But the thing gnawing at Jesse's mind was how that information had made its way to the ear of one of the board members.  One of two things was true.  Either Rheingold had secured a copy of the documents on the tablet computer that Jesse had carried to Warsaw, or Marcel and his brother had made a contact in the United States.  The latter worried Jesse on a much deeper level than the former.

On the final day of the trial, Jesse sat alone while his attorney consulted with the judge and the prosecutor.  He found himself thinking of his parents, wondering what they were doing at that instant.  He had told them about the legal issues but reassured them that it was nothing serious.  The case had not gone to the papers.  They were probably enjoying a nice breakfast at home.  His mother would be heading outside to weed the garden, and his father would be walking his collie or preparing a lecture for his classes.  They wouldn't be pleased about the predicament that Jesse was in, but they would understand and support his decisions.  He knew that much.

When the judge came out to issue the verdict, Jesse's attorney gave him a reassuring smile.

``Will the defendant please rise?'' the judge drawled.

Jesse stood up from his wooden chair, straightening his jacket.

``In the case of Rheingold versus Winter, we find the defendant not guilty on the charges of obstruction of justice.  Mr. Winter, you are free to leave.''  After a brief pause, the judge cleared his throat and added: ``Off the record, I think that this case was a poor use of the court's time.  Good day, gentlemen.''

Amid some scattered applause and hushed chatter from those around him, Jesse slowly processed the verdict.  He was cleared of wrongdoing.  But the gnawing feeling that something was left unresolved did not budge.  The nature of the trial evidence was sketchy, and the judge seemed reluctant to broach the topic during a public hearing.  This had been Jesse's first experience of a trial, but he judged from the body language of the attorneys and the judge himself that the more experienced men found the proceedings unusual.

After a final debriefing with his attorney, Jesse left the courthouse and walked toward his car.  As he fumbled in his pocket for his keys, his head spinning and reliving moments from the trial, he felt someone sock him on the shoulder.  Instinctively stepping away from the man who had appeared beside him, Jesse slowly recognized a familiar face.

``Good show, Jesse,'' Sascha Greene said with a wink.

``Hey, Sascha,''  Jesse said.  He smiled and heaved a sigh of relief.  ``Were you in there?  I didn't see you.''

``Yeah, I wanted to see what the university had up its sleeve.  I guess we saw that logical gymnastics is their main game.  How do you feel about it?''

Jesse shook his head.  ``I don't know.  Obviously, I'm happy that they didn't end my career.  I'm happy that the judge had a sense of humor.  I learned a lot about the politics of our board of directors.''

``But\dots?''

``Well, something doesn't add up.  Rheingold shouldn't have known anything about what happened in Warsaw.  All the press knows is that the university negotiated a deal and kept the students out of hot water.  And that was the official story from the university.  How did he know?''

Sascha nodded, looking over his shoulder for a moment.  ``Let's get some lunch.  My treat.''

Jesse shrugged.  ``All right.  I'm starving.''


``You look like you haven't slept since Warsaw,'' Sascha said, dipping a shrimp in his cup of ranch salad dressing.

``You'd be surprised how close you are to the truth.''

``Have you ever met Stewart Rheingold?''

``I've seen him at a few board meetings.  He likes to talk.''

``Unfortunately for you, he likes to act too.  The guy is a modern day Joe McCarthy.  His big issue is national autonomy.  If anyone mentions the global economy, he starts frothing at the mouth.  I saw him take a senior professor to task because of the tie he was wearing.  Something about the supply chain that was involved and the China's currency tactics.  Anyway, it's probably nothing personal.  He saw an opportunity to teach us all a lesson and make an example of the school that's supposed to be under his supervision, and he pounced.

``The part that really has you wondering, though, is how he sweet talked his way into a copy of the materials that you and your boss thought were private.''

Jesse looked up from his plate, studying Sascha's face.  ``That's right.  Although that's only one possibility.  The guys in Warsaw can't be happy about how things went down.  They might have tipped off Rheingold about what was said.''

``It's possible,'' Sascha said, ``but let me tell you something.  Yesterday when I was leaving work, I overheard an interesting conversation in the elevator.  One of the administrative assistants for the Department of Financial Aid was telling a her co-worker about a document that she had retyped and converted into digitally signed and encrypted format.  Not too many documents get that treatment, so it piqued my interest.  She mentioned that the topic was a trip to eastern Europe.  She also knew -- and this is the key -- that the university was planning to exercise an unorthodox strategy to secure the cooperation of the Polish guys.  Something about the students' parents.  She didn't give details, but it sounded complicated and legally questionable.

``That much by itself isn't interesting.  The administrative assistants know a hell of a lot about what happens on campus, but everyone knows that.  Here's the rub, Jesse.  The woman in the elevator is good friends with Rheingold's wife.  They vacation together each year.  When the paperwork for your trip got processed and the students were released, it would have been easy for the administrative assistants to put the pieces together.  The trip could have come up in casual conversation, and Rheingold would have been very keen to hear the details.''

Jesse nodded, letting the new information percolate.  ``We'd better hope that there are no more lawsuits like mine.  Eventually someone will find a team of lawyers that can bring down the whole program.  They made a reasonable case against me, and I feel like the judge let me go on a technicality.''

``Hardly,'' Sascha said though a bite of his sandwich.  ``The judge made a reasoned decision.  Any evidence that you made a threat would be hearsay or circumstantial.  He couldn't convict you on that.  Remember, Jesse, you did the right thing.  Those guys picked their strategy twenty years ago, and you know how it turned out.  They've been in damage control mode ever since.''

``Thanks, Sascha, but the fact that they're in damage control isn't comforting.  What if they still see us as a threat?  Taking me down could be revenge, or it could be a simple matter of neutralizing a threat.''

``If they wanted to neutralize the threat, they would have gone after your bosses.  You're a talented guy, Jesse, and the university will miss you when you move on.  But in a more global sense, and to these guys in Poland, you're just another guy, and CUNY could replace you in an instant.  It's the program, and the fact that we admitted a bunch of students whose families have a policital axe to grind, that are the real threat.  I hear you, man, but I think your worry is misplaced.''

``It's good to hear that from someone I trust.  I've been trying to convince myself that I was just being paranoid.  The trial has completely consumed me lately, and I think it's made me a little unbalanced.  Getting back to work will do me a world of good.''

``Get to it, then,'' Sascha said, finishing his last bite of pickle and dropping a few bills on the table.  Let me know if you need anything.  And I mean \emph{anything}.''  He gave Jesse a significant look.

Jesse nodded.  ``If I hear footsteps, you know I'll be in touch.''

\chapter{}
										
Sascha closed the office door behind him and headed toward the north elevator.  He jabbed the call button and took a moment to look out the window.  Outside, a group of students clad in bright red and yellow shirts were playing a game of touch football.  The sun was shining, giving the false impression that it was a balmy summer day.  \emph{Jesse is probably wishing that it were summer again}, Sascha thought.  Could Jesse have been right?  He couldn't imagine that the university would try him again for something related to his trip to Poland, especially after the last trial.  Espionage and destruction of property.  Two loosely-defined but very serious charges for a young professional.  Sascha knew that the university was capable of eating its own.  That had always bothered him.  Some schools had a superb legal posture and would go to bat for members of their communities to the greatest extent that they could.  But not CUNY.  When students got in trouble for sharing files on their computers, the university would forward the Cease and Desist notice directly to the student who was using the network address that was implicated in the document.  This was poor form for the university, since a student could be falsely accused if someone happened to be the victim of a virus or worm, or if someone had used a weak password to protect his wireless network.  Guilt, intent, and liability were very complex matters to sort out in a place as technologically dense and active as a university campus.

In this case, though, Sascha thought that it was bad business for CUNY not to step up and defend Jesse.  He had been cleared of wrongdoing in the Warsaw trial.  It had become clear afterward that William Rheingold was motivated by political and personal concerns, and not because of any moral obligation.  Jesse had dealt with many other concerns in the meantime, but Sascha hadn't heard any more gossip from the press or the ladies upstairs.

The elevator arrived, and Sascha stepped in.  He punched the button for the fifth floor.  \emph{Jesse will be in, and he needs to hear about this before the police come and start asking questions again}, Sascha thought.  He heard the chime indicating arrival at the fifth floor, stepped out, and froze.  The door to Jesse's unmarked office, which sounds an alarm if it remains open for more than one minute, stood ajar.  The lights were off, an odd thing at 10:30 on a Monday morning.

Sascha approached the office slowly, pulling his phone from his shirt pocket.  He leaned to his right, looking around the door.  He could only see the corner of the office, making him wish that the door had been reversed.  Sascha had been a U.S. Marine before college, and his tactical disadvantage on this side of the door made him uncomfortable.  Backing away and glancing at his phone, he dialed Jill's number.

``Jill Cumberland,'' the pleasant voice said.

``Jill, it's Sascha.  Is there maintenance in 503B scheduled for today?''

``503 beta?'' she asked, sounding surprised.  ``Isn't that the investigative journalism war room?''

``Yeah,'' Sascha replied, glancing down the hallway.  ``Anything on the grid?''

``Two seconds here,'' Jill said.  Sascha could hear her fingers working her keyboard.  There was a long pause.  Then she spoke again, this time with slight concern.  ``This is odd, Sascha.  There is nothing on the schedule, but someone added a ticket for 503B and then deleted it.  Is everything all right?  Do you want me to call security?''

``I'm not sure.  Don't call security just yet.  The door is ajar, and the lights are off.  I'm going to have a look around and make sure that everything is kosher.''

``All right.  Be careful, Sascha.  I'll send someone down there if I don't hear from you in a few minutes.''

``Give me ten,'' Sascha said, and quietly flipped his phone closed.  His pulse had quickened.  The information that Jill had given him, thought it probably seemed suspicous, would seem ultimately inconsequential to anyone else.  To Sascha, though, the news was unwelcome.  He knew the scheduling system very well, and he knew his building.  Someone had scheduled maintenance for the room where Jesse normally worked, and someone---probably the same someone---had cancelled the service ticket.  By itself, that could be attributed to a mistake by the person scheduling the ticket.  With the door ajar on a secured office that was normally occupied during the day, the situation changed.


The CUNY ticket scheduling system was complex once one looked behind the pleasant computer interface that the maintenance workers and administrative assistants used.  It synchronized its database with servers that handled service requests for the university at large, enabling reassignment of jobs when one maintenance crew was overloaded, as well as distributing the work of creating tickets.  Each building had to be inspected by the local fire chief at least once every three years, for example, and the university as a whole had to be fully reinspected every five years.  Requiring the staff of each building to coordinate with the fire chief would be time consuming for all parties, and so the administrative staff for CUNY negotiated the inspections for the entire campus.

Like most useful digital tools, the scheduling system cut both ways.  The flexibility and division of labor that was won by using distributed ticket creation also meant that tracking an errant ticket was nearly impossible.  There were hundreds of workstations across the campus, and many faculty and staff left their computers turned on and signed in when they left for the day.  The two minutes of waiting that they saved each morning left many sensitive resources at the mercy of anyone who didn't mind picking a lock, breaking a window, or scaling the wall of a building.  The deleted ticket would include the name of the person whose computer was used to create it, but that might be useless for tracking the real identity of the person at the keyboard.

There was one other thing on Sascha's mind as he slipped the phone back into his pocket.  The ticket scheduling system was designed to interface with the door security system.  This was a convenience feature for maintenance workers who would become annoyed when the office that filed a maintenance ticket was protected by an elaborate security system that would reliably go off during their visit, leading to an automatic call to the police.  In order to save the university a few thousand dollars in police fines and lost productivity, the board of directors voted narrowly to install an override system and, regrettably, to give the scheduling system the ability to trigger it.

The political details of the override aside---it had been a very messy, heated discussion on campus---there was also the not-so-small matter of a bug in the ticket scheduling system.  CUNY had hired a private software firm to develop the software for the system.  When it was decided that the scheduling system should interface with the door security system, the same firm had made the changes.  But since the change was made as an afterthought, five years after the software was developed, it was difficult for anyone to recall the details of its design.  There were small details that were neglected, and periodically a consequence of those details would crop up.  For their part, the software firm had been helpful in troubleshooting and resolving difficulties as they arose.  This one, however, had gone unresolved for months.

When a new maintenance ticket was created, the ticket system sent a secure message to the door security system that marked a room, floor, or building as a maintenance zone for a period of time -- normally the length of time specified on the ticket.  When the ticket expired, security was restored to its normal operation.  About a year before, however, Sascha had noticed a glitch.  After hours on a Thursday night, he had gone to the second floor to pull a file on a student who was under investigation for fabricating evidence for a research project.  This involved going into the records vault, which is a large space that is normally well-protected by security cameras, doors with card-key access and retinal scanners, and black-out times when the doors would not open.  On this Thursday night, Sascha found the door unlocked.  After calling security and waiting for an agent to arrive, he checked the vault and verified that everything was in order.  It didn't appear that anyone had been in the room that night.

After saying good night to the security detail, Sascha had gone back to his office to check the ticket system.  Maybe someone scheduled maintenance and forgotten to follow through, or maybe it was scheduled for the wrong room.  When Sascha pulled up the ticket system and checked the record for the vault, his blood ran cold.  There had been a ticket created for the vault so that a technician could install an upgraded ethernet switch as part of a campus-wide migration to a faster network architecture.  The next line in the record, though, showed that the ticket had been deleted several days later -- 2:18 p.m. on the same day that Sascha had gone to the vault.  The reason given for the deletion appeared to be that the technician had called in sick, and there was no one else available to do the work.  \emph{If the ticket gets cancelled, the door security system still opens the room}, Sascha had thought, his mind reeling at the thought.  \emph{This is almost as bad as the power to the raptor cages going out in /Jurassic Park}\dots.

When Jill told him that the schedule showed a ticket that was created and immediately deleted, Sascha felt his heart start to hammer.  Seeing the door ajar was enough to activate his old Marine instincts.  Having backed away from the door to make his call, he now crept back toward it, a small canister of pepper spray in his left hand.  The only sound coming from the room as he moved silently along the wall was a rhythmic rustling of paper and the sound of an old dot-matrix printer.  Sascha reached the door and began slicing the pie, an old technique that he had perfected in basic training.  When a team of Marines needed to silently clear a room, they began with one man on each side of the door, guns drawn and shouldered.  This gave each man a sliver of the room to examine.  Next, they would turn slowly away from one another, increasing the portion of the room that they could see.  When the entire room had been examined from this position, they would signal to their comrades that it was safe to proceed.  Today, though, Sascha had to rely on his intuition.  The wall gave him an advantage and a disadvantage.  The bulk of the office would be to his right, making it easy to slice the pie.  However, he would need to open the door in order to see the room.  If its hinges squeaked, he would be partially exposed, and the walls were not designed to stop or deflect gunfire.  \emph{It's just an office}, he reminded himself.  \emph{This is a university in the U.S. of A}.  With his right foot resting against the door frame, Sascha peered around the edge of the door.  There was an empty desk against the wall on his left, and a bookshelf on the far wall.  Above the desk was a steel matrix of time cards and pamphlets.  In the corner was a black, non-descript waste bin.

Sascha took a deep but quiet breath and tugged on the door.  It was a heavy door, but it gave way slightly at Sascha's touch.  No squeaks.  He still couldn't fit his body between the door and its frame, but pulling it open had given him a better view of the room.  It was much narrower and deeper than he expected.  There was another desk on the right wall, and there appeared to be a stack of dark coats draped over the office chair next to it.  Near the back of the chamber, Sascha saw a computer workstation with four monitors attached to it.  The computer was running, its screens aglow, but no one was using it.  \emph{Odd}, Sascha thought and he scanned the room for movement.

Seeing nothing suspicious or out of place, Sascha opened the door enough that he could slip into the room.  When he did, he stopped short.  What he thought was a pile of coats was now clearly a man, and he was slumped over onto the desk in front of him.  Sascha approached closely, not immediately recognizing the man.  There was no sign of a struggle.  As Sascha reached out to put his hand on the man's shoulder, the dark mass inhaled sharply and sat up, turning his head toward Sascha.  Instinct taking over once again, Sascha backpedalled, raising his pepper spray into his line of vision.  ``Who are you?'' he ordered.

The man stared at him, clearly confused.  The low light on the room made him appear pale and green, almost alien-like.  ``Sascha?'' the man said, his voice rough from sleep.

Sascha found a light switch beside him and flipped it on, his pepper spray still aimed at the other man.  As the lights came on and his senses sharpened, Sascha dropped his pepper spray and rushed forward.  ``Jesse.  Are you all right?  Did someone break in?''

``Are you kidding?  I was taking a nap, man.  I've worked three nights straight.  Are \emph{you} all right?  You look fit to kill.''

Sascha exhaled, allowing himself a nervous chuckle.  ``Boy, Jesse, do I have a story to tell you.  First, though, I'm going to lock the door.''

Jesse frowned, looking past Sascha at the heavy steel door that stood open.  He rubbed his eyes, his brain not up to speed just yet.  ``It was already locked.  It's always locked.''

``Not this time.  I think I know why, too.  Grab your coat.  Let's take a quick walk.''

Sascha pulled out his phone, dialled, and told Jill that everything was cool, at least for now.  Then he walked over to the computer terminal, loaded the program that controlled the security for the room, and pushed \texttt{RESET}.  Immediately, he heard a click and a dull thud from the far end of the office as the door lock engaged.

\chapter{}

Jesse felt like his legs were made of cement.  Sascha was the fitter man of the two of them, and he was walking at a pace that reflected the speed of his brain at that moment.  They were walking between a line of tall hedges that led from the building where they both worked to the CUNY cafeteria.  ``We're not going for lunch,'' Sascha said.  ``We're just going to find a place that's noisy and can give us same privacy.''

Jesse puffed his cheeks and willed his legs to keep up.  The long hours at the office had been taking a toll on his fitness, but this was simply unnecessary.  ``Sascha, why do we need privacy?  Was someone offended that I was asleep at my desk?''

Sascha stopped suddenly, giving Jesse a hard look.  ``Check your pockets.  And your collar.''

Jesse looked momentarily puzzled.  ``Wha-'' he began.  Then understanding appeared on his face.  Jesse checked his breast pocket, then his pants pockets.  Nothing.  He turned up his collar, running a thumb along the crease.  As he passed the back of his neck, his thumb caught something hard and plastic.  He glanced up at Sascha, his eyes wide.

``Show me'', Sascha whispered.

Jesse turned around, flipping his collar open so that Sascha could see the device.  Sascha put his face very close to Jesse's back, trying to work out how to remove the bug without breaking it.  After a moment, he put his mouth next to Jesse's ear.

``Put your collar down and follow me,'' he whispered.

The two men walked into the student cafeteria.  They looked distinctly out of place in their jackets and slacks, but the students paid them no attention.  Sascha gestured toward the kitchen.  Jesse followed him through the swinging doors.

``May I take your jacket, sir?'' Sascha said, putting on his best impression of a waiter.

``Certainly.''

Sascha took the jacket from Jesse and turned on the sink.  He opened the collar, exposing the bug, and quickly submerged it in the water that had collected.  After the water had time to soak into the bug, he pried it from the fabric.  Sascha shook his head, turned the sink off, and handed the jacket back to Jesse.  Then he yanked the stopper out of the drain, sending the bug into the bowels of the university.

``After you, good sir,'' Sascha said with a self-satisfied grin.

Jesse allowed himself to laugh before giving Sascha a fist bump and walking back into the cafeteria.

``All right,'' Jesse said, settling into a booth.  ``I'm awake now.  What the hell is going on?  I was asleep, and then I was bugged, and now I'm in the student cafeteria at a table with a short leg.''

``I know, it's been a whirlwind,'' Sascha said.  ``This won't make sense unless I give you some background.''

Sascha told Jesse about the problem with the vault's security.  He went into detail about the scheduling system.  As soon as he mentioned that the security doors unlocked despite the fact that the event had been cancelled, Jesse groaned and hung his head.  Jesse had debugged enough code in his life to recognize a programming error when he saw one.  This wasn't the type of error that would crash the system, but in some ways it was worse than that.  This was a logic error.  The programmers who had designed the module that was responsible for informing the security system that a ticket had been cancelled had neglected to wire up that bit of code properly.

``How long have you known about this bug?'' Jesse asked.

``Months.  I reported it the day that the vault was unlocked, but nobody seems to know when it will be fixed.  Some of the programmers are on vacation, our people are busy with other things, and it only happens on rare occasions.  You know the drill.''

``Yeah.  I bet it would get done faster if the president's office had been the one unlocked.  Who else knows about this?''

``I reported it to the director of I.T.  I have to imagine that his deputies are aware of it, but I don't know.''

``Those guys are all underpaid for what they do, including the director.  It makes me nervous under the best of circumstances.  And now they have a huge security hole at their disposal until it gets fixed.  It's a ripe situation for someone to steal records, modify personnel files, or--''

``To bug a staff member who works on high-profile cases.''

``I can't figure it out.  Who would want to spy on me?  We don't have any cases in the queue right now.  The most that they could learn is what the students in Brazil are learning about how the fish markets influence fuel production.''

Jesse's thoughts turned back to the morning.  He had been doing paperwork for the students in Brazil and Venezuela when he fell asleep.  At that time, the door had been locked.  No one else had been around all morning.  Whoever came into the office had free reign.  Jesse had looked through the office after Sascha woke him up, and nothing looked out of place.  \emph{Neither did the bug in my collar, though}, he thought.

``I wish I knew, Jesse.  At the very least, we need to look at the security tapes and figure out who was in your office.  In the meantime, let me make a suggestion.  Grab your papers from 503B and find another office for the week.  I'll take a look at the room assignment chart and see what's available.''

Jesse sighed, looking out the window at the growing shadows on the lawn.

\chapter{}
				
Sascha sat next to the security officer, staring at an array of computer monitors.  They had trawled through sixteen hours of closed-circuit TV footage, trying to find evidence that someone had entered Jesse's office.  Luckily for the hunters, there was little traffic on the fifth floor.  Unfortunately, their prey had eluded them up to this point.

``There!'' Sascha called out, pointing a finger at a monitor labeled with the numeral 8.  ``Who is that?''

Jesse stood behind the others, leaning forward and squinting at the grainy screen.  ``Oh, that's Tammy from upstairs.  She brings me the documentation on new student cases.  I didn't see her today, though.  What's the timestamp?''

The security officer pushed a few buttons.  ``Eleven fifty, sir.  Just before the top floor goes to lunch.''

``Just after I fell asleep,'' Jesse said, sounding sheepish.  Then he suddenly perked up.  ``But she would have noticed that the door security was turned off.''

``Looks like she keyed in her passcode and opened the door,'' the officer said.  He rolled the tape forward by a few seconds.  ``For a lot of folks it's a rote process.  They know that the numbers open the door, and the details are someone else's problem.  That door should have closed itself when she left, though.''

Jesse had a brief flashback to the day when the heavy steel door was installed.  The technicians had pulled three large magnets out of a crate, mounting them inside the door frame.  They had spent several hours installing a new wiring block and upgrading the power supply to the office to support the new equipment.  ``Don't pinch your fingers in this one,'' one of the workers had said with a nasty laugh.

``It's magnetic,'' Jesse said.  ``And I'd bet you anything that it's tied into the security system.''

The officer turned to face him, an odd expession on his face.  ``They installed a magnetic door for a staff office?  Those are normally reserved for vaults or emergency weapons stores, things that need protection from determined criminals.  A heavy spring would have done just fine.''

``I questioned the wisdom at the time, but the higher-ups were convinced that the extra protection was worthwhile,'' Jesse said.

Sascha heaved a sigh.  ``So we know that Tammy was in the office to deliver paperwork and that the security door would have let her in without a passcode.  How long did she stay?''

The security officer rolled the tape forward again.  ``Thirteen seconds, sir.''

``Barely long enough to install any tracking equipment.  There's no way that she bugged Jesse.''  Sascha turned to face Jesse.  ``All right, let's think about this.  Did you leave your coat unattended anywhere in the past couple of days?''

Jesse looked toward the ceiling, rolling his mental tape backward and forward.  He had gone on a disastrous date on the previous Friday.  \emph{Never go on a date when you haven't slept in 50 hours}, he had concluded.  Over the weekend he had gone to his favorite coffee shop in Queens and had slept through a B-list film that was showing at a local theatre.  He took the bus to Queens each weekday morning and then walked to CUNY.  On this particular morning, he had purchased a copy of the \emph{New York Times} and read it on the bus.  The bus was crowded and stuffy, so he had taken off his jacket and, out of habit, draped it over his seat.  The newspaper had occupied him until his stop, leaving plenty of time for someone to mess with his jacket.  The collar would have been fully exposed, and the bustle of passengers always left him numb to small amounts of jostling.  ``It must have been on the bus this morning,'' he said finally.  ``I was reading about the Fed's latest argument for quantitative easing, and I had thrown my coat over the back of the seat.  There would have been ample time for someone to bug me.''

``And ample time for the person to get away unnoticed,'' Sascha added, sounding momentarily defeated.  ``This doesn't bode well.  I had hoped that we would find someone local, someone affiliated with CUNY.  This makes it a much bigger problem, potentially.''

``Mr. Winter,'' the security officer said, ``if you were indeed bugged on the bus, you will need to change your habits.  Tracking someone without being noticed is not an easy business, and whoever did this has probably been watching you for a while.  You don't need to move house unless you suspect that your safety is at risk, but I suggest that you drive to work for a while, or maybe take a cab.  If you routinely go for walks in your neighborhood, be sure to change up your routes.''

``I think I need to move house,'' Jesse said.  He was suddenly feeling ill.


When they left the police station, it was full dark.  Nearly nine hours had passed since Sascha found Jesse asleep at his desk, his soft snoring being broadcast somewhere by an electronic bug on his collar.  It was unclear who had installed the bug, but the two friends had some idea of when, and where, it had been installed.  Unfortunately, there was no video surveillance on city buses, and Jesse had little recollection of anything from the bus ride.  Not enough to go on, as the detectives would say.

Sascha had offered to give Jesse a ride home, saving him the anguish of riding the bus again.  They climbed into Sascha's Volkswagon and made their way out of the parking structure.

``There was one other thing, Jesse,'' Sascha said.  ``Not that you need anything else to think about.''

``What's that?''

Sascha turned on his high-beam lights as he turned onto at empty street.  ``Your name came up on the top feed today.  That's why I came down to your office this morning.''

Jesse looked at Sascha, searching his face for clues about the news that he was about to hear.  ``Again?!  We haven't even processed any cases lately.  How can there be a case against me?''

``It's still early.  The case is being pursued by the Office of the Ombudsman, on behalf of a professor from the Department of Philosophy.''

\emph{A philosopher.  Great.}  ``I wish that people would file these cases against the program.  I've become the de facto lightning rod, and that's completely outside my job description.''  He heaved a sigh, recalling the time lost and mental anguish of the first trial.  ``What are the charges?''

``Espionage and destruction of university property.''

Jesse felt ill again.  He had been cleared of wrongdoing in the first trial brought against him for espionage.  There had been no other noteworthy trips or incidents since then, not for Jesse or anyone else affiliated with the investigative journalism program.  The most exciting event in months had been Jesse's meeting with the U.S. ambassador to India to discuss possible partnerships with Indian universities.  That meeting hadn't even received a writeup in the school newspaper.

``I think you'll have better support from within CUNY this time,'' Sascha said, sensing Jesse's discomfort at the news.  ``There's a precedent now, and the prosecutor won't be keen on repeating his last performance.  And that's assuming it even goes to trial.  I haven't heard any of the details or what the evidence is.  We'll know more in a few days.''

The remainder of the ride to Jesse's apartment was silent, punctuated only by the occasional roar of a passing engine and the clicking of the car's turn signal.  Jesse thought about his neighborhood.  About the walks that he took each morning when he wasn't too exhausted to move.  About the lady on the corner whose dog reminded him of Ozzy Osbourne.  About the apartment that he had lived in since he moved to New York after college.  About the tree in the backyard that would be perfect for a treehouse for his kids someday.

It wasn't fair that he would need to move, but Jesse never thought that life owed him anything.  It's amazing that we're here at all, this species of creative primate.  There's no pre-determined order to things.  When you mixed together cause, effect, and complexity and unleashed them in a place governed by a set of physical laws, the result was pretty amazing.  A place full of turmoil and violence and uncertainty, but also of beauty and wonder and possibility.  The world that Jesse lived in was not one to fear, but he recognized that he still needed to protect himself.  Sometimes the wolves came to your door, and it wasn't always your fault.  It wasn't always your blood that they thirsted after, but the thirst often clouded their judgment.  Societies weren't small anymore, and the loosening of bonds between people in the same community made it easier for the wolves to find easy prey.  A random guy at a random university who held a low-profile job in an unpopular department wouldn't be missed.  It would be an easy meal, and one that would send chilling ripples throughout the program.  A message to the young spies in training.

Jesse thanked Sascha for the lift and walked up to his apartment.  He took a survey of the area as he walked and knew that Sascha would be doing the same as he watched Jesse enter the building.  All was quiet, and the night air seemed sweet in Jesse's nose.  He unlocked his door and went inside, flipping on the lights and dropping his keys on the countertop.  He looked around at his kitchen and living room.  This had been the eighth apartment that he toured, and it was perfect.  It was hard to imagine living anywhere else.

Jesse pulled his phone book from a drawer in the kitchen and looked at the listings for professional movers.  He wrote a few of the names and numbers on a scratch pad and put the book away.  It would wait until the morning.  He poured himself a glass of milk and sat in his easy chair in the living room.  The events of the day had made the whole of his life seem surreal.  \emph{This morning I was on my way to an ordinary day at work}, Jesse thought.  \emph{Now I'm under investigation for a second time, and someone finds me interesting enough to listen in on my conversations}.  Having finished his milk, he set the glass on an end table and leaned his head back on the chair, closing his eyes.


Jesse woke in the morning when his alarm clock began singing and squawking about the day's news.  He hadn't moved all night, and his body was quick to inform him that sleeping upright is an acquired habit.  His neck and shoulders ached.  Jesse stood up and walked to the front door to collect his copy of the \emph{New York Times} that was delivered each morning at some obscene hour.  The day's lead story was about political posturing between the governments of Afghanistan and Pakistan.  The investigative journalism program hadn't approved any requests for projects in those two nations because of security concerns, but Jesse knew that there were hundreds of interesting stories to follow.  He knew that they would graduate students who would go on to spend their entire careers in that region of the world.

The kitchen in Jesse's apartment was clean when one considered that a busy bachelor lived there by himself.  Two boxes of cereal sat on top of the refrigerator, and the countertop was clean and empty except for a coffee maker, a toaster, and a knife block.  A large collection of spices and teas were neatly tucked into a face-height cabinet above the sink.  During less busy times at the office, Jesse would be busy preparing a cooked breakfast and steeping a cup of white tea.  Today, though, a bowl of cereal would have to suffice.

After breakfast, Jesse brewed some strong coffee and absently read a few of the headlines in his newspaper.  The question of where he would do his work today occupied him like the flames of a roaring fire.  Sascha had offered to look at the available rooms, but Jesse thought that it might be wise to go off the grid and choose a place that would be less obvious and predictable.  A faculty lounge, or even the student cafeteria, would give a certain amount of peace and would enable Jesse to work incognito.  There was always the problem of overly-curious students or faculty looking over his shoulder, but that was a reasonable risk to take given the circumstances.  He decided that the student cafeteria would be his new office, and he would find a corner that afforded a good view of his surroundings.

The next step in the morning ritual was to shower and shave.  Jesse enjoyed these two steps more than usual.  The full night of quality sleep, his first in over a week, had left him in an upbeat mood.  \emph{I can handle this.  Whatever it is, and whoever it is, I've got this}.  After getting himself cleaned up, he packed up his laptop computer, mobile phone, and his staff badge.  He locked the door and stepped out into the morning air, relieved not to find someone prowling around his apartment.  He walked down the wooden stairs that led to a short segment of sidewalk running between a small parking lot and the community garden where Jesse's neighbors grew potatoes, cabbage, and a number of leafy greens.

Jesse had never been much of a driver, and the morning traffic reminded him why he had always chosen to take the bus.  He was on the road at 7:30 a.m., an hour and a quarter before traffic would be at its peak, but it was more than 40 minutes before he arrived at the east parking structure on the CUNY campus.  Sascha was the more cautious driver of the two of them -- a fact that always surprised Jesse -- and he had delivered Jesse to his apartment from the same parking structure in eleven minutes flat.  This created a complex set of feelings for Jesse on this particular morning, but mostly he was happy to arrive at work without any complications and without any tracking devices in his clothes.

The morning went by quickly.  Jesse had installed himself in a corner of the student cafeteria a few steps from both the men's room and an array of carafes of coffee.  He burned through a stack of paperwork that had accumulated the day before.  The steady chatter from the cafeteria provided a pleasant din that was as soothing as silence.  A few students recognized him from social events that he had chaired over the years, but no one bothered him.

Jesse was packing up to get some lunch when he noticed a new email in his inbox.  The subject heading read, ``Fwd: Escalation of service ticket related to door security in the Sheckell Building''.  It was from Sascha, but the original author appeared to be an executive at the software firm that was hired to build the CUNY ticket scheduling system.  The message had been cryptographically signed and encrypted both by Sascha and the original author, meaning two things.  First, the messages were authentic, since only Sascha and the gentleman from the software firm had copies of the keys that were used to sign and encrypt the emails.  Likewise, Jesse was the only one with a copy of his own key, meaning that only he could decrypt and read the message that Sascha had forwarded.  The second meaning of the signed and encrypted message was that it was important and sensitive.


PGP, or Pretty Good Privacy, was mainly used by just a few groups: curious computer enthusiasts, privacy wonks, diplomats dealing with sensitive information, and dissidents.  When someone wanted to send an electronic message to a contact, there was no guarantee that the message would arrive without being intercepted.  Much like tapping a phone line by splicing a few wires, the data that travelled across the internet could be seen by anyone along the path that the data took on its way to its destination.  This posed a serious problem for anyone wishing to have a private conversation.  In the Middle Ages, kings had solved this problem by inventing shift ciphers.  For example, they might rotate each letter one space in the alphabet so that \emph{mouse\emph{ became }npvtf}.  This was effective, but eventually everyone learned the cipher.  More sophisticated techniques were developed by mathematicians, especially those who were employed by well-funded militaries.

When computers arrived on the scene, though, the game changed.  A computer could check every popular shift cipher in a matter of microseconds, making it difficult to encrypt messages in a way that ensured their privacy despite the presence of many prying eyes.  PGP worked by asking each party in a secure conversation to generate two very long strings of numbers.  One was called a \emph{private key\emph{, and the other a }public key}.  Each person gives everyone else his public key and carefully protects his private key.  If Jesse wanted to tell Sascha that lunch would be in the cafeteria on Friday and wanted to be certain that his boss wouldn't read the message, then he would encrypt the message using Sascha's public key.  The magic of PGP lies in the next step.  Sascha's public and private keys were created as a pair.  When Sascha receives a message that was encrypted with his public key, then he uses his private key to decrypt and read it.  His private key is the only key in existence that can do so.  If Sascha wanted to confirm the information, then he could send a message back to Jesse by encrypting it with Jesse's public key.  Since Jesse is the only one with a copy of his private key, he is the only one able to read Sascha's return message.


Jesse typed in his passcode, unlocking his private key and allowing his email program to decrypt the message.

\begin{verbatim}

-----BEGIN PGP SIGNED MESSAGE-----
Hash: SHA1

Jesse,

It's amazing what you can achieve when
you put your hand on the center of mass of a
situation and give it a healthy bodyslam.
Give this a look.

-- Sascha

> Dear Mr. Greene,
>
> Thank you for writing.  I am very sorry
> about the difficulties with the ticket
> system.  It is inexcusable that such a
> glitch would make it into the production
> version of this software, and we will be
> taking immediate steps to fix it and
> work with you to ensure that the changes
> are deployed as soon as possible.  I
> will be investigating this matter
> personally and have launched an internal
> investigation to determine how the
> problem evaded the procedures that we
> use to identify security flaws in our
> products.
>
> I feel it necessary to inform you that
> we recently terminated an employee's
> contract because he confessed to
> installing backdoors in two of our
> flagship products that are targetted at
> large corporate clients.  CUNY is not
> in the target audience for those
> products, and neither of them is
> deployed at your campus.  I believe in
> transparency even when it is
> uncomfortable, and I am simply trying
> to keep you informed as we work our way
> through the details of this
> security breach.

> Given the confidential nature of this
> issue, I expect that you will keep
> the information in this email, and
> any subsequent emails, to yourself.  
> If any tampering or foul play is
> discovered on your end or ours, we
> will work with you to prosecute the
> individuals responsible to the fullest
> extent.
>
> Thank you for your business and your
> cooperation.
>
> Sincerely,
>
> Theodore R. Ginsburg, Ph. D
> CSO, Tall Oak Microsystems
>
>
-----END PGP SIGNED MESSAGE-----
\end{verbatim}

\emph{Sascha, you rabble rouser}, Jesse thought.  He knew that there was a good story behind this email, but the details could wait.  Sascha did his best work when he was forced to make things happen on his own.  The slow pace of university politics and decision-making caused Sascha a lot of heartburn, and this was the type of situation that he thrived on.  Sascha's boss gave him a lot of latitude in his work, especially in circumstances where the system was failing.  With the safety of staff and students on the line, it was almost a blank check.  If Sascha needed to go over the heads of I.T., the dean's office, or even the board members, then that's just how things happened.  It would wait, though.  Jesse had subsisted on coffee since breakfast, and his body was ready to collect on its loan with interest.

\chapter{}

The lunch options that were available at the CUNY cafeteria were nothing short of amazing.  When Jesse was an undergraduate in Colorado, his meal plan could net him a couple of pizza slices, a burger, or maybe a bit of stirfry.  If he wanted something that actually satisfied his nutritional needs, he would need to find time in the day to visit the local grocery or farmers' market and find a place to cook the food.  It was a subpar situation, and Jesse was glad to move into his first apartment where he had access to a kitchen that was larger than a linen closet.  The CUNY students, on the other hand, had it made.  They swiped their student cards at the door and walked into a plaza that offered no fewer than eight food stations, each offering a unique couisine.  If a student was in the mood for sushi, Thai food, Chicago-style pizza, vegetarian entrees, or a peanut butter sandwich, she only needed to wander to the correct counter and point a finger.  The hot food was prepared while she waited, and the ingredients were purchased fresh from area farmers as much as possible.  CUNY had its share of problems, but they always got the food right.

Jesse stood under the Thai Bear banner, waiting for a plate of pork satay.  He admired the skill of the brown-haired student who had taken his order and thrown it onto the large, circular cooking surface behind the counter.  A six-foot diameter vent mounted on the ceiling took care of the smoke and excess heat from the food as it cooked.  \emph{I need one of those in my kitchen\emph{, he thought.  }Maybe I can cook for her next time}.  Jesse had barely finished the thought when he saw someone approach on his left.  He saw a middle-aged man with a gray beard and a full head of white hair.  The man was wearing a flannel shirt and had his sleeves rolled up neatly at the elbow.

``Mr. Winter, I'm Dr. Ian Blair from the Department of Philosophy,'' the man said, offering his hand.

Jesse nodded, shaking Dr. Blair's hand.  \emph{This must be the guy who's suing me}, Jesse thought.  ``It's nice to meet you, Dr. Blair.  How are you doing today?''

``Fine, just fine,'' Blair said. ``Please call me Ian.  I was wondering if I could talk to you for a moment.  I have a table over by the window.''  Ian motioned to a place near the main entrance to the cafeteria.

``Umm, sure.  I'll be right there.  My food will be ready in a couple of minutes.''

``Of course.''

The man walked back to his seat, picking up a magazine and picking at his food while he read.  Before Jesse could come up with a suitable excuse to get out of what promised to be an awkward conversation, his tray of food was pushed across the counter toward him.  He thanked the cashier by name and headed toward the table where Ian Blair sat, engrossed in a magazine article.

``What's the occasion, professor?'' Jesse said, sitting down across from Ian.

``Ah, yes.  Mr. Winter, I \dots understand that one of my colleagues might be making some trouble for you.  We philosophers argue about a lot of inconsequential things, and sometimes our zeal for debate and ideology spills over into the real world.  I assume that you know what I'm referring to.''

``I think so.  I learned yesterday that there was another suit being brought against me, and this time it was from a professor in your department.  The charges are serious.  Not your run-of-the-mill philosophy discussion.''

``Yes, that's why I asked you to chat with me.  I think that the rabbit hole goes deeper than anyone is willing to admit.  My colleague is known to mingle with some folks from Europe and southeast Asia.  I find them shady, and I know I'm not the only one who does.  Nothing has ever been proven, but there is a lot of anecdotal evidence that could be used to piece together a case if it was necessary.  I don't know the details of your case for obvious reasons.  But you should know that this man doesn't do anything half-ass.  If he's after you, then you'll need all the allies you can gather.''

``I appreciate the insight, Ian.  But why are you helping me like this?''

Ian looked up at Jesse and then out the window.  ``He came after me once,'' he said finally.  ``I didn't defend myself as well as I might have, and it cost me a year's salary and a series of publications in good venues.  My reputation took a big hit, and it consumed a lot of my time.  The department didn't give me any protection.  Everyone cowered like scared cats.  It's time for the culture to change, and I want to be part of the solution.''

Jesse raised his glass of iced tea in a gesture of respect.  ``Thanks.  Do you have a business card?  I'll be in touch.''

\chapter{}

The next week was remarkably routine.  Jesse kept moving his workstation in hopes of shaking off his secret admirers.  He never noticed anything out of place and never saw anyone watching or following him.  There were no bugs in his apartment, to the best of his knowledge.  His superiors had brought in a professional forensics and security crew to inspect every inch of his old office, looking for any unrecognized fingerprints, equipment, or disturbed ceiling tiles.  Nothing had turned up.

A security review of the ticket system had been conducted as well.  The records showed that the phantom ticket that defeated the door security system was created by a terminal in the basement of the Entomology Building.  The bughouse, as the Entomology Building was lovingly called by the students that spent their time there, was one of the oldest buildings on campus.  It was once the president's residence, and the university hadn't gotten around to installing surveillance equipment, making any effort to track the user of a computer in its basement a futile exercise.  The only camera in the vicinity was too far away to show any facial detail of any individual who was coming or going.

When the police chief had learned that Sascha had destroyed the bug by sending down a sink drain, he was furious.  Sascha's instinct had been to isolate and then eliminate the threat, and he had done so.  Unfortunately, this was no time for battlefield tactics.  It the police had access to the bug, then they had a fighting chance of tracking down a receiver and the person responsible for installing the equipment.  As it stood, they could do nothing but encourage Jesse to change his habits and keep a watchful eye for anything suspicious.

\chapter{}

Jesse had just finished a review of two projects that were conducted by teams of students in Taiwan and Slovenia.  The team in Slovenia had done an excellent job researching the Soviet influence on modern school curricula and what the young generation of politicians were doing to establish a new blend of Slovenian culture and national pride.  William Tweedie planned to give them an award for their work.  ``Every one of these glory stories,'' he had told Jesse in private, ``is another gaping hole in the arguments of those asshats who tried to bug you, and the guys in Poland.  We're winning, Jesse.  We're winning.''

For Jesse's entire life, he had been told that there were winning sides and losing sides.  Good and evil.  The right way and the wrong way.  He had often wondered what it felt like to be on the other side.  Weren't the evil guys just fellows who happened to come to a different conclusion; guys who thought that they, in turn, were on the right side and that everyone else was wrong?  When he went to college, Jesse learned a great deal about argument, debate, and premises.  A premise was an assumption, a leg that you used to prop up the argument that followed.  If you bet on the wrong premises, a skilled opponent would crush you.  Some folks would perform the most graceful and complicated verbal gymnastics in order to save face.  They were wrong, and they sensed it, but they would perry and dodge their way out of the fight.  This was a favorite strategy of politicians.  They would never admit obvious, well-known truths that would make them vulnerable, and thus they would put forth a flowery stream of words that diverted the conversation from genocide in Africa to the charming notion that their grandmother's coffee shop sold the best cookies on Third Street, and that's why America is the place where every kid should grow up.

The problem with all of this, though, was that smart people could construct a reasonable-sounding argument for any old thing.  Unless it was a cut-and-dried subject, or perhaps a problem on which rigorous science could be brought to bear, it was probably too complicated to fully understand.  Is it better to follow the Keynesian school of economics, or is the Austrian school the One True Way Forward?  Can a mixed economy sustainably deliver the free market that we worship \emph{and} the services that we've come to enjoy?  Politicians wrangle about these ideas endlessly, and each will break down your door to tell you that he's found the One True Way Forward.

The only answer for Jesse was to drop a moral anchor made up of the few things that you really cared about and let the chips fall where they may.  Those few things are what makes a person, and it's a miserable man who tries to live in a way that's out of sync with them.

Some people, left to their own devices, have no problem causing injury to other people who they will never meet.  The ability to separate ourselves from the suffering in the world is a vital survival tool, but it also gives rise to indifference.  When the world of each person was small, interactions with unfamiliar people usually led to violence or avoidance.  Even when the world grew larger and more industrious, people dealt locally.  They saw the impact of their decisions, and they felt the repercussions of their mistakes.  When the world grew large enough that individuals could move freely from one place to another, and to begin a new life in a new place without the knowledge of any previous acquaintance, the conditions were right for organizations of individuals to pride themselves on growing wealthy and powerful on the misfortune of millions.  Without the ability to look their victims in the eye, their evolutionary sense of altruism that had developed over tens of millions of years in the forests and open plains was snuffed out, lacking the stimulus that it needed in order to guide them.

Someone had to watch the watchers.  To watch the centers of power.  Not only was it likely that those in power would be tempted to allow its scope and potency to creep and grow, but history had delivered the unalloyed verdict that it was inevitable.  One societal power structure after another had rendered its version of justice and its vision for progress, and one after another had tumbled.  We have learned many lessons, and modern power tends to live longer and hide its bad deeds more effectively.  As power learned how to act in public, though, its watchers needed to learn to read between the lines.  They needed to learn its whims, its ambitions, and the patterns in its expression.  By staying one step ahead, the watchers could give the power an immune system.  A means of keeping it honest.  A means of giving its hand a reason to follow its mouth.

Jesse felt strongly that investigative journalism was the key antidote, or at least one of them, for the world's political struggles.  Without a press that was willing to dig in, ask questions that make politicians incredibly uncomfortable, and be willing to rock the boat, there was no check on power.  

The same can be said about the day-to-day decisions of everyday citizens.  The ordinary person on the street had neither the means nor the motivation to learn whether the merchant who sold him his lunch paid a fair price for the raw ingredients.  Or whether the merchant bought from suppliers who he knew to be ethical.  Were the men who harvested the beans for his coffee that morning slaves, or were they honest men working for other honest men who made a living for themselves?  The separation between people that led to abuses was not the exclusive province of the rich and powerful.  It was not only the merchants and the owners of land and resources.  Each decision during the day of the average man has consequences.  This is the result of the division of labor that enabled society to make vast strides in production, health, and quality of life.

When a man or woman picks up a piece of fruit in a market and savors its flavor, that transaction reinforces the chain of events and exchanges that carried the fruit to that place.  There are billions upon billions of such exchanges that happen each day, and the conscientious man demands accountability in the chains that he encounters.  His desire for ethical expression, the chance to make the world a more perfect place, gives him pause.  It is investigative journalism that meets this demand.

\chapter{}

``When will I hear more about the charges?'' Jesse asked.  Jesse and Sascha were sitting alone in the Office of Internal Affairs.  It was the end of a long day for both of them.  Jesse had formally presented the students who traveled Slovenia for two months their award from the president.  During the photo session that followed, one of them had mentioned that she had found her calling in life.  She would be traveling to Europe after graduation to work for Amnesty International on human rights cases.

``Probably after the news about the Slovenian crew dies down,'' Sascha replied, stretching his legs and his head on his arm.  ``It would be bad P.R. for a faculty member to throw down the gauntlet on the same day that the program had given a prestigious award for a successful trip.  Even the critics can't say anything bad about this one.''

``Except that the students should have left the work to the Slovenians.''

``Ah, the old `They're taking our jobs' argument, applied in reverse!  Maybe this will light a fire under some universities over there to do some honest investigation.  With the exception of Fisk, hardly anybody knows what journalism is anymore.''

``I saw Fisk speak at M.I.T. once.  I remember thinking that he would make an excellent peace broker, because he knows the state of affairs so well that nobody could feed him any bullshit.  But then I realized that brutal honesty and incisive critique are out of bounds when you're dealing with diplomats.  They're trained in soft skills and polite chat, not in history.''

Sascha laughed.  ``Don't worry, Jesse.  We've got the next group of young journalists on the blocks right now, and our professors cite Fisk all the time.  In a few years, our biggest problem will be how to keep the students on our waiting list from tearing our doors off their hinges.''

``Here's hoping,'' Jesse said.  ``I'm going to head out, man.  Thanks for the chat.''

``Any time.  Hey, let me know--''

Sascha's desk phone rang, interrupting his train of thought.  He picked up the receiver and said, ``Greene''.  He frowned, listening to the voice in the earpiece.  ``Yeah, he's here.''  Looking up at Jesse, he shrugged, looking puzzled.  ``Sure thing.  I'll send him down.  Bye.''

``I guess you'll be finding out about those charges tonight,'' Sascha said.

``They want to talk to me \emph{now}?'' Jesse said, sounding incredulous.

``You got it.  Room 290, ten minutes.  Bring your badge.''

\emph{Bring your badge}.  Jesse knew what that meant, and his heart sank.  When a police officer was under investigation, his gun and shield were taken until his name was cleared.  Around here, your badge was revoked and you were placed on administrative leave.  This was Jesse's equivalent of being neutered.

\chapter{}
										
Room 290 was cold and clean, a room that was designed either to impress or intimidate.  Which effect took precedence depended on the lighting.  On this occasion, every light in the room was burning.  There were three conference tables arranged in a U-shape, with a single seat placed at the top of the U.  There were small drink knapkins laid on the tables in front of each seat, and karafes of fresh water had been deposited at even intervals.  Jesse counted ten fifteen seats.  \emph{Full tribunal}, he thought, still puzzling over the choice of time and location.

``Mr. Winter, may I take your coat?'' a voice said from Jesse's left.  He turned, nodding to the well-dressed woman whose face he recognized, though he couldn't place it.

``Yes, thank you.''  Jesse slid off his overcoat and handed it to the woman.

``You're welcome,'' she said, pulling a hanger from the coat rack on the wall.  ``I'm Barbara Nash.  I'll be the note taker for this meeting.''

Jesse smiled, trying to hide his anxiety.  ``I remember you from the Warsaw trial.''

``With a bit of luck, this meeting will be less stressful than the trial was.''

``Do you know why this was arranged so hastily?''

``This is just a preliminary inquiry.  It was arranged more than a week ago by the Office of the Ombudsman.  Normally the defendant isn't present, but the dean decided that you should be here, in light of the unusual circumstances surrounding the case.''

``Unusual circumstances?''

``Yes, the Office of the Ombudsman has requested a closed-door hearing because the charges are being brought against a staff member whose job involves handling sensitive information.  This will be the first such trial in university history, Mr. Winter.''

Jesse was getting annoyed.  His work dealt with the same amount of sensitive material as it did during the Warsaw trial, and every Paul and Joe, as well as the press, had been allowed to sit in for that one.  \emph{As they should have been}, Jesse thought.  It worried him that they might try to conduct this trial behind a veil of secrecy.  Justice was more honest with a dispassionate audience.

``I worked on sensitive things before my first trial, Ms. Nash.  What's different now?''

``It will make sense when you're presented with the formal charges, Mr. Winter.  Please have a seat.''  Ms. Nash gestured toward the lonely seat at the opening of the tables.  From her tone, Jesse inferred that the conversation was over.

In the ten minutes that followed before others began to flow in, Jesse made up his mind.  He was tired of being accused and tired of being singled out.  He silently rehearsed two of the best opening statements that he had delivered in his college debate club, warming up his argumentative instincts.  If being aggressive lost him his job, then he would pay the price.  He would be in touch with his attorney first thing in the morning, but for tonight he was on his own.  The arrangement of the room played into his hand, and Jesse was sure that the meeting organizers wouldn't expect that.  Far from being a shy animal under the bright lights, Jesse was free to stand up and move around.  In his college debates, he frequently made use of the physical space as a persuasive tactic.  It made some opponents nervous, and in Oxford-style debates, where the audience chose the winners, it made an emotional connection with the gallery that reliably brought Jesse votes.

The Ombudsman himself was the first to arrive, taking a brief and disinterested look at Jesse before sitting at the table to his left.  Two of the Ombudsman's staff members arrived a moment later, along with a woman whom Jesse recognized as an administrative assistant from upstairs.  The dean of Internal Affairs, Sascha's boss, arrived next and took the seat nearest to Jesse at the table on his right.  He offered a polite smile before opening a leather binder and looking through some papers.

A slow but steady stream of people entered over the next few minutes.  The next to enter the room was a man in a sharp black suit.  He was carrying a stack of folders under one arm and tapping on his smart phone with the opposite hand.  He had the distinct air of a lawyer, and Jesse's suspicion was reinforced when a gray-haired man in a polo shirt followed him.  \emph{A stuffy old philosophy professor}, Jesse thought.  \emph{My accuser}.  There were only two seats left at the table, and the man fit the bill.  

When everyone was seated, the Ombudsman rose to his feet.  ``Ladies and gentlemen, thank you for being here.  This is a preliminary inquiry in the case being brought against Mr. Jesse Winter by Dr. Sam Malta.  Both Mr. Winter and Mr. Malta are present tonight, and we would like to get all of the details out in the open.  This needs to be a transparent discussion.  If we can't have a rational discussion, then we'll keep doing this until we do.  Understood?''

There were murmors of agreement from each table.

``Mr. Winter,'' the Ombudsman continued, ``we can begin with a statement of the charges, if you would like.''

``Yes, I think we should do that,'' Jesse said, his tone suggesting that this was obvious.

The Ombudsman cleared his throat.  ``All right, then.  Mr. Winter, this is Mr. John Shelby, and he will be representing Professor Malta.''  He gestured first to the man in the sharp suit, then to the gray-haired man that had followed him into the room.  Then he nodded to Mr. Shelby and sat down.

Shelby stood up, absently adjusting his tie.  ``Thank you, sir.''  He picked up the piece of paper at the top of his stack and propped his reading glasses on the end of his nose.  ``Mr. Winter, these are the charges.  You are accused of destroying university property on two occasions.  The first occurred on July 28th of this year.  The university-owned materials in question were the records of a student who is enrolled in the investigative journalism program and conducted a study in Venezuela.  The second incident occurred on August 4th, also of this year, and involved a set of computer backups that contained the only remaining copy of the student's records after the hard copies were destroyed.  You are also accused of espionage with the intent to disclose university documents to foreign entities stemming from your involvement in an investigation of a student doing research in St. Petersburg, Russia.''

If Jesse's face reflected the state of his mind, then he had a look of disbelief on his face, and his right eyebrow was raised.  ``This is complete rubbish, Mr. Shelby'' he said flatly.  ``What's the evidence?''

The lawyer smirked, considering Jesse for a moment before he spoke.  ``It's more than enough to convict, Mr. Winter.  And with a mouth like that, you can be glad that we're not at trial.''

``Mr. Ombudsman,'' Jesse said, ``I was told that we would have a transparent discussion.  If the evidence isn't discussed, then we will never come to an understanding.''

The Ombudsman cleared his throat, looking at Mr. Shelby over the top of his reading glasses.  ``He's right, counselor.  You wanted us all to be here for this meeting.  You need to play ball.''

Jesse could tell from the body language among the attendees that nobody was thrilled to see Shelby.  Even his client had given him a few sidelong glares that suggested a disagreement over how to proceed.  Shelby knew that he had to build a case according to the rules of the courtroom, but Dr. Malta almost certainly had plans to make a larger, more philosophical point.

Shelby shuffled his papers for a moment, apparently thinking through the evidence that he had collected.  ``Let's begin with the destruction charges, then.  The student records were found in Mr. Winter's waste bin on the day in question.  The computer backups were shown by CUNY computer technicians to have been deleted by an account belonging to Mr. Winter.''

``Do you have anything other than circumstantial evidence, Mr. Shelby?'' Jesse said.  ``So far, this is a weak case.  It makes you look like you have a vendetta.  My office disposes of records on a continuous basis, and we can't be bothered by someone who wants to dig through our laundry.  I can tell you that if the records were in the waste basket, then it was a mistake.  They should have been securely destroyed in accordance with our document retention and disposal policy.  How will you demonstrate that the records weren't intentionally placed in the trash by someone else?''

For all Jesse knew, Shelby had never heard of him before this case.  There may be no vendetta.  University lawyers tended to know each other, though.  Even if Shelby had no personal beef with Jesse, one of his colleagues may have had his ear.  Jesse's job had given him a nice view of CUNY politics and bureaucracy at work, and he knew that he needed to cover himself.

``Throwing the conspiracy and vendetta cards awfully early there, Mr. Winter,'' Shelby said, his eyebrows dancing as he spoke.  ``As I said, we have enough evidence to convict.  We'll let the judge decide who dropped the records in the trash.''

The Ombudsman cleared his throat loudly, interrupting the conversation.  ``Mr. Shelby, let's move on to the second set of charges.  We all have other places to be tonight.''

``Without a doubt, sir,'' Shelby said, giving something that resembled a short bow.

\emph{Litigious prick}, Jesse thought.  Judging from the resentful smirk on Barbara Nash's face, Jesse was not the only who was irritated by Shelby's manner.

``Mr. Winter,'' Shelby began, ``the evidence against you regarding the charge of espionage with the intent to disclose university documents to foreign entities is as follows.  We have a sworn statement from a Hungarian national who claims that he acquired private student records from you.  The gentleman was arrested for his role in an armed robbery at the CUNY post office last year.  He appears to be an activist who has been in the United States on a student visa for two years.  He has supported himself through a series of jobs for a Hungarian political organization that has ties to Hamas and a Mexican drug cartel.  Most of the jobs have been illegal.''

Shelby paused for a moment to take a drink from his glass of water.  ``The district attorney's office brokered a deal with this guy.  If he confessed to the charges against him and gave verifiable information about the jobs that he performed, then his sentence would be reduced.  You were mentioned by name in his statement, Mr. Winter.''  He looked directly at Jesse.  ``He claims that he paid you five thousand dollars to make copies of a student's academic and criminal record.''

There were soft gasps.  Half of the people in the room looked at the Ombudsman.  The other half looked at Jesse.

``Is the university authorized to have copies of criminal records at all?'' Barbara Nash asked, sounding shocked.

The Ombudsman looked down at his notes, contemplating his response.

``In some circumstances,'' Jesse said.  Every eye in the room turned toward him as he spoke.  The Ombudsman looked up from his notes, looking relieved.  ``If a student has been convicted of a felony, then we're obligated under federal law to maintain records for that student.  If the felony charge is for something that might put students or CUNY employees at risk, then we're obligated to hold a special hearing before the student is admitted.  The paperwork is held in the secured vault along with other sensitive records.''

``Mr. Shelby, is there anything further?'' Jesse asked, rising from his chair as asked the question.

Shelby looked surprised.  ``No, Mr. Winter.  You've been read the charges being brought against you.''

``If I may, then, Mr. Ombudsman, I would like to make a statement of my own.''

The Ombudsman nodded, gesturing for him to continue.  Shelby took his seat.

``First, I would like to thank everyone for taking the time to come to this meeting.  It isn't easy to put together legal cases.  The evidence is often unreliable.  Witnesses can lie.  Even a perfectly-constructed case can be derailed by politics and other factors.  At the very least, we hope to give a fair shake to everyone involved---especially the accused.''

Jesse took a couple of steps forward, putting himself within the rectangle formed by the tables.

``I'll admit that I'm not an unbiased party here.  But I think that this hearing, and the trial that may follow, is a waste of your time.  It's a waste of the university's time and resources.  I am aware of the circumstances that made the each of the charges possible.  The problem is that Mr. Shelby and Mr. Malda are looking for intent where there is none.  My office has accidentally thrown away records before, and we have a procedure to deal with it.  No documents are allowed to remain in the trash, and each trash bin is checked before it's taken away.  All documents that need to be discarded are shredded by a cross-cut shredder and then incinerated.  If something managed to make it out of the office intact, then it was an honest, but still unforgivable, mistake.  We owe it to our students to protect their personal records.  However, it sounds to me as though someone plucked a discarded document from our office and is now trying to argue that we did something horribly wrong.''

A few heads were nodding, but Jesse knew that he wasn't out of the woods.  The second charge worried him.  He recalled hearing from a man with a European accent.  The man had claimed to be a police officer for the New York Police Department who was working on a case involving a CUNY student.  Jesse had followed the proper procedures for sharing information with the police.  He had returned the call by looking up the phone number for the NYPD, ignoring the number that the man had provided in his voicemail.  This simple check thwarted attempts by some clever individuals to provide a fake phone number.  Everything had checked out, and Jesse forwarded the student's records by secure email.  If the man had been an imposter, then he was \emph{very} good---and Jesse had fallen for it.

Jesse paused for a moment, moving slowly through the space between the tables.  He noticed, with some amusement, that Shelby was looking annoyed.

``Now, regarding the charge of espionage.  I remember the phone call from this man.  He said that he was a police officer from NYPD who was looking for information about a student.  We are required to comply with police requests, and again, we have procedures for this.  I followed the protocol.  I called the officer back at the official NYPD number, used my judgment, and sent the information that he asked for.  You will find no breakdown in the process that was followed.  Nor will you find any misconduct.  If there was intent to do anything illegal, it was on the part of the gentleman from Hungary.  If our processes were defeated by someone who was cleverer than we are, then that's a problem that we need to deal with.  Accusing employees of espionage is not the right answer.''

Jesse nodded to the Ombudsman and returned to his seat.  There were fleeting glances between the tables at this point.  Jesse knew that he had hit the mark.  Even thought he couldn't offer any direct evidence for his arguments, he knew that he had sewn just enough doubt to cut the legs from Shelby's case.

The Ombudsman cleared his throat.  ``Thank you, Mr. Winter.  At this point, I have to recommend that the charges against Mr. Winter be dropped pending a more thorough investigation.  ``Mr. Shelby, do you have anything further?''

``No, sir,'' Shelby replied.  He gave Jesse a sour look.  This was not how the meeting was supposed to go.

``Very well,'' the Ombudsman continued.  ``Mr. Winter, thank you for joining us.  I think you will be hearing from us again as we try to get a handle on exactly what is behind these accusations.  Everyone else, thank you very much for your time.  Meeting adjurned.''

Jesse stood up, taking a deep breath and turning toward the door.  As he began to walk, he felt a soft hand on his arm.

``Jesse, you don't know the bullet that you just dodged,'' Barbara Nash whispered as she fell into step beside him.  ``Shelby has a reputation for this type of case.  The board has been trying to get rid of him for years, but he has some friends in high places.''

``I don't think I've heard the last word from him,'' Jesse said, ``but maybe I saved everyone some time in the courtroom.''

``I think you saved us more than that.  Nice work.''  She smiled at him and turned away, heading toward the restroom.

Noticing that it was after 9 p.m. and Sascha would be gone, Jesse went home.  He had learned enough for one night.

\chapter{}

Jesse woke the next morning with a headache.  Looking out the window confirmed his suspicion.  When low pressure storm fronts moved through, Jesse's head was a better barometer than the evening weather report.  He took a pair of acetominophen pills with breakfast, hoping to ward off the pain before it spoiled his day.

The headlines for the day included a shooting in Memphis, a large anonymous donation to the National Science Foundation, and a nice article about a newscaster who had recently passed away.  Turning to the next page, Jesse noticed a blurb about a foreign student from New York who had been arrested for his involvement in an organized crime ring.  The full article was on page 6B, and he quickly located it.

From the first line, Jesse knew that this was same case that he had been lectured about the night before.

\indent	HUNGARIAN NYU STUDENT ACCUSED OF SPYING, DRUG TRAFFICKING

\indent NEW YORK - Hungarian student in his third year at New York University was arraigned last week on charges of extortion, espionage and obstruction of justice.  The student, whose name has not been released by the NYPD, is accused of working for a Mexican drug cartel and Hamas, a Palestinian political organization.  He was arrested in July during a joint police and FBI raid of an abandoned warehouse in the Bronx.  It was determined that the student had lived in the warehouse for his three years at NYU and was in the United States legally on a student visa.  At a press conference on Tuesday, NYPD chief of police Sam Blanczyk said that Immigrations and Customs Enforcement was working with Hungarian officials to determine how the student would be tried.
	
\indent ``We have been tracking this student for several months, and it's believed that he has connections to activist political organizations in France, Poland and the Czech Republic,'' Blanczyk said.  The student's activities have also come under scrutiny from NYU, and City University of New York has expressed concern over an incident in which the man impersonated a police officer in order to steal sensitive student records.  CUNY president William Tweedie could not be reached for comment by press time.  John Shelby, an attorney for CUNY, said that his office was investigating the case and would be pressing charges if mistakes by CUNY staff had led to the disclosure.  When asked whether CUNY would be filing charges against the accused student, Shelby said that he would let the FBI and NYPD handle that litigation.
	
\indent\indent\dots
	
Jesse stared at the newspaper.  He had been reluctant to associate the current batch of events with the events that led to the Warsaw trial.  The charges against him were different, and the details that he learned at the meeting the night before had given him no reason to think that there was a connection.  This article had given him a shadow of a doubt, however.  The student had connections to unscrupulous, and even dangerous, political organizations who might in turn be connected to organized crime.  The background material that Jesse was given for his flight to Poland had included a section on the many connections of the political party whose office the students had bugged.  The party was very old, and it had established relationships in agriculture, textiles, and more recently high technology.  The material went on to explain that despite these well-regarded relationships, the party had made a fortune by bankrolling crime bosses and helping them stay out of trouble.  This symbiosis made the party popular because it appeared to have organized crime under its boot heel, but the facade was exactly that -- a front -- and one that could only hold sway until the seas got rough for either the crime bosses or their friends in the party.  It wasn't clear what the nature of their relationship was in 2010, but some inside sources had said that there were signs of fractures in the old alliance.

It now seemed possible that there was a connection.  Jesse picked up his phone and sent an SMS message to Sascha.  He wondered if Sascha, another avid \emph{New York Times} reader, had seen the article.  A moment later, the response glowed on the tiny screen: \emph{Saw it.  Meet me at the cafeteria at 8:30 and we'll figure out what to do}.

Apparently Sascha had not only read it, but he was having the same thought that Jesse was.

Jesse had begun to wonder just how deep this rabbit hole went.

\chapter{}

``It's all connected, isn't it?'' Sascha asked, digging into his eggs benedict.

``You know, I'm trying not to fall into the trap of drawing a link where there isn't one, but it feels like there's something.  What does it really mean if there \emph{is} a relationship between this student and the guys in Poland, though?  People who participate in these crime rings are part of a relatively small community, so is it all that strange to think that one bad guy has done business with another?  Unless the guy has a personal connection to the goons in Poland, there's no risk to any of us.''

``That's where you're wrong, Jesse.  Even if they don't have a personal connection, this student has been doing dirty work for crime rings for years.  He's in the business of taking orders, collecting his pay, and keeping his head down.  He won't be the one to pull a gun on you, but Shelby already made it clear that he knows where you work and how to get in touch with you.  And he knows that you can be buffaloed into giving up sensitive records\dots''.

``Hey!'' Jesse protested sharply.  ``I followed all the protocols.  The guy knew how to act.  If I trusted my instincts the way I prefer to, then \emph{nobody} would get any information from me.''

Sascha laughed, holding up his hands.  ``I was just joking, dude.  You did the right thing, and any judge or jury worth their salt will see that.  Shelby has his agenda.  He made that abundantly clear in the paper.''

``Yeah, that made me laugh.  `Hey, let the FBI do the real stuff.  It will serve as a good cover for my underhanded politicking.'  Anyway, what's the next step?  Should we probe this a bit and try to find out how strong the connection is, if there is one at all?''

``I think it's safe to assume that there is some linkage.  Look at it this way.  If we ignore it, then the student will go to prison and the trail will go cold.  The guys in Poland will hire someone else to do their bidding.  If we dig a little, then we might be able to help the feds and CUNY to shed some light on groups that have been dogging them for decades.''

Sascha held up his hand, fingers closed, waiting for a fist bump.  ``Are you in?''

Jesse didn't hesitate.  Investigating the sneaky, ugly practices of these groups was exactly the reason that he had joined CUNY to work in the investigative journalism program.  ``I'm in,'' he said, closing his fist and thrusting it toward Sascha's.

``Let's go talk to the guy who represented you in the Warsaw trial,'' Sascha said, rubbing his hands together.  ``What was his name?''

``Browning.  Chuck Browning.''

``Right.  He's familiar with the case, and we stand a better chance if we work with him instead of Shelby.  We'll need to find out who prepared all of that intelligence that you were given on the flight to Warsaw.  And come to think of it, we should probably talk to the NYPD.  Maybe they'll ask the Hungarian kid a few questions for us.''

Jesse enjoyed watching Sascha work once he got on a roll.  ``Anything else?'' he said, grinning.

``Yeah.  This coffee sucks.  Let's go that place on 3rd that roasts their own beans.  It's on the way to NYPD headquarters.''

\chapter{}

Jesse kept pace with Sascha on the way to the NYPD more easily than he had a week before.  He wasn't waking from deep sleep, for one thing, and Sascha wasn't coming down from a huge dose of adrenaline.  As they walked, Jesse's mind began to grapple with the fragments of evidence and information that he had learned in the past few days.  He had been punked, although more seriously than in the television show.  One of the lead prosecutors for CUNY had decided to pick on him.  The prosecutor had made public statement, to no less a source than the \emph{New York Times}, that he preferred to stay out of the real business of prosecuting confessed criminals.  Instead, he was going to find people within CUNY who disagreed with his vision for the world, and he would make it his personal mission to litigate those people into jail time, resignation, or both.  If that didn't work, then it seemed that he was willing to plant evidence in order to force the issue.  Either way, the guy had a reason to come after Jesse  and anyone he worked with.

Jesse suddenly emerged from his mental jigsaw puzzle, realizing that Sascha was talking to him.

``---but really, they should be willing to let us get some tough questions in on this guy.  We have a right to know, especially if he broke into your office.  We're a directly affected party at that point, and we should be part of any suit that's brought against him.''

``You mean---oh, oh shit,'' Jesse said, his eyes widening.  He slowed his pace, tasting the new thought that Sascha had given him.  One that he should have realized on his own.  ``You mean that this guy might have been the one who bugged me?''

Sascha gave Jesse an odd look.  ``Yeah, that's exactly what I mean.  What?  You didn't make that connection until just now?''

``I guess not.  I was so absorbed in what Shelby said during the meeting last night, and then there was the article in the \emph{Times}.  Wow, I---.  But---that should help us to find him, right?  There might be CCTV footage from the bus stop?''

``Maybe,'' Sascha said.  They were starting to walk at their original pace again.  ``This just reinforces the fact that we need to meet this guy face-to-face.  You might recognize him, and then we'll have something to build on.''

\emph{He'll certainly recognize me}, Jesse thought.


With better coffee in hand, they walked another three-quarters of a mile down the road to the main entrance to NYPD headquarters.  They walked through the revolving door, Sascha first, and waited a moment for their eyes to adjust to the lowered lighting.  There was a reception station on their right, and a uniformed woman waved them over.

``Welcome to the NYPD,'' the woman said.  ``May I direct you or call someone for you?''

``That would be great,'' Jesse said, surprising Sascha with his prompt response.  ``We're from the City University of New York's Office of Internal Affairs, and we would like to talk to someone about an ongoing investigation that might have involved one of our staff members.''

``Right away, sir.  If you don't mind my asking, is it in relation to the case involving the NYU student who was arrested last week?''

``Yes,'' Jesse said.  He was surprised by this question, but it saved a lot of awkward explanation.

``Okay,'' the receptionist said.  ``I'll need to see government-issued identification for each of you----sorry, it's protocol---and then I will get someone out here to talk to you.''

Jesse and Sascha pulled their driver licenses from their wallets and handed them to the receptionist.  She held them under a special light on her desk, apparently checking for a watermark.  Satisfied, she handed them back.  ``Thank you very much,'' she said.  ``I'll have someone come out and get you shortly.  In the meantime, please have a seat over by the watercooler.''

``Okay, thanks,'' Jesse said.  He followed Sascha over to the set of cushioned blue chairs that flanked the watercooler.  There was a middle-aged man reading a novel in one chair, but the rest were empty.  They settled in, taking in the rest of the scene.

``This place isn't at all what I expected,'' Sascha said.  ``It's nothing like a regular police post.  With the desks, the computers, the polished floor, it's more like an Apple store.  It must be nice to have the resources of a big city PD.  This is better than anything that I saw in the Marine Corps.''

``Where were you stationed, Sascha?'' Jesse asked.  He had been meaning to ask about Sascha's time in the service for a long time, but the moments when it was appropriate to ask something like that were rare.

Sascha took a deep breath and tilted his head toward the ceiling before he answered.  ``Fort Bliss for a while.  Texas.  I spent a few years in Japan, and then five in Germany.  Got to see the east and the west.  The east was better.  For me, anyway.  There was a lot more respect in that part of the world.''

``For elders?''

``For everyone, really.  There are folks in every country and culture who are disaffected and don't give a hoot about anybody, but the people I met in the east were much more formal.  Social situations didn't have the loose, hipster feel that they do here.  It was nice.  I never got used to the rice, though.  I guess I'll always be a western guy when it comes to food.''

Jesse laughed.  ``I can't see you eating rice,'' he said.  ``Or steamed veggies and trout, for that matter, but you eat it daily during the summer.  We're all complicated critters.''

``Speaking of complicated critters,'' Sascha said under his breath, looking across the room.  A man with salt-and-pepper brown hair and thin-rimmed glasses had just come through the door at the back of the room.

``Greene and Winter?'' the man said, looking toward the seating area.  The man reading the novel looked up with a hopeful look on his face.  As Sascha and Jesse stood up and signaled that they were the party in question, the man sighed and fell back into his book.

The man who had read their names was wearing a black pair of jeans and a gray dress shirt with the sleeves rolled up neatly to his elbows.  He was slightly shorter than Sascha, slightly taller than Jesse, and well built.  ``Gentlemen,'' he said, ``I'm detective Allan Forth.  How are you doing today?''

``Fine,'' said Jesse and Sascha in unison.  They each shook Forth's hand.  ``We're hoping to talk to you about the case involving the NYU student,'' Sascha added.

``Sure, right this way,'' Forth said, gesturing toward the door from which he had emerged a moment earlier.

They walked down a narrow corridor with offices and conference rooms on either side.  Every few steps, Jesse caught a fragment of a different conversation.  One was ordering food from a restaurant that served chow mein, another was asking whether someone would be available for a second interview on Friday, and a third was having an intense discussion that was riddled with legal terms.  The sound of phones ringing and fingers typing on noisy computer keyboards seemed to pound Jesse's ears from every direction.  It reminded him of a casino without the flashing lights, and that wasn't a memory that Jesse wanted to recall.

They made a right turn, taking them through a quieter hallway that held more conference rooms.  The detective stopped at a locked room that was labeled Conference 8N.  He pulled a large keyring from his pocket, unlocked the door, and invited them inside.  The room was smaller than Jesse expected.  It was no larger than an average dorm room.  There were four simple chairs, two on either side of a jet-black table that was bolted to the floor.  There were two large lamps overhead that were bright enough to remove any chance of a shadow.

``Please, have a seat,'' Forth said, gesturing toward the chairs on his right.  He locked the door behind him and checked something on his smartphone before sitting down.  ``Sorry for the spartan conditions in here.  This isn't our most comfortable conference room, but it's by far the most private.''

Sascha frowned.  ``Is someone trying to spy on us?'' he said with a polite chuckle and a hint of irony.

``Well, this case has done a real number on us around here,'' Forth said, his Brooklyn accent becoming more pronounced.  ``We should probably start at the beginning, just to be sure that we don't get our signals crossed somehow.  What's your interest in the case?''

Jesse spoke first.  ``Well, we're staff members at CUNY.  Sascha here works in the Office of Internal Affairs.  I'm a specialist from the Office of the Ombudsman.  Several months ago, we got a request from someone claiming to be an officer from your department.  He was interested in the personal and criminal records of a particular CUNY student and had filed an official request for them.  The responsibility for fulfilling that request falls to me, so I began the procedure for verifying his credentials and transferring the information.  Everything checked out.  I called the NYPD number, asked for him by name, and completed the transaction.  I didn't notice anything unusual.  My instincts tend to work well, and everything seemed kosher.

``Fast forward to Monday of this week.  I was informed by one of the attorneys for CUNY that an NYU student was arrested last week, and he had confessed to impersonating a police officer and explained that he had stolen student records using that false identity.  I was the person at the other end of the phone when he called.

``Now, that by itself might not give me any reason to associate myself with this investigation.  It's the next part that gets interesting.  Within the last two weeks, the two of us---Sascha and I---discovered an electronic recording device attached to my suit jacket.  Thinking back over that day, I think that it was probably attached when I was on a city bus on my way to work.  I didn't see or feel it happen, though, and we had no idea who might have seen the need to spy on me.  When we heard about this student, though, the pieces started to fall together.  He has some connections to groups in Europe that have grievances against CUNY and its investigative journalism program.  He has a reason to spy on someone like me, since it's my office that deals with the thorny legal and diplomatic issues that some of our student groups face when they work overseas.  We believe that he may have been the one who bugged me.  We have no hard evidence for any of this, but since the guy is in custody, we were hoping that your department and ours might be able to help each other out.''

Jesse looked briefly at Sascha, who quietly nodded his approval.

Detective Forth shifted in his seat and rested his chin on his hand.  ``Very interesting.  We had known about the stolen CUNY records, but this is the first that I've heard about any involvement with staff members.  Has he made any attempts to contact you since he requested those records on your student?''

``No, not any that I know of.  The request seemed open and shut, very run-of-the-mill.  Until two days ago.''

``It's an interesting story, that one.  The kid was here to answer some basic questions about a carjacking that happened in his neighborhood.  He wasn't involved, but he witnessed the crime and gave us a very detailed description of the thief and the events that unfolded.  He has a great memory, and he can be very convincing.  No hint of nervousness, anxiety, or self doubt.  The officer who was asking him questions stepped out for a moment, assuming that he could be trusted.  That's when you talked to him.  He had worked out the timing to perfection.  We \emph{still} can't figure out how he got the paperwork that he sent you.  Everything was done by the time the officer stepped back in, and to him---as to you---nothing seemed amiss.

``We'll need you to make a statement for the record, Mr. Winter, just to keep things air-tight.  But that doesn't need to happen today.  There are a few more details that I think you'll find noteworthy.  This guy has made a habit of prowling around universities.  He knows how they work.  He knows the security routines, the lax judgment, the sluggishness of administation to do anything in a reasonable amount of time, \emph{et cetera}.  They're big, slow moving beasts, and he's someone who knows how to be light on his feet, watching for the right moment to make his move.  We suspect that he's defeated the security systems at three schools so far.  He's used those skills to change grades, but on one occasion he held a security guard hostage for six hours and forced him to share a bunch of political dirt from closed-door board meetings.  How he knew that the guy had been in the meetings is anybody's guess.  If there's anything that you've noticed that fits that bill, this would be a good time to share.''

``We've had a couple of incidents like that,'' Sascha said.  He glanced at Jesse before continuing.  ``Most recently, we discovered that our maintenance ticket scheduling system has a security glitch.  When a ticket is created and then deleted in just the right way, our door security system doesn't get the memo.  It still unlocks the doors during the non-existent repair window.  Jesse's office was scheduled for some repairs a couple of weeks ago, but there was no official record of the work to be done.  He had been working long hours and fell asleep at his desk for a bit.  I had gone down to talk to him around the same time and found him asleep, but the door to the office was ajar.  This is a heavy, magnetized door with an alarm that triggers if the door is open for more than a minute.  When I woke Jesse, he had no idea that anything had happened.''

``Until we found the bug on my jacket,'' Jesse said, his eyes downcast.

``Yes,'' Forth said, looking distracted for a moment.  Then his eyes sharpened again, and he looked back at Jesse.  ``Let's talk about that.  You said that you found an electronic recording device attached to your jacket.  What did it look like?  Do you still have it?''

``No, we don't have it anymore,'' Jesse said, stifling a laugh as he recalled the bug's fate in the cafeteria kitchen.  ``It was made of plastic.  Black.  Smaller than a dime, but the same circular shape.  It was thin like a coin, too.  I didn't see any markings on it.''

Forth took notes on his yellow legal pad as Jesse spoke.  ``All right,'' he said, ``and where was it attached to your jacket?''

``Under the collar.  Back side of the neck.''

Forth stopped writing for a moment as Jesse said this.  After resuming and finishing his notes, he sat back in his chair and twirled his pen.  ``On at least eight occasions,'' he said, ``this guy has planted bugs on people.  All of them have been under the collar.''  He sat motionless for a moment (except for the pen), waiting for Jesse and Sascha to digest this news.

Jesse looked at Sascha and shook his head in disbelief.  Sascha had been right.  Again.  And Jesse had been distracted by the antics of a university prosecutor, completely disregarding the fact that someone had been spying on him and had been close enough to do him physical harm.

``That's the main reason why we're here,'' Sascha said.  ``Once the news broke about an NYU student who had stolen records from CUNY and we learned about his other exploits, the story began to make sense.  We knew that our chances of understanding the big picture were much better if we talked to the NYPD.  We think that Jesse's dealt with some of the student's political contacts in Europe in the past as part of his job, so we might have something to contribute to the investigation.''

``Can you tell me a bit more about that, Mr. Winter?'' Forth asked.  ``What does your job entail?''

``Sure,'' Jesse said.  ``CUNY has a well-known investigative journalism program.  We teach students how to do the hard work of looking for good, substantive stories, and then we work to improve their interviewing, research, and diplomatic skills.  It's worked well so far.  We've sent students to 80 countries, and many of them bring back work that's good enough to publish.  In many cases, it's also good enough to land them a job at a respected news agency or private investigative firm.  As you can imagine, though, their work can lead them into unknown territory.  We've had eleven students end up in court as a direct result of their research projects.  That's a small percentage, but the cases can be very serious.

``I hold a special position that deals with the thorny details that our students run into.  I do other things as well, but that's how I spend the bulk of my time.  I've negotiated on behalf of the students, I've posted bail for them, and I've met with foreign diplomats to arrange for their safe return to the States.  It isn't always pretty, but I believe in the mission of the program, and I think we're all better for it.''

Forth nodded as he took more notes.  He was on his fifth page already and would be onto the sixth soon.  ``I have to agree with you.  Off the record, of course.  So, I've heard of the program.  The mayor's office and the NYPD have received a number of complaints about it, mostly from people who don't understand civics.  The rest probably complain directly to CUNY.''

Sascha laughed.  ``Yeah, we get our fair share.''

``It's a normal thing when you're pushing boundaries and trying to root out corruption.  A lot of times, it's been going on for decades, and you're just the latest speed bump.  Young guys like this end up doing the dirty work.  It's good that we caught him when we did, for your sake.''

``What's he like?'' Jesse asked.  ``Personally, I mean.  Have you interviewed him yet?''

``Yeah, I have,'' Forth said, opening a manilla folder and pulling out a batch of papers.  ``His given name is Ingo, but he goes by Josh when he's in the States.  His European friends all call him Ingo, from what we've gathered.  He's five foot three and has wiry brown hair.  He's crafty, as you already know.  Some of our interviewers have gotten into arguments with him and found themselves tangled up in arguments over syllogisms and world history.  He's been a tough nut to crack.

``We've been in touch with Hungarian officials.  Ingo went to the University of Budapest and took a degree in political science.  Before that, he did his obligatory military service and got special forces training.  He specialized, unfortunately for us, as a hostage negotiator and military psychologist.''

``I've known a few of those,'' Sascha said, suddenly looking worried.  ``They're the type of guys who can convince you that your brown hair is actually blue.  That everything you've ever known was a fraud.  I'm not surprised that the interviewers have struggled when they sparred with him.  Think Hannibal Lecter, but with rifle training and a government sanction to kill or maim when it's expedient.''

Forth nodded and pursed his lips.  ``That's the trouble.  We've had to call in some expert interviewers, so the case has been held up for a few days.  A couple of the guys we requested are from the Navy.  They'll be here by Friday.''

There was a moment of silence as the conversation stalled.  Nothing that they had learned so far had been surprising, but Jesse felt much closer to the man who had been spying on him than he had previously.  Ingo was probably being held somewhere nearby.  By now, he would have worked out how to convince the guards to let him have a pen and paper to work on a novel.  They might even give him paperclips, thoroughly convinced that he's being held for some petty crime.  Jesse shut it out of his mind.

Then he had a thought.  ``Detective, you mentioned earlier that he's bugged a number of people.''

``Yes, that's right,'' Forth said.  ``At least eight.  At least nine, including you.''

Jesse leaned forward.  ``Has he told you how he did it?  How he planted the bug, I mean.''

``He's given us a few details, but I would be interested in hearing your intuition.  When were you vulnerable on the day you were bugged, and where do you think it happened?''

``Well, we checked the surveillance video footage for my building on that day, and the only person who entered my office when the door system was off was an administrative assistant whom I've known for years.  She must have seen that I was asleep because she didn't stay more than a few seconds.''

``And she didn't notice that the door was unlocked?''

``A lot of our staffers get used to the routine of unlocking the doors.  She keyed in her passcode and turned the knob, which would have opened the door on any other day.  The keypad doesn't give any audio or visual feedback.''

``That's a mark against your security contractor, but probably not against the assistant.  All right, so she seems an unlikely collaborator.  What else?''

Jesse thought for a moment, recalling the day.  ``I took the bus to work, like I always do.  There was an engrossing economics article in the \emph{Times}, and I tuned everything out while I read it.  My suit jacket was hung over the back of my seat for the whole ride.''

``How long were you on the bus?''

``It takes about thirty-five minutes on an average day.  I get on around 7:20 a.m., and there's a lot of traffic.''

``Plenty of time to install a bug,'' Forth said, scratching out more notes.  ``But it doesn't match his style.  Ingo likes a challenge, and a sitting target doesn't thrill him.  Was there any time when you were walking through town, maybe in a crowd?''

Jesse's face clouded.  He had always worried about someone picking his pocket on the streets of New York.  He'd seen it happen to young people, old people, and even children.  You had to take precautions.  To reduce your attack surface, as the security pros said.  After moving to The City, Jesse had begun leaving his wallet at home.  He would carry a cash clip deep in his pants pocket and keep his credit cards somewhere else.  He never wore expensive rings or watches.  The messenger bag that carried his laptop and sensitive documents was always worn so that the strap was over his neck.

But he hadn't been ready for a bug.

``I---.  Yes,'' Jesse said, his voice wavering.  I always walk a few blocks from the bus stop to the university.  There are a few places where the foot traffic gets heavy in the morning.''

``I know that it's hard to remember,'' Forth said, ``but did anyone brush up against you?  Did anyone try to stop you or talk to you?''

Sascha and Jesse had spent so much time talking about the bus ride that it was difficult for Jesse to remember anything from earlier in the day.  He tried to roll his mental tape backwards, to remember anything that stood out.  Anything that he could use as a mental foothold.  It was a complete blank.

Forth watched Jesse's face as he tried to remember.  ``My mother used to say that adrenaline was like sugar,'' he said with a chuckle.  ``One of them spoiled your appetite, and the other spoiled your memory.  She had the memory part half right.  Adrenaline seems to galvanize your memory of everything after you get the dose, but anything before that is just gone.''

But Jesse wasn't listening.  The word \emph{mother} had trickled into his mind, and in a flash he saw Bedford Avenue.  \emph{Escape velocity?  It's too much---}.  His shoulder ached suddenly.  That morning, someone had bumped into him as he waited to approach the Bedford intersection.

``I know when it was,'' he said, looking up at Detective Forth.  ``That morning, I heard someone say something that reminded me of my mother, of something she used to say all the time.  I stopped walking and looked around to see who said it, but she was gone.  And then someone ran into me, hard.  He apologized and was gone, but that must have been it.  That was Ingo.''

``Possibly,'' said Forth.  ``Do you recall what he looked like?  What he was wearing?''

``Yeah, but only vaguely.  He was short.  Dressed in black.  And I think that he was wearing an old top hat.''

Forth took a deep breath and wrote another note on his pad.  ``Yep.  You've met him.''

Sascha turned to look at Jesse, whose face had flushed.  ``You never told me about that,'' he said.

``I hadn't thought about it until just now.  It was just another bump on the street.  It's probably happened hundreds of times since I moved here, and I've never had anything stolen, or\dots''

``Gifted?'' Forth said, attempting to soften the tension.  ``We had him demonstrate it on one of our guys here.  He's bugged so many people over the years that he can pull it off without giving any hint of danger to his target.  Our reflexes that evolved over hundreds of millions of years of avoiding becoming dinner for a lion or being ambushed by other primates don't stand a chance.  That's the reason why you didn't think twice when he nudged you.  He's good.''

``After he demonstrated the attack for us,'' Forth went on, ``he wanted to talk about it.  Ingo's crafty, but he's also a talker.  He likes to share his big ideas.  Each hit goes something like this, with minor variations depending on his mood and the situation at hand.  He tracks his target from a range of about fifty feet -- far enough that you'd never notice him, especially on a busy street, but close enough that he can get to you within seconds.  This goes on for a while.  He's probably been watching you for a few days and knows where you're going.  Most people don't change up their routes on a whim.  When he's convinced that you're in la-la land, he moves up in the crowd.  If anything changes, he falls back and bides his time.  If everything looks good, he waits for a moment when you have a bit of space around you.  He's right-handed, so it's best if you're on the right side of the sidewalk.  Easier to conceal what he's doing.

``When the critical moment arrives, he lengthens his stride.  The bug is in his between the forefinger and thumb in his right hand.  It comes with a little delivery mechanism that does some neat tricks.  He's walking exactly out of step with you, so that he can put his right foot forward when your left foot is forward.  It's all part of the strategy.  When he's ready, he slams the heel of his left palm into your left shoulder blade.  That doesn't feel nice, and he's counting on the pain as a diversion.  Between the time when you feel the pain in your shoulder and when you turn to see who's hit you, he reaches his right hand up toward your neck.  The delivery tool has a metal prong jutting from it, and he uses that to flip up your collar.  With a gentle squeeze from his fingers, the tool attaches the bug to your jacket the way a stapler binds a staple to paper -- push through, and then snag.

``After that, it's an apology, a tip of the hat, and you get on with your day.  With the pain in your shoulder and the little surprise of his delivering a nice dose of adrenaline, chances are good that you never saw him anyway.  It's good that you heard your mother's words just before he hit you, or we might not have made the connection.''

Jesse nodded, his mind still reeling from Forth's description and what he had just learned about Ingo.  ``So, if that's all accurate, then what should we do?  What \emph{can} we do?''

It was a moment before Forth answered.  ``I think that his gift of the gab is out best asset,'' he said.  ``He loves to talk about the clever things that he's accomplished.  I think his handlers keep a tight lid on him and probably take him for granted.  Ingo is just a tool to them.  A high-quality tool, to be sure, but at the end of the day he's a cog in a big machine.  Traditional psych tricks won't work here.  He's too sophisticated for that.  But if we can get him talking, then we might be able to learn something about why he was spying on you.  If there's a connection to your students, we might learn something about that too.  Time will tell.''

``Seems reasonable,'' Sascha said.  ``What would you like us to do?''

``If you're available next week,'' Forth said, ``I'd like you to sit in on the next round of questioning.  You'll be on the other side of a one-way mirror, but you might be able to help us tease out some good gossip.  If you need a jury summons or some other official paperwork to give to your dean, let me know.''

``I think the Ombudsman will be plenty interested in this case.  We'll take care of it.''

``Great.  You'll hear from me by the end of the week.  In the meantime, keep an eye out for spies.''

Forth stood up, offering his hand to each man across the table.  The conversation seemed to have lightened his mood.

``Thank you very much, detective,'' Jesse said.  ``By the end of this, maybe I'll be able to sleep at night again.''

Forth laughed.  ``That's why we're here.  Right this way, gentlemen.''  He led them back out to the lobby, asked the receptionist to hold his calls for an hour, and bid Jesse and Sascha farewell.


``That was an enlightening forty minutes,'' Sascha said as he and Jesse walked back toward the CUNY campus.  They were walking more slowly than before.  Both were digesting the various tidbits that Forth had shared, and neither had said much since they left the NYPD.

Jesse felt numb.  He was grateful for the openness that Forth had displayed, but the weight of the facts had left him feeling vulnerable and bewildered.  He tried to think through every case that he had worked on over the past year, willing his memory to show a connection.  But there was nothing.  Warsaw was the most likely link, but Jesse knew that he was reaching.  Ingo had worked for eastern European political groups, but there were thousands of them.  Only one or two would know that Jesse existed, and he hadn't heard a peep from them.  If they were going to counter-sue or send mercenaries after someone at CUNY, they would have done so almost immediately.  Attention was always in short supply in political circles, and the goons who did their dirty work had even shorter attention spans.  It didn't jive.

Sascha, for his part, was intrigued by the idea of sitting in on, and even being allowed to participate in, a criminal interrogation.  What would they learn next week?  Maybe the guy would roll over on his bosses.  Maybe he would give the FBI some dirt on few politicians in Europe.  Maybe he would go on endlessly about his skills as a spy for hire.  The odds of convincing him to do anything were poor.  The NYPD could only drop hints and try to lead him in directions that might reveal something interesting.  Hearding cats would be easier by comparison.

``Yeah,'' Jesse said finally.  ``Sorry, I'm just feeling dazed about it all.  Great job putting us in touch with Forth, though.  I feel like we're getting somewhere, even if I don't know where that is, or whether I'll like it.  At the very least, I want to know who Ingo works for.  That will tell us a lot.  If it's someone we've squeezed, or someone who has us on their shit list, then we'll need to follow whatever leads he gives us.  If it's not someone we need to worry about, then I guess it might make an interesting project for next year's junior class.''

Sascha smiled.  ``Always thinking ahead,'' he said simply.

``Yeah,'' Jesse said, ``but apparently I don't think behind very well.  We never would have gotten anywhere with this on our own.''  He paused.  ``I wish that there was some way that I could prepare myself for the questioning.  Maybe we should sit down tomorrow and build a list of possible connections.  All the reasons that he might have for watching us, and the people who might pay him to do it.  I have a list of our projects on my laptop, and it would be easy to pull out all the one that touched European countries.''

``We'll definitely need to do that.  Forth might also want us to be thinking of direct questions that we'd like to ask, even if nobody ever gets the chance.  The more the interviewers know, the better.  Those guys are very creative.''

``Sascha, what if someone is after us?  What if this is an organized plot to watch us and then sabotage the program?  What if our students are at risk?''

``Hey.  It's a big job, and a dangerous one.  The students know that.  Their parents know it, too.  But like you always say, democracy needs them to do it.  That argument works, you know.''

Jesse grinned sidelong at Sascha and rolled his eyes.  ``Jerk.''

They climbed the stairs to the east parking structure.  Neither Jesse nor Sascha saw the woman with close-cropped black hair and frameless glasses walk by, glancing at her watch.


Jesse stepped out of his car, glad to be home.  Some folks went home only when they had to.  Maybe it was for dinner, or to sleep, or to catch up on the bills.  To Jesse, his apartment had always been a sanctuary.  Even though he enjoyed going to work most days, it was a blissful relief to come home to his ordered, clean unit on Blake Drive.  He kept a simple lifestyle, not indulging in the latest expensive computer toy that many of his friends spent half of their paycheck on each month.  He was a geek with a capital `G', but his interest was in learning and solving problems.  Owning complicated gismos that drained precious hours from his day was something that he carefully avoided.

It was a clear, cold night.  Winter would be arriving soon, probably in the form of morning frost and afternoon flurries.  Jesse heard someone shuffling along the sidewalk behind the covered parking spaces.  He smiled and pulled his coat out of the backseat.  His neighbors were always out and about.  They made the neighborhood seem old fashioned.  It was common to see groups of three or more residents milling around outside of an apartment, talking about the latest news or sharing cooking secrets.  Jesse had chosen the right place to drop his anchor.

Feeling warm despite the chill, Jesse closed the back door on his car, tucked his coat under his left arm, and started toward his front door.  He glanced up at the night sky as he walked, admiring the small number of stars that managed to make themselves seen through the thick envelope of light pollution that hung in the air.

As he reached the steps that led up to his door, Jesse fumbled for his keys.  The blow came suddenly, from behind him, and Jesse knew only a flash of bright light before the night sky swallowed everything up.

\chapter{}
										
Sascha woke suddenly and wondered why his alarm clock was going off.  ``It's still charcoal,'' he said to the empty room.  ``Why beep at me?  The ladder outside needs tuning again.''

He rolled over, poking at the snooze button on top of his alarm clock that was resting precariously on the edge of the nightstand.  The noise continued.  Sascha rubbed his eyes, and as the mental fog lifted, he realized that his mobile phone was glowing.  And the phone was ringing.

Without looking at the screen to identify the caller, Sascha grabbed the phone off of the nightstand and punched the talk button.  ``Greene,'' he said, trying to sound alert.

``Mr. Greene, I'm sorry to bother you at this hour.  This is Martha Smith from the SpanTel, and we have an emergency alert for you.  There was an incident at one of the addresses that you registered with us.''

``Which one?'' Sascha said, sitting up suddenly.  SpanTel was a new startup company that monitored law enforcement and emergency medical service notices.  For a monthly fee, someone could register a set of addresses and be notified when an emergency was reported there.  They took care of requesting permission from the owner of the address---a courtesy, since the notices were public information---and had staff available 24/7.  Sascha had registered the addresses for his parents' house, their cottage, the building he worked in at CUNY, and several of his close friends.  Whoever this call was for, it was important.

The voice said: ``Sir, the address is 481 Blake Drive.''

\emph{Jesse}, Sascha thought, his pulse accelerating.  ``What happened?'' he said, a bit more harshly than he intended.

``It's not clear, sir, but the call was placed by a neighbor who witnessed a physical assault and possible mugging.  Would you like to req---''

Sascha flipped the phone closed.  His mind raced.  ``I knew I should have called him earlier,'' he said.  His voice echoed off of the bare walls.

There was only one thing to do, Sascha decided.  He got up, pulled on his jeans and a t-shirt, and ran to his kitchen.  He pulled a box of .380 ACP ammunition from a small drawer and went back to the bedroom.  Retrieving his Beretta handgun from a drawer built into the back of the nightstand, he loaded a cartridge and tucked the gun into his belt.  Satisfied that he was ready, Sascha grabbed his car keys and ran out to his car.

\emph{Driving at night is my catnip}, he thought as he ran through the gears and merged onto the expressway.  \emph{No worries, no obstacles, and plenty of speed}.

Sascha knew that it was serious as soon as he rounded the tall hedges that bordered Blake Drive.  At least six police cruisers were parked in the driveway of Jesse's apartment complex.  Each had its spotlight on and pointed toward Jesse's front door.  An FBI van with tinted windows was across the street with its engine running.  A uniformed officer stood in the middle of the road and held up a flat palm to Sascha as he approached.  The officer approached Sascha's car, and Sascha rolled down his window.

``Good evening, officer.''

``Hello there.  We're doing an investigation here, so this road will be blocked off for a while.  You can head back the other way and cut through on Thomas Road.''

``Thanks, but I think I know the guy who was assaulted.  His name is Jesse Winter, and I'm a good friend of his.  We work together at City University.  My name is Sascha Greene.''

The officer looked at him for a moment.  ``Just a second, sir,'' he said.  He stepped away from the car and spoke into his radio.  Sascha only caught fragments of the words that were exchanged, but it sounded promising.

After five or six exchanges on the radio, the officer stepped back toward the car.  ``Mr. Greene,'' he said, ``please park your car over there by the hedge.  We'd like to ask you a few questions, if you don't mind.''

``Of course,'' Sascha said, putting the car in reverse.  \emph{Dunno what they want with me, but I want to hear Jesse speak and see him move his limbs}.  He parked the car, killed the engine, and opened his door.  In a moment of clarity, he nonchalantly reached down to his waist, retrieved the Beretta, and tucked it under the seat.  \emph{No need to get off on the wrong foot here}.

Sascha walked with the officer toward the apartment building.  He could see a group of people huddled together beside the steps to Jesse's apartment.  More than half of them appeared to be emergency personnel, and the remainder were a mix of uniformed police and men in unmarked black coats.  Sascha suspected that Jesse was at the center of the huddle, but his eyes couldn't see a familiar shape in the dark mass.

``The feds are here, too,'' the officer said in a low voice.  ``It could be a long night for all of us.''

``Do you know what happened to him?''

``Blunt force trauma.  That's all I know.  A witness said that she knows more, but you'll have to talk to the detectives about that.  She probably knows more about what happened than your friend.''

As they reached the edge of the group, one of the men dressed in black turned to face them.  The officer introduced Sascha and mentioned that he was a good friend of Jesse's.  ``Ah, thanks for coming by,'' the man in black said.  ``Let me guess, SpanTel?''

``Yes sir,'' Sascha said.  ``I signed up for it earlier this year.  Apparently it works.''

``Indeed.  Mr. Greene, I'm Agent Tim Felton with the FBI.  Mind if we ask you a few questions?''

``Of course,'' Sascha said.  ``Can I see Jesse first?''

``There will be plenty of time for that.  The medical team is checking him over right now.  Please step over here.''  Felton walked toward a police cruiser a few paces away, and Sascha followed him.

``How do you know Mr. Winter?''

``I work with him.  We're both staff members at City University.  Office of the Internal Affairs for me, and Office of the Ombudsman for Jesse.  We've been good friends for years.''

``Do you know where he was today?''

``Well, we both went to work.  I saw him a couple of times during the day.  We walked down to NYPD headquarters for a meeting, and then we walked back.  It was dark by the time we got back to campus, and we both headed home.  Everything seemed fine.''

``What time did you last see him?  Do you know if he was going anywhere else?''

``If he was, he didn't mention it.  I don't know an exact time, but a little while after the sun went down.  Maybe six-thirty.''

``So everything seemed fine at that point.  You didn't see anyone following you today?''

``No, nobody.  But we were in crowded places for most of our walk, so it would be hard to tell.''

``Has anyone made threats against Mr. Winter recently?  Would anyone have a reason to hurt him or steal from him?''

Sascha took a deep breath, thinking about how to answer the question.  ``Well,'' he began, ``it's complicated.  Jesse found a bug---sorry, a surveillance device---on his jacket a few weeks ago.  He deals with a lot of tough legal issues in his job, so it's natural that he ruffles some feathers.  We never expected someone to spy on him, though.''

Felton frowned, pulling out a thin notepad and scratching some notes.  ``Has that incident been reported?''

``Yes.  That's the main reason that we went to headquarters today.''  This wasn't precisely true, but Sascha wanted to avoid going into full detail about the visit.

``All right,'' Felton said.  ``Who did you talk to down there?''

``Detective Forth.''

Felton wrote down the name and closed his notebook.  ``All right.  Thanks, Mr. Greene.  We'll be in touch again, since this doesn't seem to be an isolated incident.  You can go see your friend now.  It looks like they've cleaned him up.''

``Thank you,'' Sascha said, walking quickly back to the mass of medical staff that still surrounded Jesse.  He pushed through them, eager to hear his friend's voice.

Jesse was on a stretcher, his head and torso secured with straps.  There was a wide abrasion on his cheek and blood on his right hand.  One of the paramedics was preparing his hand to be bandaged.

``They tell me the straps are just a precaution,'' Jesse croaked, attempting a smile.

Sascha breathed a sigh of relief.  ``What did you do, man?'' he said.  ``Piss off the evening mail man?''

``No idea, Sascha.''  Jesse coughed, his whole body convulsing slightly.  ``I remember pulling into the driveway, and then nothing at all.''

Sascha put his hand on Jesse's arm and squeezed.  ``Hang in there.  I'm going to get some answers.''

Sascha turned around.  There were at least ten uniformed men and women standing near him, but none of them stood out as especially sociable.  He saw Felton in the distance and walked toward him.

``Agent Felton,'' Sascha said.  ``Sorry to bother you.  One of the officers mentioned that your crew interviewed a witness to the attack.  Can you tell me what you learned during the interview?''

``We'll put the details in the police report,'' Felton said.  ``I'll give you the basics, though.  One of his neighbors made the call.  She happened to hear a car door close, and she looked out the window.  A woman was walking behind the parking shelter.  She said that the woman had a strange gait, like she was wearing heavy shoes.  Anyway, she saw Mr. Winter emerge from behind his car and walk toward his apartment.  The woman rounded the edge of the shelter and headed toward him.  She stayed on the grass, even though it would have been easier to walk on the sidewalk.  As he reached the steps, she clubbed him on the back of his head with something.  He didn't fight, and he didn't move once he was down.''

``And then?''

``It's not entirely clear.  The witness thinks that the girl rummaged through Mr. Winter's messenger bag.  She couldn't tell if anything was stolen.  The girl seemed satisfied after a moment, and she ran off -- staying on the grass again.  There's no surveillance equipment at this complex, so we have no idea where she went.''

``Huh,'' Sascha said.  It was all he could think to say.

Felton looked at him steadily, waiting for him to speak again.  When he didn't, Felton continued.  ``Do you know what Mr. Winter was carrying?''

Sascha frowned, trying to remember whether Jesse had opened his bag during their travels earlier that day.  ``I know that he was carrying a notebook, and he normally has a laptop with him.  Beyond that, I don't know.  He's too smart to carry anything sensitive with him.  His laptop's hard drive is encrypted, and he's the only one who knows the decryption key.''

Felton pulled out his notepad and made some more notes.  ``There was no laptop in the messenger bag when we found him.  He couldn't remember whether he had left it at the office.  His notebook is still there, and there's a small wooden clock.  Nothing else.''

``Well, it'll be easy to find out whether he forgot the laptop,'' Sascha said.  ``But what then?  If this woman stole it, and she's smart, then we won't be able to trace it.  There's no video surveillance.''  His confidence was beginning to wane.

``We're going to try fingerprinting.  If that doesn't teach us anything, then we'll probably lose the trail.''

Sascha looked toward Jesse and the paramedics and sighed.  Two invasions of his privacy and autonomy.  Two \emph{attacks}.  It was no longer an isolated incident.  The consequences began to flood into his mind.  Changing his habits wasn't enough.  It didn't matter what streets Jesse took to work, what bus lines he frequented, or whether he left at 7:00 or 7:26 a.m. to go to work.  Whoever was tracking him knew where he worked and where he slept.  Even if Sascha had walked him home, they both could have been ambushed.  The two places where Jesse felt most alive, at home and at CUNY, wouldn't be available to him any longer.  For the first time since he had been deployed in a combat zone, Sascha felt trapped.

``He'll need to find a new place to stay for a while,'' Felton said, sensing that Sascha was feeling overwhelmed.  ``We have some guidelines for people who are at rest.''  He glanced around before he went on. ``Not here, though.  We'll brief him somewhere secure once he's medically cleared.  You're welcome to stick around.''

``Thanks,'' Sascha said.  ``I'll see what I can do about his spirits.  He's never liked being the center of attention.''

Sascha walked back to the stretcher where Jesse lay, now with a nicely bandaged hand.  He glanced around as he walked, hoping to spot someone lurking in the shadows.  Nothing.

Jesse seemed to sense that Sascha was still on the hunt.  ``Don't bother,'' he said.  ``She's long gone.  Probably took my laptop thinking that she had hit the jackpot.  Too bad everything on it is just random noise without the right key.''  He coughed again, wincing as his bruised ribs ached.

``How are you feeling?'' Sascha asked.  ``Head-wise, I mean.''

Jesse's smile faded.  ``I was hoping that you wouldn't ask.  I was in such a good mood on the way home.  Now I feel like this is the only safe place in the world, and as soon as these guys leave, there won't be any safe place.  I could handle it when I thought that I had been bugged at work.  Getting bugged in the street was fine.  Anything can happen in a public place.  This is my home, man.  They know where I live.  What if they come back another night when I'm asleep?''  He looked away, his eyes glistening.  ``It makes me doubt everything.  How can I go to work?''

One of the paramedics lifted Jesse's left arm and slipped a baggy sleeve over it.  ``Almost done, Jesse,'' he said.  ``We need to check a couple more things and you'll be good to go.''

Sascha stuck his hands in his pockets.  ``Don't worry about that.  I was talking to Agent Felton over there, one of the feds, and he has some ideas that'll keep you safe.  The important thing is that you're all right.  We'll find these guys, and they'll realize that they kicked the wrong hornet's nest.''

Jesse sighed.  ``Hey, how did you know that this happened, anyway?  And don't tell me that you happened to be in the neighborhood.''

Sascha laughed.  ``Well, you remember that SpanTel gig that you told me about?''

Behind them, the lights of the city cast a pale glow on the shelf of clouds that had rolled in while nobody was looking.

\chapter{}
											
Jesse looked around at the furnished but bare apartment and wondered, as he always did, who else had lived there before him.  This was New York, after all.  The city of dreams.  A troubled artist might have sat by the window with her feet on the radiator, avoiding the first stroke on her break-through painting.  A young entrepreneur could have had leaning stacks of books everywhere, empty pizza boxes punctuating the empty space that he hadn't yet filled with a new idea.  Whatever it had been before, the drab feel of the apartment made Jesse glad for his imagination.

Setting his suitcase on the floor, he wandered into the living room and sat down on the sofa next to the front window that overlooked North Piata Street.  Across the street was a pharmacy, a shop that sold model airplanes, and a sushi restaurant.  Jesse pulled out his new smartphone and pushed a few buttons, trying to find the screen to make a call.  Nothing was right.  His clothes didn't even seem to fit.

Finding the right screen, Jesse dialed the number for the Chinese restaurant across the street and ordered some salmon and rice.  He tried Sascha's number next, but his voicemail picked up immediately.  Jesse sighed and leaned back into the sofa, closing his eyes.

Agent Felton had been professional but firm about Jesse's choices after he was attacked.  The fact that he had been the sole target of two separate incidents was reason enough to take steps to secure his safety.  Jesse had tried to argue that the purpose of the attacks was to steal information rather than to harm him.  Felton had listened patiently and then shared a story about a recent case that resulted in the violent death of an innocent civilian who had also been the target of spying.  Jesse was told that he needed to move house, but it wasn't a simple process.  A police escort had to be called whenever Jesse went to his old apartment, and his personal effects could not be taken directly to his new apartment since that would reveal its location.  The feds had handled similar scenarios before, and they took care of transporting everything that he needed.  Non-essential things were put in storage outside of the city.

The girl who attacked him had taken his laptop.  It was the obvious thing to do.  Jesse did most of his correspondence on the computer, and the university kept as much paperwork as possible on the computer to reduce waste and keep its processes efficient.  If Jesse had worked with something sensitive that was interesting to the people following him, then there's a good chance that it was cached on the laptop.  Unfortunately for them, the laptop's hard drive was automatically erased at the start and end of every work day.  Jesse pushed a special button on the keyboard when he left for coffee each morning, and again when he left to wash out his cup at 5 o'clock.  The computer took care of synchronizing any document changes that he had made that day, erasing the hard drive, and reinstalling the basic system so that the computer could be used again.  The system depended on a human to start it, but it was otherwise a brilliant piece of automation.  Whatever this girl had hoped to glean from the laptop's hard drive contents, she would soon find out that it was a more effective paperweight.

It was the phone that broke Jesse's courage.  His smartphone was protected with a four-digit security code, but that would provide no defense against someone with technical skills.  His parents' phone numbers, Sascha's work and mobile numbers, and countless others were on there.  He had set up the phone exactly the way he liked it, and it had taken months to get everything right.  The model in his hand was almost an exact replica of the stolen phone, but somehow it felt too bulky.  It wasn't \emph{his} phone, and the awkward mismatch felt like a microcosm of every aspect of his life.

Jesse wasn't the type of guy to ask ``why me''.  Sometimes bad things happened, and sometimes they wandered in packs.  The good news was that, on the average, there were good things that came around in packs too.  You just had to keep your head and not sink into a pit of self-pity.  For Jesse, this was far more comforting than the convictions of people who saw intention in everything.  The hand of god, if it existed at all, didn't affect things directly.  He was content to let the universe do its thing.

The current task was to keep his mind off of things until the sushi arrived.  Jesse had waited a bit too long to eat, and that didn't help his mood.

He tried calling Sascha again.  This time, the phone rang.

``Greene,'' Sascha's voice said.  He was using his professional voice, which told Jesse knew that he didn't recognize the new number.

``Hi,'' Jesse said.  ``I'm interested in buying stock in SpanTel.  Do you have any for sale?''

Sascha didn't recognize the number, but Jesse hadn't done anything to conceal his voice.  ``It wasn't the call that I wanted,'' he said, ``but that's a slick service that they offer.''

``Did you really want any call at that hour?'' Jesse asked.

``Have you ever tried to turn your alarm clock off and then slowly realized that your phone was ringing?''

``No, but you know that I'm a light sleeper.''

``I wouldn't wish it on you.  I mumbled a few things that were probably incoherent, but luckily the girl at SpanTel let it ring more than twice.  Wish the telemarketers would do that.  Anyway, what's going on?  You get your new digs?''

``Yeah, I just brought some clothes and books.  The rest of the stuff is in storage.  I plan on moving again as soon as we're done with the case, and I'm not moving all that furniture twice.''

``Good thinking.  Hey, I heard from Forth today.  He and Felton got together to compare notes after the attack.  They're convinced that the two events were related.  That much seemed obvious, but it's good that the feds and the NYPD agree on it.  They don't want you to tell anyone where you are -- even me.''

``All right.  Dry gossip so far.''

``I'm getting to the good stuff.  He mentioned that Ingo is being interviewed tomorrow morning.  They want us both to be there.  An agent will be calling you to set something up.  They had a chat with Ingo after you were attacked, and I guess he shared some interesting stories.  Whoever his bosses are have a number of sleeper agents that they call up whenever they need something.  Most of them are college kids.  People who need money and aren't above getting a little physical as long as no serious harm is done.  They offer the kids full deniability.  Posters are put up around campuses offering easy work--- part time, great hours, flexible schedule, the usual tripe---and when someone calls, it's a recording.  They leave the basics on the answering machine, and a few hours later someone calls back from an unlisted number.  If the kid gets spooked, then they hang up.  No harm, no foul.  But plenty of them will stay on the line.

``When a job gets sent out, it's delivered on a sheet of paper by a private messenger who takes cash.  No paper trail.  Sent with the sheet of paper is a key to a storage locker somewhere obscure.  The worker drops off the goods in the locker, leaves the key in the locker, and waits for another messenger to bring an envelope of cash.  You got hit by the master the first time, but this second hit was probably some econ major with a bad haircut.''

``So Ingo is definitely our point of focus, then.  He's been dancing in the same circles for a while, and he'll have some idea who calls the shots.  He might know the puppeteer, or maybe he can show us the money trail.''

``That's the idea, my friend.  Enjoy your evening.  I've got to get this top feed sorted out before I go home.  I'll catch you tomorrow at headquarters.''

``I'm not on the feed, am I?'' Jesse asked, half serious.

``Let me see,'' Sascha said, scanning the list.  ``Nope.  Better luck next time!''

``Right.''

\chapter{}
										
NYPD headquarters was a zoo.  As Jesse stepped through the door into the lobby, he didn't recognize it.  There were reporters everywhere he looked, setting up their equipment and doing microphone checks.  A few guys in expensive-looking suits stood in one corner.  They were either politicians or lawyers, Jesse guessed, and that was reason enough to move toward the other side of the lobby.  The secretary who had helped Jesse and Sascha the week before was swamped.  If she wasn't answering questions on the phone, there was someone approaching her desk with papers or inane, I-already-answered-that-for-you-sir questions from the reporters.  Looking around, though, it was clear to Jesse that nobody else in the room could help him.

Jesse made his way toward the secretary's desk.  He did his best to stay out of cameras' lines of sight, which reminded him of action movie scenes when the hero contorts himself to avoid detection by red laser beams.  When he finally found the desk, the secretary had stepped away to resolve a dispute between reporters from the Sky News affiliate and another camera crew who insisted that they had left their gear \emph{on} the blue chair, and \emph{not} beside it.  She returned a moment later, shaking her head and looking harrassed.

``Grown up question for you,'' Jesse said, offering a smile.

``This will be a first,'' she said.  ``How can I help you, sir?''

``I'm here for a meeting with Detective Forth.''

``Yes, of course.  Is he expecting you?''

``Yes, he is.''

``All right.  I'll ask his secretary to send someone out to get you.  Stay right here, please.  If I ask you to move somewhere else, I might never find you again.''

Jesse took the spare moment to stretch his neck and shoulders.  He checked his phone to make sure that Sascha hadn't tried to call.  \emph{No canceling this one}, he thought.

He listened to the banter of the reporters.  From the chaos, he was able to work out that there had been a scandal involving the president of a New York bank, and he was expected to turn himself in any moment.  Why every camera crew in the area needed to broadcast a live remote from NYPD headquarters that wouldn't even show the man walking in the door was beyond Jesse's comprehension.  He had grown up on nightly news and magazines, and the dizzy media blitzes that filled the airwaves left him feeling empty.  It's all empty calories and no fiber, he told Sascha one day at work.  ``At least the news can't make us fat,'' Sascha had retorted.

Just when Jesse thought that the noise might drive him mad, a young man dressed in a black shirt and jeans emerged from the crowd.  ``Mr. Winter?'' he called to everyone nearby, raising his voice.

``Yes, right here,'' Jesse said.

``Right this way, sir.''

He led Jesse through the crowd to the same door that he had gone through on his last visit.  As the door closed behind them, Jesse heard the hiss in his ears that accompanied a sudden drop in noise level.  Everything was now too quiet, but at least he could think clearly again.

``Mr. Winter,'' the young man said, ``I'm Bradley Johnson.  I'm an intern here, working with Detective Forth.''  He lowered his voice slightly.  ``I was a graduate of CUNY's investigative journalism program last year.''

``Ah!  Pleased to meet you, Bradley,'' Jesse said, offering his hand.  ``I hope that I'm not too late.''

``No, no.  They're just getting set up.  It takes a while to prepare the room, get everyone ready for the interview, and then move the suspect.  You know, we don't see too many outsiders in these interviews.  I'm hoping that you'll bring some fresh angles.  This guy is a real nut, and none of the traditional methods have yielded anything.  He's too smart for that.  Loves to tell stories about himself.''

``Sascha Greene and I made a list of questions that might tease out some nuggets.  Sascha sent it to Detective Forth a few days ago.''

``We've worked those questions into the line-up.  We'll need you to listen for them, and if you have any suggestions for the interviewers, then shout them out.  We want to get the guy to open up and keep him talking.''

They rounded a corner, and Bradley ushered Sascha into a dim room.  There were two tables pushed agaist a wall, a gray coatrack in the corner, and the wall appeared to be glowing.  Once his eyes adjusted, Jesse realized that he was looking into a one-way mirror, and on the other side was a bright interrogation room.  Inside the room there were two simple chairs that reminded him of the cheap variety that his elementary school had used.  He discovered, with relief, that the chairs on the near side of the mirror were cushioned and comfortable.  There were four chairs in total, two at each of the tables.  A pitcher of water and six plain-looking glasses sat on the table to Jesse's left.

``Please make yourself comfortable, Mr. Winter,'' Bradley said.  ``I'll be back in a few minutes after we finish our prep work.''

Jesse nodded.  He pulled off his coat and hung it on the coatrack.  Retrieving his phone from one of the pockets, he checked the time.  It was nearly 4:15 p.m., and both Jesse and Sascha had been asked to drop by at 4.  Jesse used the free moment to send an SMS message to Sascha's phone.  Given the circumstances, Jesse wanted positive contact.  Any number of delays could have held Sascha up, but he almost always responded to messages.

\chapter{}

\emph{Three blocks.  You couldn't do this in another \emph{three blocks} when I'm next to the cop shop?}  Sascha relaxed his tense frame and glared into his rear-view mirror.  Looking back at him was a young man, maybe 20 years old, whose eyes were wild with panic.  The man's face was larger in Sascha's mirror than it should have been.  Seconds earlier, Sascha had stopped to let an ambulance cross the street, and a blue Mercury Sable that had been drifting from one side of the road to the other---\emph{probably texting}, Sascha had thought---slammed into the back of his car, shattering the rear window and puncturing the fuel tank.  Traffic around them had continued to flow, but a few drivers behind the Sable had gotten out of their cars to investigate.

``Well, we're all still alive,'' Sascha said aloud.  ``Less paperwork this way.''  He unbuckled his seatbelt and stepped out of the car, getting a strong whiff of gasoline vapors from the fuel that was pouring onto the road and flowing backwards.  He let the door rest against the body of the car, being careful not to create a spark.  Then he took a few steps toward the Sable and gave a curt wave to the bystanders who were approaching.

Sascha looked inside the car that had hit him.  ``PUT THAT OUT!'' he yelled suddenly, much to the surprise of the bystanders.  The man had produced a cigarette and was taking a long drag on it.  His look of panic returned as Sascha sprinted to his window.  ``The car you just smashed into is \emph{leaking gas}, and it's flowing right underneath you.  Unless you want to burn up along with \emph{your} car, I suggest you snuff it out.''  Sascha was now leaning in the window, and the look on the man's face was one of sheer terror.  With a shaking hand, the man pushed the cigarette into the ashtray in the car's console.  ``S-s-sorry,'' he stammered.

Sascha pulled his head and shoulders out of the car and looked around.  Two vehicles in the adjacent lane had slowed to watch.  The bystanders who had left their vehicles were glancing at one another and slowly backing away.  Sascha put his hands up in a sign of apology.  ``Sorry, folks,'' he said.  ``I just have a strong aversion to dying in a fireball.''  He backed away from the Sable, giving the man enough space to open the door and a little extra for good measure.

``I'm Sascha Greene,'' he said.

The man looked at him warily.  ``Joe Marek,'' he said finally.  ``Sorry about your car.''

``Don't worry about it.  I'm glad that you're all right.''

Sascha turned toward the group bystanders.  ``Can someone please call this in?'' he asked.

One of the men responded that he had called 9-1-1 a few minutes before.  Sascha walked back to his car to find his phone.  He was going to be late for the meeting at the NYPD, and he knew that Jesse would be worried about him.  His phone had moved from the console between the front seats to the rear window.  Not stopping to work out how it had bounced up there, Sascha carefully pulled the phone out of the glass shards and entered his passcode.  He had an SMS message from an unfamiliar number asking where he was.  The number matched one of the calls in his recent call log, and he recalled that Jesse had called him from his new phone number.  He sent a message back: \emph{Car accident. All ok, but I'll be late. Don't wait up. Only 3 blocks away, but there will be paperwork\dots}

Sascha saw a police car moving slowly toward him, honking at the traffic to break up the rubbernecking.  He sighed.  \emph{My first interrogation.  Down the tubes.}

\chapter{}

Jesse felt his phone vibrate in his pocket.  He pulled it out and looked at the screen.  \emph{Car accident?}  He felt his adrenaline spike.  \emph{Not again, not another attack.  Not Sascha.}  It took three re-readings of the remainder of the message to convince himself that running out the door and into the street wasn't necessary.  Everything was all right.  \emph{Focus}.

He hadn't planned on working with the interviewers by himself.  Jesse was the brainy one, cautious and slow when it came to thinking on his feet.  He had taken up debate in order to tame his fear of becoming tongue-tied and getting humiliated by someone who was cooler under pressure.  It was an irrational fear.  Jesse had always done well at speaking engagements, but he struggled to reconcile his thoughtful side that ached for justice and calm rationality in the world with the firy, emotional rhetoric that was sometimes required to knock people off of their cozy ideological perches and secure that justice.  Sascha was his front man, the one with the instincts that almost always found the sweet spot in complicated situations.  Sascha's dead reckoning had kept him alive in dangerous situations at least twice.

\emph{Focus}.  Jesse closed his eyes, took a cleansing breath, and then looked around the room.  Nothing had changed, but his mind was now in detail mode.  There was a box on the wall next to the mirror with a set of buttons and two circular dials.  He guessed that the box was an intercom that was finely adjustable to provide a specific experience on the far side of the mirror.  If they wanted to be quiet and sensitive, or perhaps loud and intimidating, the technology was there.  The lights were probably tied into it as well.  Maybe even the color of the walls.  There were a few plugs below the box that looked like digital video connectors, but Jesse couldn't be sure.

He let his eyes relax, and the interrogation room came into focus.  Every wall looked the same: plain white with a wide black rectangle in the center.  There was a small gray square above each of the rectangles.  The larger ones were probably mirrors like the one he was peering through, but there was no way to be sure.  In the center of the room, the two chairs were set facing each other, angled slightly away from the mirror.  Jesse thought that this was probably strategic.  Ingo would be expecting the chairs to be set so that the people on the other side of the mirror could see his face.  He would probably try to turn that into an advantage, but their unorthodox viewing angle from the side room might provide some surprising benefits.

Jesse shielded his eyes from the glow of the lights in the interrogation room and tried to see its ceiling.  It was an alarmingly tall room.  The bright lights dangled on long, dark cables that were flanked by silver chain links.  The ceiling itself was flat with no sound-dampeners.  It would be a noisy room.  That might play into Ingo's sense of his own grandeur, giving him the sense that he was speaking in a large hall.  The ceiling was tiled, too.  Jesse wondered if the echoes ever really stopped.

Just as he was studying the door and its hinges, a man came in with a cart piled high with computer monitors and cabling.  He grunted a greeting at Jesse and began unloading the gear onto the tables.  As he connected power cables and hooked the monitors into the box on the wall, a white picture began to appear on the screens.  Once the monitors warmed up, Jesse realized that he was looking at several angles of the interrogation room.  He tried to orient himself by comparing the picture on the monitors with the room that he saw through the mirror.  \emph{It must be the gray squares}, he thought.  If his reckoning was correct, the third monitor from the left was the one mounted on the opposite wall from where he sat.  The camera images were incredibly sharp, but the black rectangle on the screen gave no indication that someone was sitting behind it.

The man finished his work and left the room, taking the empty cart with him.  Jesse watched him leave and then turned back to the monitors.  A moment later, Forth walked into the room holding a mobile phone to his head and carrying a stack of folders.

``--need you to get the signatures for all three,'' Forth was saying.  ``N-No.  No.  The first way.  Yes.  All right.''  He smiled at Jesse and set the phone on the table along with the folders.

``Good to see you again, Mr. Winter.  Is Mr. Greene on his way?''

``Well,'' Jesse said, ``he's going to be a few minutes.  There was a car accident.  Everyone is fine, and he's only a few blocks away, but he doesn't want us to wait for him.''

Forth gritted his teeth and squeezed his eyes closed for a moment.  ``Ouch.  Well, we'll do what we can in the meantime.  Are you ready to go on without him?''

``Yep, I'm ready.  Will you be bringing Ingo in soon?''

``Five minutes.  He's being brought into the building as we speak.  The hubbub outside slowed the convoy down.  You wouldn't think that a washed-up politician turning himself in after getting caught with his hand in Wall Street's cookie jar would bog down every road around NYPD headquarters, but here we are.

``Anyway, he'll be coming through the door behind one of our officers.  There will be another officer following him.  Once Ingo is seated---and he'll be in the chair on your left---they'll use a pair of handcuffs with a long chain to attach him to the chair.  There's a steel loop on the edge of the chair to keep him from doing anything cute.  They'll take the handcuffs off his hands and feet next, and then they'll leave the room.

``He'll be alone until just before the interview.  My suggestion is to take a few minutes and soak up his mood.  His mannerisms.  His tics.  The way his eyes move.  This is all Psych 101 bullshit, but it's core content when you're interviewing someone who's combative.  Soaking him up before the game starts will get your mind into the hunt.

``One more thing.  We went through the list of questions that you and Sascha put together, and there's some good stuff in there.  We'll be mixing those questions in, especially toward the beginning.  If you think of any strategies as you listen to his answers, then tell the agent sitting next to you.  He has a direct line to me the whole time.  I'll be in the room to your right.  There will be two other guys in the room across from you.  The man who will be in the room with Ingo is named Jack Turnbull.  He's with the FBI and does some amazing work.  Cool as a cucumber even during these high-profile interrogations.  Any questions?''

``Yeah, two questions if you don't mind.''

``Fire away.''

``First: is anything out of bounds during an interrogation like this?  Does he have legal cause to refuse to answer anything?''

``He can answer anything he likes, but all of it is voluntary.  Normally someone like this would have an attorney present to help him decide which questions to ignore, but so far he's flown solo, and he's done pretty well for himself.  Even during a criminal investigation like this, he has the right to keep his mouth shut.  I'm not sure that he's ever exercised it, though, to tell you the truth.''

``All right,'' Jesse said.  ``And second, where is the men's room?''  He hoped that he didn't look as sheepish as he felt.

``Out the door, first room on the left.  They're always built close to these rooms.  No intermissions, so we have to be fast.''  Forth stood up and smacked Jesse on the shoulder.  ``Thanks for coming in.  I'll catch up with you afterward.''

As Forth left the room, Jesse wondered how he would react to seeing someone who had bugged him.  Identified, isolated, and then \emph{attacked} him, really.  Up to the moment when he was bugged, every problem that Jesse had dealt with in his job had ultimately been a problem for someone else.  He wasn't liable for any damages, and in most cases none of the details of the case affected him personally.  Even when he traveled to Poland to secure the release of some stranded students, it was all business.  The man in the cafe hadn't attacked him, and the students handled their affairs privately after they were back on U.S. soil.  It was a job---one that he loved dearly---but just a job.

This time was different.  He had no vendetta against Ingo.  Like Jesse, he was doing his job.  But when Jesse thought of the big picture, he saw the lines connecting him to Ingo, and Ingo to a group of well-connected men who \emph{did} have a vendetta.  He knew nothing of their motives, but the fact that they were out there, and still protected by a veil of secrecy, was enough to focus Jesse's mind like a laser.

Suddenly the door to the interrogation room opened, and a uniformed officer backpedaled slowly into view.

\chapter{}

Sascha stood with one hand tucked into the pocket of his jeans.  He had just explained the accident to the young officer who responded to the accident.  They were standing well away from Sascha's car to give a fire engine enough space to put down dispersant and oil dry on the road.  So far, there had been no sparks.  Sascha expected an opportunistic stranger to fling a hot cigarette onto the slick of oil and gasoline, but his fellow man had pleasantly surprised him.

The officer finished writing, glanced toward the driver of the Sable that had demolished Sascha's car, and frowned.  He stepped closer to Sascha and spoke in a low voice.  ``Did you get a look at the guy before he ran into you?'' he asked.

Sascha sucked his teeth.  ``I remember that he was drifting from side to side,'' he said finally.  ``Someone blew a horn at him once.  I didn't get a good look inside the car, though.''

``He seems a little over-amped to me.  I've seen a lot of accidents, and it's normal to be nervous or even panicked at first, especially if you're at fault.  But this guy can't sit still.  Wait here, Mr. Greene.  I'll be with you again in a few minutes.''

The officer walked away, leaving Sascha with a few more idle moments.  He pulled out his phone and opened the web browser.  ``If I can't get away myself, I'll find out what's up in the Big Apple,'' he muttered to himself.  He downloaded a pair of articles from the \emph{New York Times}'s website and flipped his phone into landscape mode for easier reading.

Sascha had just finished the first article when the young officer approached him again, this time looking worried.  ``Mr. Green,'' he said, ``can you tell me where you were going before the accident happened?''

``NYPD headquarters,'' Sascha said.  ``I was asked to participate in an criminal investigation.  Figures, right?  Why do you ask?''

The officer raised his left hand and turned the mobile phone in it so that Sascha could see it.  There was an SMS message on the glowing screen.  Sascha read it twice before the meaning sunk in.

\indent INCOMING MESSAGE
	
\indent ABORT. ABORT. DO NOT FOLLOW. WAIT FOR NEW INSTRUCTIONS.
	
``This is \emph{his} phone?'' Jesse asked, his eyes widening.

The officer nodded.  ``We just handcuffed him and put him in the patrol car.  This is out of my control now because we have reason to believe that this man meant to commit a crime today, and somehow it was connected to following your car.  You might want to see the earlier messages, too.''

Sascha took the phone and pressed the button for the previous message in the conversation.


\indent OUTGOING MESSAGE
	
\indent only one guy in car. should i still follow?
	
	
He swallowed hard and pressed the button one more time.


\indent INCOMING MESSAGE
	
\indent Follow license plate F553G9. Blue Chevy Cavalier. Two men.
	
	
That was his car.  Sascha frowned.  \emph{Two men?}  He pressed the button a third time.


\indent OUTGOING MESSAGE
	
\indent im at the parking structure. what next?
	

Sascha cringed.  He had been at the office all day, getting his normal duties out of the way to make space in his schedule for the interrogation.  This guy had followed him all the way from CUNY.

\emph{God damn it}, Sascha thought.  \emph{Why didn't I walk?  If I had cut out one coffee break, then I would have had enough time.}

The phone indicated that there were no more messages, and he handed it back to the officer.  ``Where should I go, officer?'' Sascha asked.  ``They'll need to ask me questions, and I think that the rabbit hole just got deeper for me.''  He sighed.   ``It may be connected to the case that I was supposed to help investigate today.''

``The same place you were headed,'' the officer said.  ``The feds and one of our detectives will meet you at headquarters.  I'll give you a lift, unless you feel like walking.''

Sascha caught the sarcasm.  ``Thanks,'' he said with a lopsided grin.  ``I'd better not take any more chances today.  What about my car?''

``We'll take care of the towing.  You can settle up and retrieve it tomorrow.''

They walked toward the police cruiser.  The officer gestured toward the front passenger seat, and Sascha took a look at the handcuffed man in the backseat as he opened the front door.  The look on his face was still one of utter terror.

\chapter{}

\emph{Here we go}, Jesse thought.  After the first officer, a short man wearing dark pants and a tan polo shirt shuffled through the door.  He took short, awkward steps to avoid tripping over the handcuffs around his ankles.  This demanded most of his attention, giving Jesse a chance to study him.  Ingo wore glasses with large, rounded lenses.  His wiry hair was bushier than Jesse remembered, but this was the first time that he had seen Ingo without his trademark top hat.  His ears were small in proportion to his round, plump face, and he had long slender fingers.

Once Ingo reached his chair, he obediently sat down.  There was no point for him to raise a fuss here.  He held out his hands to the officer who entered first.  The second officer attached a set of handcuffs with an elongated chain to Ingo's right leg and the metal loop on the chair.  The first officer then pulled a set of keys from his pocket and removed the handcuffs from Ingo's feet first, followed by the set on his hands.  Their ritual completed, the officers left the room.  Jesse heard the door's heavy lock click through a speaker that was built into the box on the wall.  The speaker must have been activated by someone in another room.

A moment later, another man dressed in a black shirt and black jeans---detective wardrobe---entered the room and sat down at the other table.  He greeted Jesse.  ``I'm Detective Walsh.  Thanks for being here.  Detective Forth told me that you've been very helpful already.''

``Pleased to meet you,'' Jesse said.  ``I don't know about helpful, but I'm hoping that we can solve this puzzle together.  My ability to sleep at night depends on it.''

Walsh laughed and leaned back in his seat.  ``It's great that you've kept a sense of humor.  That's key, if you ask me.  Even in a sticky spot like this when someone's clearly got your number, you'll be all right if you can mind your head and remember that the justice system works well most of the time.  I think you'll see some good work here today.''

``I already have.  Now we need to get some dirt on this guy.''

The heavy lock in the room clicked again.  ``Here we go,'' Walsh said.  He poured two glasses of water, setting one on Jesse's table.

Ingo turned his head to look at the door.  The door opened after a few seconds, and a tall man in a white dress shirt walked across the threshold.  Something about him looked distinguished.  Jack Turnbull's hair was gray, but in several places there were light brown hairs that betrayed his true color.  He was surprisingly thin.  Ingo was not a large man either, but Jesse suspected that he could win a fight with his interrogator.

Turnbull smiled at Ingo.  As he came around in front of him, Turnbull stuck out his hand.  ``Nice to meet you, Mr. Sternoza.  My name is Jack.  Jack Turnbull.''

Ingo shook Turnbull's hand.  Then he spoke for the first time: ``You're taller than I expected.''

Jesse and Detective Walsh chuckled.  Ingo had a raspy voice that reminded Jesse of the voices of Italian mobsters in American movies.  His accent was sharply articulated with an unmistakable eastern European flavor.  Jesse sometimes struggled to understand the type of accent that Ingo had, but he was confident that this time would be different.  There was enough New York English in the mix that Ingo seemed less a foreign student than a child of Hungarian immigrants.

Turnbull laughed, taking his seat.  ``Well, now we're the same height,'' he said.  ``I'm all leg.''

``Are you a runner?'' Ingo asked, tilting his head to the right.

``No, not a runner.''

``Basketball?''

``In high school.  I still have a hoop at home.''

``I like the Celtics.''  Ingo let this statement hang in the air for a moment.  ``Yes,'' he went on, ``it's difficult to live in New York and be a fan of a team from Boston.  I don't put flags on my window.''

``Even during the playoffs?''

``Well, during the game I might make an exception.''

``Did you ever play basketball, Ingo?''

``Only in the street.  I saw Michael Jordan in Budapest once, and he inspired me to come to America.  He was speaking about humanitarian work in Africa at a library.  Have you seen him speak?''

``No.  Well---sorry.  I've seen him speak on television.  After basketball games.''

``Ah.  He says nothing useful during those speeches.  I think athletes go to school to learn an empty language for press conferences.  Lots of words, nice sounding, but nothing is said.''

``You're right.  I think that they waste a lot of time doing those interviews.  Fans want to hear their personal thoughts and feelings, but all that they can say is that they need to work harder and move the ball better.''

``Yes, exactly.  I hear more interesting things from the man in the next cell.  He tells me about his guitars and the time when he put a stun grenade into a swimming pool.''

Turnbull laughed again and shifted in his seat.  ``All right.  And what happened afterward?''

Jesse admired the give and take in the conversation so far.  Ingo seemed relaxed, and Turnbull seemed to be encouraging him to talk.  He wasn't ostensibly trying to steer the conversation and had let Ingo control the tempo.  Turnbull's calm demeanor and easy talking skills were something to behold.  Jesse knew that he would be nervous and awkward if he were on the other side of the mirror, but Turnbull made it look so effortless that someone passing by could be forgiven for thinking that it was a friendly interview for a magazine piece.

So far, Jesse was drawing a blank.  He was trying to study Ingo and get his ears fully tuned to his accent.  The rasp and sharp articulations in his voice were no problem on their own, but together they created an exotic mix that required a listener's brain to be steeped for a few minutes before it felt natural.  The other men in the side rooms were probably doing the same thing.  The first few minutes of any interview or interrogation were key to setting the mood, and this one, despite being a criminal interrogation, was designed to be open and free-flowing.  Ingo was a special case that called for special tactics.  The detectives had made that fact painfully clear.

\emph{But what does it really mean?} Jesse wondered to himself.  \emph{If Ingo is so smart, then wouldn't he know that they're trying to play him into giving himself away?  Wouldn't he see through that?}

Ingo grinned.  His teeth were small but very neat and white.  There were no dimples on his cheeks.  ``The pool filled with a dark green substance.  There was some movement in the water after the explosion, but not as much as he wanted.  The experiment was a failure and he had to clean the pool.  His dog wouldn't go near it.  But it gives me a good story to tell.''

``It certainly does,'' Turnbull said.  He smiled, pausing to allow Ingo to continue if he wished

``So I was thinking,'' Ingo began.  He looked slowly around the room as he spoke, pausing as he looked at each of the mirrors.  ``How long will you keep holding me?  The food at the prison is bad, and the men are dull.  Are you going to ask me real questions this time so that I can go back to my life?''

\emph{Interesting}, Jesse thought.  \emph{Doesn't he realize that they often learn useful things from the non-real questions in the initial interrogations?}

``Soon enough, Ingo,'' Turnbull said.  ``We won't keep you any longer than we need to.  Are you ready to start with the questions?''

``Yes, please do,'' Ingo said.  He had finished surveying the room and now looked squarely at Turnbull.

``All right.  You mentioned before that you like history and politics.  Is that right?''

``Of course.''

``Okay.  Is there a particular political theme, or maybe a time in history, that interests you the most?''

``Really, Mr. Turnbull, this is not an interesting question.  But I do love politics.  It has been most interesting to follow the elections here and listen to your politicians play games at your expense.  They always say the same things---same as athletes after a game---and you listen to them, expecting a revelation.

``In Hungary, politicians do not become wealthy when they are in public office.  They cannot.  This is true in many European countries.  One does not need to be rich to become elected, and one cannot become rich after election.  Business does not corrupt politics with too much money.  If someone is elected, he wants to serve Hungary.  One of your writers said that nobody who wants to be president should become president.''

``Douglas Adams,'' Turnbull said with a shadow of a grin on his face.  Jesse smiled more broadly.

``Bless that man,'' Walsh said quietly, glancing at Jesse.

``Yes, he is the one,'' Ingo said.  ``He is right, but only in systems like yours.  If you remove the money, then you solve many of your problems.  In Europe we have known this for some time.''

``Is the same true in Poland?'' Turnbull asked.

\emph{Subtle}, Jesse thought, \emph{but a little sharp.  Come on, take the bait\dots}

Ingo frowned and glanced at the ceiling.  ``Poland is an interesting case.  I know many politicians from Poland, and they play two games.  There is little money in politics for them, but their offices are very well connected.  The parties have been involved in many businesses since two centuries ago, and each new politician who is elected is asked to grow the businesses and create new wealth for them.  Many of them look to other countries for this.  Sometimes it is not proper.''

Walsh began to write on his notepad.  Jesse sensed that the conversation was going well.  He wanted to generate some ideas for the detectives, but he was content simply to go along for the ride.

``Not proper?'' Turnbull asked.

``Yes, they take bribes and favors from foreign businessmen in exchange for favorable treatment for companies.  It begins to look like your country.  The relationship between the parties and businesses in Poland was good when the leaders knew each other.  Everything was local.  The world has more people now, and the men only know email addresses.  Voices.  They never get to shake hands!  What is trust?  Where is trust?  They don't shake hands, and they don't feel remorse when they, ah, \emph{burn} each other.  Stab in the back.  They make deals with drug lords in South America.  They sell weapons in Russia and south Asia.  No remorse.''

``Do you know of any specific examples of this?''

``Oh, yes, there are many.  The Jasna family is one of the most powerful in Polish politics.  They control one of the parties, and it has all of the connections that I mentioned.  They are dangerous.  The party is popular because they are careful to isolate their business ventures from their voters.  The international connections make this easy.  They invest money through legal, hidden channels to the third world, and they receive their dividends indirectly.  This is very hard to trace.  Their voters are happy because the party has the resources and leverage to make things happen, and the party is happy because it is a fat cat that makes its members wealthy.''

Jesse sat frozen in place.  He had been ready to take notes on this part of the conversation, but his brain had gone numb.  Poland.  Jasna family.  International connections.  It was not a coincidence that he had heard those concepts in the same breath before.  He had met with Marcel Jasna in Warsaw when he helped to bail out the students who bugged a political office.  Jesse hadn't known the relationship between Marcel and the rest of the party, but he suspected that it was a close one.  He had clearly underestimated how close.

Turning his head toward Walsh, Jesse finally managed to say, ``I think we need to stay on this topic.''

Walsh raised his eyebrows.  ``Oh?  You've heard of the guys he's talking about?''

``I've met one of them, and the guy might have a reason to track me down,'' Jesse said.  He swallowed uneasily.  ``He might be carrying a grudge.''  \emph{He might try to have me killed.}

``Let's relay that to the boss.  He'll be glad to know that we're getting somewhere.''  

Walsh touched a button on the wall-mounted box and spoke into the air: ``Boss, we need to push on this.  Winter says he's heard of the Jasnas and met one of them.  Might have a motive.''

The disembodied but unmistakable voice of Detective Forth came back almost immediately.  ``Ten-four.  Is there a person of interest?''

``Marcel,'' Jesse said immediately, and Walsh relayed the message.

The box was quiet.  ``Will we need to wait until Turnbull takes a break?'' Jesse asked, a little worried that they would lose the thread before the message made its way into the interrogation room.

``Nope,'' Walsh said.  ``Turnbull has a tiny earpiece that Ingo can't see.  I'm sure he already knows the scoop.''

Ingo was still talking.  ``They have all become obsessed with their story.  The family has done nothing for two generations, but they act like kings.  It makes me wish for a new party, but it is not my country.  They are entrenched, and it will be very difficult for reform to come to Poland.''

``I'm wondering, Ingo,'' Turnbull said.  ``Do you know anything about Marcel Jasna?''

Ingo looked at Turnbull for a moment, clearly trying to work out how Turnbull was interested in a specific foreign political figure.  ``Yes, he is very interested in you,'' he said, pausing meaningfully.  ``In all of you.''  He looked around the room at each of the black mirrors.

The room went quiet for a moment, and Jesse was suddenly uncomfortable.  It was unlikely that Ingo knew that he was nearby.  Forth had promised to keep Jesse's presence secret, both to ensure the integrity of the interrogation and to keep Jesse from being even more tangled up in the case.  But the fact that Ingo was playing to the camera also suggested that he was aware of the circumstances and might not be interested in breezily discussing his conquests and personal political flavor for very long.  For a while, though, he had been on a roll.  If he was aware of what the detectives wanted to learn from him, then he didn't let it show.  Jesse hoped that Turnbull would be able to keep Ingo off balance.

Turnbull tilted his head and frowned.  He took a deep breath.  ``What interest does a Polish political figure have in a few New York City police detectives?''

``Oh, it is a tired story,'' Ingo said.  His eyes were downcast for a moment.  ``You know what happened.''

``I'm having trouble remembering the details.  Was it an immigration dispute?''

``You could say that.  The Jasnas have made many friends.  They also have enemies.  Most of them are former business partners.  People become, umm, disenchanted with the way that the family operates.  They do not like their choices.  When they leave, the family makes it clear that they must not tell anyone what they have learned.  They do not make a threat, but it is understood that the family will protect itself and its secrets.  Most people who do not respect their wishes find that their problems in life become larger.  Some have become suddently ill and died after they had been very healthy.  The family is dangerous.  Some lucky people leave the country and avoid the reach of the family, but they hold grudges.  Sometimes their children are not able to suppress their feelings.  Is this familiar to you now?''

``It's coming back to me, yes.  Please go on.''

``There was a story in the \emph{New York Times} some time ago.  The author spent a lot of time talking about a City University student who had gone to Poland and tried to spy on some city council members in Warsaw.  They were caught and taken to jail.  They should have been prosecuted.  Under Polish law, they should have stayed in jail for many years.  The author of the article had nothing to say about how the spies escaped their crimes.

``Marcel is my friend.  If you are trying to trap him, then you will hear nothing helpful from me.  Yes, he is a member of a powerful family that is dirty in business and politics.  But he has had no part in that.  If there is a member of the family that is true in character, it is Marcel.  After those students were allowed to leave Poland, I spoke to him.  Marcel was very upset.  A man from the City University came to him in Warsaw and begged to let the students go free.  He wanted them to face no penalty.  To pay no damages.  To answer for \emph{nothing}.''

Ingo took a deep breath and straightened his posture.  He had worked himself into a fury and took a moment to regain his composure.

``This man from the university was a fraud.  He had no interest in justice.  He was only interested in making a victory for America and keeping the spy training program at CUNY.  It is very unpopular, you know.  They teach students how to go into other countries and interfere in their politics, and when the students run into trouble, they send someone to grovel and plead for their release.  It is sickening.''

\emph{Keep him going}, Jesse thought.  \emph{Spur him just a bit more.}  Jesse's blood was boiling, but he had the presence of mind to take the abuse and remember why he was there.  They needed to know more about Marcel.  Ingo hadn't given them anything but spin.

``But the students were allowed to leave Warsaw,'' Turnbull said.  ``Why was that?''

Ingo exhaled forcefully and shook his head.  ``It makes me sick.''

``Never seen him worked up like this,'' Walsh whispered to Jesse.  ``The next few minutes should be good.''

``He threatened them,'' Ingo spat.  ``He threatened them.  He said that he had a recording of a conversation from more than twenty years back.  The family does not make mistakes.  But one of them, Marcel's father, said something during a walk one day that was overheard by an old woman.  The woman was from a rival political family, a rival party too, and she made some demands.  No one else knows what he said.  But this CUNY man, he claimed that he had a recording.  A recording of the conversation with the old woman.  If Marcel did not let the students go free, then the man was going to publish the recording.

``I have known Marcel for many years.  He is a good friend.  He cares for his family.  He does not know what happened between his father and the old woman, but he said that it could ruin his family.  He knows this because his father spent many years negotiating to keep the conversation a secret.  When the woman passed away several years ago, Marcel thought that the problem had gone with her.  But now he is not so sure.  It has caused him much grief.  His father is very troubled by these events.  Do you see?  Because of your country's interference, his family cannot have peace.''

``I'm sorry to hear that.  Is he trying to set things right with the woman's family?  Could he find peace that way instead?''

``No, it is not so simple.  When he talks to her family, they snub him.  They have become bitter from the many disputes.  There will be many years of difficulty ahead for Marcel, and these student spies have ruined any chance at peace.  Now he must waste valuable time trying to learn who else knows about his father's conversation and how a record might have been made.  If a record exists, then he will find it.  I have nothing else to say about this.''

Jesse wrote furiously on his notepad, trying to put down every lie and detail that Ingo had just shared.  Despite the slanderous speech that he had just endured, Jesse felt something like relief.  He had a much clearer picture of the plot against him, but everything was still a hunch.  Marcel wanted to destroy any copies of his father's dirty laundry, and he was willing to break the law to do so.  The cafe chat in Warsaw had left a bitter taste in his mouth, and it was time to take care of business.

Just as Turnbull crossed a few lines off on his notepad and began to ask the next question, the box on the wall clicked.  It was Forth's voice again.  ``Winter?  There's someone in room 63 waiting for you.  It's urgent.  We'll get you caught up on Ingo later on.''

Jesse was puzzled, but he stood up and collected his notepad.  Taking a last drink of water, he gave Walsh a thumbs-up and saw himself out.

Room 63 was a tiny waiting room near the lobby.  Jesse wandered back through the labyrinth of offices and interrogation rooms and eventually found himself in the main hallway.  The door to 63 was closed, and there were no windows.  He knocked, and Sascha opened the door.

Jesse had known Sascha for years, but he had never seen the face that was staring back at him.  Sascha looked \emph{amped}.  He was ready to fight, and Jesse didn't know whether to turn on his heel and flee or be comforted that his partner in crime had made it out of the jungle alive.  He sensed that there was a good story in store for him if he stayed, and at any rate his feet refused to move.

Sascha grabbed Jesse by the shoulder and hauled him into the tiny room, heaving the door closed behind him.

Room 63 was little more than a broom closet.  It was about six feet deep and no more than 4 feet wide.  A tiny table sat against the far wall, and two chairs sat in the middle distance.  Jesse took the chair farthest from the door.  Sascha stayed on his feet.

``Are you all right?'' Jesse asked.  ``You look like you've had a week's worth of caffeine.''

Sascha didn't take the joke.  ``The next hit was on both of us,'' he said.  ``They thought that we were going to be in the same car.''

``Who thought that?''

``Whoever has been after you.  The jackass who rear-ended me today was texting right before the accident.''

``Big surprise there.  What's that got to do with whoever's stalking me?''

Sascha pulled a few sheets of folded paper from his pocket and handed them to Jesse.  ``Check those out.  The juicy bits are highlighted.  It's a copy of the guy's text messages for the past three days.  He's been in contact with someone who wanted to follow us.  Remember what I told you about the posters around campus that advertise flexible hours and great pay for students?  He was one of their catches.''

Jesse tried to read the messages and found that he couldn't remember the beginning of each message once he reached the end.  He had been in the zone during the interrogation, but the revelations from Ingo and being asked to leave the room to handle an emergency was enough to kill his concentration.  He was familiar with the feeling.  On the rare occasion when he was able to enter the mental state known as ``flow'' during his work day, he was unbelievably productive.  But the state was fragile.  If a thought crept into his mind that he was in flow, or if a co-worker came into the room and talked to him, then it was lost.  He had read that life-long meditators had a markedly improved ability to enter flow, but Jesse had never been able to get into the habit.  At the moment, he was glad that Sascha was standing next to him, visibly anxious to share his theory about the noose that seemed to be tightening around them both.

``So they're after both of us now,'' he said, struggling to get his mind back into gear.  ``What does it all mean?''

Sascha cleared his throat and pounced.  ``It's still all about \emph{you}, but they know that we're acquaintances.  They were expecting us to be driving to the NYPD together.  I don't know how they knew that.  It makes me wonder if Ingo has a contact in prison that he's using to funnel information.  The prison crew knows that he's a high-value prisoner, and they've been monitoring his mail and phone conversations.  There's been nothing useful.  The feds have even been watching for familiar codewords and steganographic stuff, but they've come up empty.  He's either really good, or he's using a channel that we're not privy to.  I'm guessing the latter.  He still has quiet moments when he can chat with the other prisoners, and Ingo has been around the New York crime scene for long enough to make some connections.  If he found one of them in there and has been able to send messages, then we're at a serious disadvantage.  As soon as he gets back to his cell, whoever is after us could know exactly what we've been asking during the interrogations.''

``Do you think that they could keep him in solitary until we can learn something about what we're up against here?''

``It's going to be a tough sell with Forth.  He knows that Ingo is connected, and that puts us in a nice catch-22.  If we cut him off, he'll shut down and feed us bogus information---if he talks at all.  His messages to the outside will suddenly stop, and whoever he's talking to will know that something is up.  That's bad for you and me.  On the other hand, if we let it ride, the guys on the outside will know that we're mining Ingo for gossip.  He'd be a fool not to recognize that, especially with the questions that they were planning to ask today.  That's bad for all the good guys, and it's pure gold to the outsiders.''

Sascha withdrew and took a few steps around the room, tapping a finger in his chin and drumming his other hand on his jeans.

``I think I know what they want,'' Jesse said.  His voice was low, almost apologetic.

Sascha stopped and turned toward him.  ``Say again?''

``Today in the interrogation.  Just a few minutes ago.  Ingo let it spill that the guys from Warsaw---you know, from the party with a capital `P'---they want the recording.  Remember?''

Sascha's eyes were distant for a moment.  Jesse was ready to relive the interrogation again in order to put them both on the same page, but then Sascha's face cleared.  ``The old woman.  In the street.  Nineteen eighties.''

Jesse nodded.  ``It's about her,'' he said.  ``When I flew to Warsaw, I told the guy in the cafe that a recording of that conversation still existed.''  He sighed.  ``It's true.  And that guy---his name is Marcel---is a member of the family.''

Sascha looked at him levelly.  ``So it's complicated.  \emph{Real} complicated.''

``Yeah, but I think there's a picture starting to emerge.  The rabbit hole goes deeper than we know; I'd bet money on that.  But we know of a group of people who feel that they've been wronged.  They feel threatened.  They have a focal point for that threat---that would be me---and they're starting to dissect my personal and professional web of contacts.  I don't think that there's any risk to you or anyone else unless you get in their way.  They've been trying to figure me out.  To find out where I might hide something that's very important to them.  People are really bad at hiding things, and there's no reason for them to think that I'm any different.''

``Too bad for them.  But does this thing really exist?''

Jesse had been looking at the ground.  He glanced up at Sascha and nodded almost imperceptibly.  Sascha seemed to take the hint---Jesse was trying to give him deniability, just in case someone happened to be watching or listening.

Sascha chewed on a fingernail for a moment and paced around the small space again.  ``There's something else that I haven't told you yet.''

There was a knock at the door, and both men jumped.  ``Occupied!'' Sascha said, turning halfway around toward the door.

``Sorry,'' a woman's voice said.  They heard the diminishing sound of heeled shoes clicking on the tile floor in the hallway.

``What is this room, anyway?'' Jesse asked.

``It's a room for filling out job applications,'' Sascha said with an odd look in his face.  ``Who would want to work here after getting stuffed into a cigar box like this?''

``Anyway,'' he went on, ``there's another wrinkle here, and it's the reason that I asked for a room like this.  When the detectives ran the query into the text message records of this guy, they found something odd.  We didn't see it on the guy's phone because it was a message that he had deleted.  The page that I gave you was printed right off of the phone's SIM card.  Someone sent a message to him just after he called to apply for the job listed on the flyer, and it showed up on the records from the phone company.''

Sascha took out a yellow sheet of paper with perforated strips on the edges.  It had been printed on an old dot-matrix printer and was difficult to read, but Jesse was able to make out the body of the message:

\indent Need a quiet visit to an apt. 40 1/2 N Piata. Tues late.
	
\emph{North Piata}.  ``That---'' Jesse began.

Sascha nodded and held up a hand.  ``It's worse than we thought.  We need to get you out of here.  Sooner rather than later.''

``But the interrogation isn't done yet.  Shouldn't we go back and talk to Forth?  If we don't know how it turns out, then we'll be in the dark once we're out of the building.''

Jesse dropped his voice to a whisper.  ``What if I can't go home tonight?'' he said.  ``They have---''  He jerked a finger at the page that he held in his hand.  \emph{My address.}  ``There's nowhere safer than this building, and you're talking about leaving without telling anyone?''

``It's the best thing.  If Ingo is channeling information, then it's best if you're gone and nobody knows how to pick up your trail.  We'll call them tonight and get the four-one-one.  Right now, I need you to tuck that sheet in your waistband and take a few breaths.  Get your head back in the game.  I know you have a routine for that.''

Jesse considered him for a long moment.  It was starting to sink in that his evening plans were not going to come to fruition.  After the interrogation, he was going to visit a restaurant where he had eaten when he first moved to the city.  It was a quiet place that served bar food and was in a neighborhood off the beaten path that none of his friends or colleagues had heard of.  That would have to wait.  Sascha would have a plan, and Jesse knew that his only play was to trust the former marine's prodigious judgment.  \emph{He's better half asleep than I've ever been}.

Finally, Jesse nodded his acceptance.  ``Where are we---'' he began, but Sascha cut him off again.

``Don't.  Just come with me.  The circus in the lobby is still going on.''

Jesse tore the scrawled-upon sheets out of his notebook and tucked them safely into his waistband, leaving the notebook on the table.  ``Crib sheets for the next contestant,'' he said with a grin.

Sascha gave a single nod and opened the door.  There was an unusual amount of foot traffic zipping past the open door, and they could hear what sounded like rushing water from beyond the door to the lobby.

For Jesse, the two minutes that followed were terrifying.  He had never liked large crowds, and the throng in the lobby was worse than any English football game that he had seen.  Reporters were yelling into their microphones or barking orders at their camera crews.  The accused council member's motorcade had pulled up outside the building, making an exit impossible.  Jesse stayed close to Sascha, narrowly avoiding being tripped by stray wires and the bodies that were rushing across the crowded room.

Sascha had realized that there was no hope of making an exit through the front doors and stood on his tip-toes, searching for another door.  Suddenly he cut to his right, ducking under a large light and interrupting a live remote feed from the local NBC affiliate.  The reporter looked scandalized and tried to grab Jesse's sleeve as he rushed by, but he shook himself free and was back on Sascha's heels before the reporter had regained his composure.

\emph{You've got to be kidding me}, Jesse thought.  Sascha had seen an exit.  His instincts weren't bound by the same shackles as Jesse's, and he had spied a large parcel delivery door in the wall of the lobby that was visibly unlocked.  The door was tucked behind an unmanned reception desk but was plainly visible.  It was normally used for receiving large freight shipments that would not fit through the reinforced front door of the lobby, but it was more than large enough to allow a man to pass through.  The large padlock that normally kept opportunistic intruders out of the NYPD office was hanging open on a hook next to the door.

As they reached the door, Sascha glanced toward the front door.  At that moment, the door of a limousine opened on the street, and the noise level more than doubled.  Flashbulbs exploded across the chamber, leaving Jesse and Sascha dazed.  Sascha pulled the door up and its tracks, grabbed Jesse's arm, and hauled him through the door.  As he emerged on the other side, the door slid closed with a metallic bang.  A group of puzzled onlookers stared at them.  A young Asian man began to fumble with his camera, and Sascha gave Jesse another firm tug to get his feet moving.  They were out of sight before the man could take the photo.

As they walked past the edge of the window into the lobby, Jesse saw Bradley Johnson, Forth's assistant, emerge from the hallway holding the notebook that Jesse had left behind.  Johnson looked around, clearly uninterested in the hubbub that filled the lobby.  Jesse thought that he saw panic in his eyes as the window passed out of view and Sascha's pace quickened.

``All right,'' Jesse said quietly, ``can we talk about where we're going now?''

``You know, you're not very good at this game,'' Sascha said.  ``At the moment, we need a car.  I'll figure out the rest.''

Jesse looked around, trying to orient himself on the mental map of the city that he had build over the past ten years.  There was nothing but restaurants and video rental shops in the blocks around them, but Jesse knew that he had a bad sample.  Soaked in anxiety, he could only remember the places that he visited regularly.

``Any rental places around here?'' Sascha asked without turning his head.  ``Cheap ones with junk cars are what we want.''

His memory thawed, and Jesse suddenly recalled a pair of shops that were about ten minutes' walk.  They were owned by an Italian family that prided themselves on keeping a fleet of older cars on the road.  Older didn't mean classic, and it didn't mean pretty, but the family found a rental market that nobody wanted to recognize---people who wanted to stay under the radar.

Eleven minutes later, they were standing in front of a grubby counter and waiting as an older man with a Cicilian accent poked the keys on a computer keyboard.  Sascha had negotiated a cash payment, and a generous tip was enough to secure a free ride out to the rental lot where the cars were kept.  It was well outside of the city and would give them a few minutes to sort out their thoughts.

\chapter{}

The old man was a stud.  He gripped the steering wheel of a non-descript green sedan that had more horsepower than anything Sascha or Jesse had driven.  Despite his age and the slow, awkward way that he walked on the way toward the car, there was no doubt that his mind was sharp.  He knew the city's alleyways, lights, and traffic patterns.  It was no more than fifteen minutes before they were on a sparsely-travelled, unlit road.  The man kept their speed within six miles per hour of the posted speed limit, reflecting his first-hand knowledge of the habits of the local police.  They would get to the destination as quickly as was safe and prudent, and no faster.

Sascha could trust no one.  His gut had never failed him, even in the moments on foreign battlefields when he was sure that the young men pointing their kalashnikov rifles at him and shaking with fury were going to make an example of the poor American who had fallen into their lap.  He had always been an instinctive man, not putting much stock in raw intellect.  The rational mind just didn't have the staying power that one needed to survive in the jungle.  It wasn't syllogisms or philosophical debate that kept Sascha's distant ancestors, doubtless a line of stupendous badasses, from getting eaten by a pack of lions.

He had begun to count the cards that were on the table, putting together a map of the political landscape that had become clear in the past twenty-four hours.  The guys after Jesse were connected to Europe.  Eastern Europe.  It was the only group that had a vendetta, and the only one that would care enough to make a play on Jesse's person as well as his apartment.  The apartment was clean, presumably a place that was arranged by the feds or the NYPD.  If someone else knew about it, then two things were possible.  If the place had been used to house endangered people before, then knowledge of its location could have leaked through a dozen different channels.  Failing that, then there was someone in the investigation who had dropped breadcrumbs.  Whether it was done intentionally or not was an academic question at this point.  Not being able to trust the authorities was the card that weighed most heavily on Sascha, and it was the reason that he had been so keen on moving Jesse away from any known places or people.  He needed some time to think, and he knew that Jesse was reeling from the stress of the entire situation.

Sascha had been absently turning his mobile phone over in his coat pocket.  He remembered suddenly that it would be reporting his position and quickly pulled the case off.  He nudged Jesse and made a show of removing the battery.  He gave Jesse a significant look and glanced down at his pocket.  Jesse took the hint and did the same.

When they finally reached the parking lot, the old man handed Sascha a pair of keys on a ring with a dirty tag and wished them well.  ``If you see any white smoke, give the oil a check,'' he said.  ``You might need a few drops.''  He revved the engine and drove off, leaving Sascha and Jesse alone in a quiet parking lot in the middle of nowhere.

Sascha looked up at the sky.  The Milky Way was barely visible overhead.  The glow of the city was clearly visible to the east even though they were more than forty miles away.  ``Hungry?'' he asked.

``Yeah, I could use some comfort food.  Where are we, anyway?''

``No clue.  That's an advantage right now.  We just need to hope that nobody else knows where we are.''

The adrenaline had finally filtered out of Jesse's bloodstream, and his hunger had returned in full force.  He was shaking slightly as they spoke, making him look nervous.  He took a deep breath of the cool evening air.  The thought that his world was shrinking had begun to sink in.  The part of the world that he could trust, anyway.  He couldn't go to his beloved apartment, nor could he go to his new place.  The university had proven unsafe.  And now the NYPD was on the list of places that should be avoided.  Pulling the battery from his phone had been a breaking point.  He needed some good news, and he had a feeling that it would be a while before Sascha could deliver it.  It could be a very long night.


The road lines on the interstate rolled by in time with Jesse's heartbeats.  He was still too wired to sleep, and he was waiting for Sascha to give him some news about where they were going.  They had turned at least a dozen times since leaving the parking lot, snaking their way along dark roads that all looked the same.

``I need to know something about that recording,'' Sascha said as he pulled into the parking lot of a small diner whose pink neon sign had only one illuminated letter.  He killed the engine and set the parking brake.

``You don't need to tell me where it is.  Or what it is.  I just need to know whether it exists and whether someone could find it using information in your apartment.''

Jesse looked out the windshield at the diner.  A middle-aged couple was having dinner and enjoying a lively conversation.  He ran through the items in his temporary apartment, trying to recall what he had left there in his suitcase and in the boxes of essential items that the police had brought from their storage site.  He convinced himself that there were no copies of the recording in the apartment before saying anything.

``It's real,'' Jesse said finally.  ``The recording exists, and I've listened to it.  The original was created on a handheld cassette recorder, so the quality is terrible.  I mean really awful.  But you can make out what's being said.''

``So it's a tape that they're looking for?'' Sascha asked.  ``A plastic cassette tape?''

``Oh, no.  No way.  The original might still exist, but it's been digitized and cleaned up.  There's no specific physical object that these people are after, but they might not know that.  I don't have a copy in my possession.  There isn't a copy on any of the computers or thumb drives in my apartment.''

``At the office, then.  That's where it is?''

Jesse looked at Sascha for a moment and then looked back at the diner.

Sascha looked incredulous.  ``Jesse, you know that's not safe,'' he said.  ``You were bugged while you were at work for most of a day, and we know that they've had access to the computer system.  If there's a digital copy somewhere, then they have a great chance of finding it.''

``Don't worry about that,'' Jesse said.  ``I've thought about the basic security threats, and this is what I came up with.  I know it doesn't seem smart, but it's in the safest possible place given the circumstances.''

``All right.  Let's get some grub.''

\chapter{}

The diner was cozy and bright, and Jesse was able to wake up.  The miles on the road and the lack of adrenaline in his system had made him tired.  He knew that a hearty dinner would only exacerbate his fatigue, but holding off much longer would make him groggy, irritable, or both.

He followed Sascha to a booth in the far corner of the diner.  A magazine rack sat against the far wall, and through the window the rental car was plainly visible.  There were no patrons in the two booths closest to them.  The diner appeared to be staffed by a bartender and a waitress who were discussing something behind the bar.

After they took their seats and shook off the cold, Jesse quietly thumped the table with a flat hand.  ``So,'' he said.  ``What's the plan now?''

Sascha pulled a menu from the end of the table and opened it.  ``I think we should go back,'' he said.

Jesse raised an eyebrow.  ``You're kidding,'' he said.  ``The people who've been stalking us are going to be dropping by my apartment tonight, and you want to go back?  Shouldn't we let the police handle that?''

``Have you put the pieces together yet?  Do you think that we can trust them?''

``Yeah, I've put the pieces together.  Only the police knew about the new apartment, but somehow the address wound up on that guy's phone.  That means that either someone stole the information or it was leaked.  Either way, we need to be careful.  Do you think that we could call in an anonymous tip?''

``Oh, I plan on it.  But we need to know exactly when they're on site.  Even the NYPD won't be happy about hanging out all night \emph{just in case} someone happens to make a surprise visit.''

``Sure, but it's an apartment of someone who's part of a criminal investigation.  Wouldn't that be enough?''

``It depends on how the address leaked.  If we give away where we are, we don't know who will handle that information.  I'm not ready to make ourselves available yet.''

``Good point.  So what's the strategy?''

``I think that we should go to North Piata, park a couple of blocks from your apartment, and lie low.  If they show up, then we make the call.  The police will respond the way that they always do, and we'll bag the stalkers.  Or, at the very least, we'll bag another set of their employees.  Eventually they'll get greedy and do a job themselves, and we need to be ready for it.  These failures---Ingo, the girl who mugged you, the guy who rear-ended me---must be wearing on them.''

Going back to the city, much less his tiny, bare apartment, was the last thing that Jesse wanted to do, but he knew that Sascha's intuition had an internal logic.  If they stayed on the run, then Jesse's apartment would be ransacked and he would need to file a police report.  Whoever leaked the apartment's address would be working on the next step and helping the guys who were stalking him to tighten the noose even further.  If they made a more aggressive move, though, then they might be able to use the break-in as leverage.  Someone within the police department would be keen to paint the incident as a random home break-in and dismiss any other call for investigation.  Connecting the dots between the text message containing Jesse's apartment address, the break-in, and the ongoing investigation would make a compelling argument, and it was an argument that they needed to make.

``Come on,'' Sascha said, seeing a shadow of doubt in Jesse's face.  ``Where's your heart?  Aren't you the one who's always delivering the grand visions of the future of investigative journalism?  You've been behind the curtain since college, but I know that your fire is still going.''

Jesse grinned and shook his head.  ``You're right.  Of course you're right.  You're also a jerk, you know that?''

``Yeah, and a hungry jerk at that.  Time for some food.''

The waitress arrived, bringing a wave of fresh air with her.  ``What can I get you, gentlemen?''


When they climbed back into the car, Jesse's mind was much refreshed.  He imagined the view of his apartment from several blocks away.  The streets were dim at night, giving them an advantage.  It would be easy to find a place that was well away from the glare of the street lights.  The door to the apartment was directly under a light, and there was an additional safety light on the front of the building.  While it was unusual for a New York apartment, there was no rear exit or fire escape.

``How long?'' Jesse asked, checking his watch.  It was nine o'clock, and the message had requested a late visit.

``Forty minutes,'' Sascha said.  ``I took a very circuitous route on the way out here, but we can get back much more efficiently.  It could be a late night, but we'll want to be there early.''  He downshifted and accelerated as they reached a long stretch of straight road.  ``Have you thought of a place where we can see without being seen?''

``I know just the spot.  And there's a restaurant with great take-away about 100 yards away.''

``Perfect.''

\chapter{}

The old car rolled along Piata Street, its headlights casting odd shadows on the endless buildings and bus shelters.  Jesse hadn't approached his apartment from this direction before, but the street and building numbers were approaching the range that he was used to seeing.  It was unusually quiet on the street.  Normally, couples and shop owners would be walking along the sidewalks until midnight or later.  On this night, though, the streets were deserted.  If there was a curfew, Jesse had missed the memo.

``Doesn't anyone take walks in this part of town?'' Sascha asked, taking a broad look around them.

``Normally I would be taking one.  The officer who dropped me off mentioned that this was a very active neighborhoods.  Apparently there are 5K runs five times per year during the warmer months.  It's my kind of place when I'm not being stalked.''

They drove past a red bus shelter where an old man with a cane was waiting for the red line bus to take him out of town.  The lights were still on in a barber shop on the next block.

``Slow down,'' Jesse said.  He had spotted the Thai restaurant that had become an important navigation marker for him as he learned the neighborhood.  ``Do you see the pair of benches right there?  Next to the fountain.''

Sascha squinted, trying to make out the dark shapes through the glare of the lights above the street.  ``Got it.''

``Just past those benches is a dark spot.  People park there and go into the shops or go for a walk in the park behind the fountain.  The car won't look out of place, and we'll have a great view.''

Sascha pulled over and tucked the car's right tires into the gutter, being careful not to scrape them on the curb.  He killed the engine and set the parking break, leaving the keys in the ignition this time.

``So, I've been wondering,'' Jesse said.  ``How did you know that my new place was on North Piata?  Wouldn't that seem like any old address?''

``The detective running the phone records for me let it slip that the NYPD maintained a few housing units for witness protection and such in that area.  Loose lips, I know.  And it definitely clouds the picture when we're trying to figure out how the address got out.  He made me swear that I wouldn't mention it to anyone, and I suppose I still haven't.''

Jesse shook his head and glanced at the side mirror of the car.  The night was completely still except for the occasional car that rolled past.  They both felt out of place on the street, but the potential payoff was worth the discomfort and boredom.

After a couple of hours, Jesse began to nod off.  It had been a long and difficult day, and there was absolutely nothing happening that could hold his attention.  Sascha noticed the sudden stillness around him---outside the car as well as \emph{inside}---and nudged Jesse awake.  ``I'm getting hungry,'' he said.  ``Can you watch for a few minutes?  I'll get enough for both of us.''

``Sure,'' Jesse said, rubbing his eyes.  The short sleep had refreshed him, and he was suddenly eager to watch the activity outside of his building again.  It was after midnight.  There would be no motion on the street until the bars closed for the night.  Jesse hoped that Sascha would be bringing coffee along with dinner.

Jesse thought he saw a dark shape moving along the sidewalk in the distance, very close to the wall of the buildings.  He blinked a few times to clear his vision.  He couldn't be certain, but a couple of times he though that he had seen movement.  It was too high to be a cat.  The motion might have been at waist level on an average-sized person.

Before Jesse could decide whether his eyes were deceiving him, Sascha emerged from the front of the Thai restaurant.  He was carrying a drink tray and a large bag.  \emph{Good man,} Jesse thought.  The momentary distraction, though, was enough to cause Jesse to lose sight of the apparition that he had been tracking.  He tried to find it again while Sascha walked toward the car.  There was nothing there.

Sascha took a detour through the park to throw off his scent, just in case someone happened to be watching the street.  Counter-espionage was a delicate business.  He approached the car from the fountain, opened the door, and quickly got in.  ``Did I miss the bad guys?'' he asked as he pulled a box of rice from the bag.  He handed the rice to Jesse along with a fork and a pair of chopsticks.

``Well, not exactly.  But I thought I saw something.  Along the left-hand sidewalk, against the wall of the buildings.  I saw movement a couple of times, but then I lost it.''

Sascha stopped rooting on the bag and looked at Jesse.  Then he turned his head toward the street, focusing on the sidewalk that Jesse had mentioned.  Suddenly his face changed.  ``There,'' he said in barely more than a whisper.

Jesse, who had been watching Sascha rummage in the bag of food, snapped his head around to look at the street.  A tall woman had just appeared out of the darkness.  She was wearing a long black jacket, a dark brown scarf, and a fedora.  Her hair was shoulder-length and dark.

``Something about that doesn't look right,'' Sascha said.  ``Do you recognize her?''

``Not at all,'' Jesse said.  He squinted again, trying to make out the woman's face.

Sascha pulled a small pair of binoculars from the breast pocket of his coat and unfolded them.  Looking through the glasses, he was able to see her face.  ``She has a small black box in her left hand,'' he said.  ``Dangling earrings.  White gloves.  I'm no fashion expert, but it's a strange ensemble.''

They both watched as the woman approached Jesse's building.  As she walked into the glow of the light over the building's front door, her pace slowed.  She walked up to the door, used a key in her right hand to unlock the door, made no visible effort to check her surroundings, and slipped inside.

Sascha set the binoculars in his lap.  ``Do you have any neighbors up there?'' he asked.

``Yeah, there's one other unit,'' Jesse said.  ``It will be on the left side of the door.  I haven't seen anybody come or go from that place.  Maybe she'll turn her lights on.''

They watched the building for several minutes, ignoring the food that had completely occupied their attention only moments earlier.  Nothing changed.  There were no lights on, and the woman had not left the building.

Jesse frowned.  ``All of the rooms have windows that face this direction,'' he said.  ``If there was a light on, we would see it.  What's she doing?''

Suddenly, the door opened again, and the woman stepped into the light.  She closed the door firmly, walked across the street, and walked back in the direction from which she came.  Sascha quickly picked up his binoculars and watched her fade into the blackness.

``Should we go check it out?'' Jesse asked.  If they checked the lock, then it would be easy to determine whether the woman had picked it.  On the other hand, by entering the building they might cause any further action to be aborted.

``No,'' Sascha said.  The firmness in his voice left no room for argument.  ``She wasn't in there long enough to look for anything.  If she was one of the goons we're looking for, then her job was probably to check out the scene, determine whether you were home, and then pick the lock if necessary.''

``And if I had been home?'' Jesse asked.

``Hard to say.  She might have knocked on your door and then apologized, making a simple excuse.  Maybe she was looking for her friend, and she must live in the other apartment.  Or maybe she had a nine millimeter under her coat.''

Noting the strained look on Jesse's face, Sascha went on: ``Hey, don't sweat it.  That's why we're out here.  Keeping watch is our only shot at snaring these cats.  My gut tells me that there'll be another visitor tonight.  These guys have probably done this before, and it works best in stages.  We did it the same way during drill exercises.''

``All right,'' Jesse said.  ``You're the boss.  What did you bring me for dinner number two?''

Sascha divided the food between them, and they enjoyed the hot meal.  Sascha had bought two large cups of coffee to keep them alert during the wee hours.  Cream and sugar for Jesse, and straight black for Sascha.  Drinking the coffee also took their minds off of the boring work of watching the building, making it easier for them to stay awake and on task.

One hour went by, and the two.  And then three.  The closing time for the bars had come and gone, and they saw only a handful of people walking on the sidewalks.  Jesse and Sascha talked about the projects that the current batch of students were pursuing.  When that topic dried up, they talked about a date that Sascha had had the previous week.  When Sascha asked about Jesse's night life, he just laughed and changed the subject.

When the sky had begun to glow toward the east, Sascha asked if Jesse wanted to give up for the night.  ``I might have had their timeline wrong,'' he said with a shrug.

``I don't know,'' Jesse replied.  ``I'm not in a hurry to go in there, especially after seeing that woman earlier.''

``Don't worry; I'll go in there with you.  If anything seems fishy, then we'll find a safe place to get some sleep.''

Jesse finished his last drop of coffee and opened the car door.  The morning chill was brisk, so he pulled on his gloves and buttoned his coat.  He walked around to the driver's side of the car, checked for traffic, and walked with Sascha to the other side of the street.  They both felt alert for the hour.  It was quiet enough that Jesse could hear the soles of his shoes grinding against the sidewalk with each step.

They crossed one street, then another, and finally reached the entrance to the building.  Jesse pulled his key from his pocket, unlocked the door, and pulled it open.  He stepped over the threshold and held the door for Sascha, who made a visual sweep of the street in both directions.  No one was in sight.

In front of them was a staircase that led to a small landing.  There were two opposing doors, one on either side.  Jesse checked the doorknob, first visually and then with his hand.  ``It's locked,'' he said.

Sascha shrugged.  ``Let's check it out, then,'' he said.

Jesse found another key on his keyring and stuck it into the doorknob.  He turned the handle and pushed the door open.  As he took a step forward into the apartment, he heard a sound from behind him and froze.

``Stay where you are.  I want to see your hands.''

It was a woman's voice.  Jesse slowly raised his hands into the air.  He looked around at Sascha and saw that his jaw was clenched and his eyes were tightly closed.  \emph{Caught us both.}  The voice had an accent, but Jesse couldn't place it.  It wasn't anything that he had heard before.  He could feel his heart pounding in his chest, and his sleep-deprived, caffeine-soaked brain wasn't providing any good solutions.

Sascha turned his head to look toward the voice.  ``Who the hell are you?'' he shouted.  ``What do you want with us?''

The sound of a revolver being cocked echoed through the hallway.  Sascha felt his adrenaline surge.  ``That depends on how cooperative you are,'' the voice said.  ``Go inside, and keep your hands showing.''

Jesse and Sascha took several steps forward.  They heard the woman mutter something in a foreign language.  The sound was muffled, as though she was speaking into her sleeve.  She stepped into the apartment and closed the door, keeping her gun raised and pointed toward the two men.  For the first time in his life, Jesse wished that he kept his knives on his countertop instead of in an overhead cabinet.  Even in his temporary apartment, he kept them carefully out of sight in case someone broke in when he was asleep.

``Turn around.''

Sascha turned around first, anxious to see his captor's face.  She was older than he expected.  It was not the woman whom he had seen through his binoculars.  She was wearing a gray overcoat and tall black boots.  The look on her face was one of boredom.  Sascha knew that a dozen hours of waiting could do that to you, but his boredom had ended with a click on the landing outside.  He knew instinctively from the way that the woman held the revolver that she wasn't a trained in small arms use.  She was also standing in a way that made her easy to topple.  Sascha wasn't ready to act on that information just yet, but he was glad to know that his trained mind was working on his behalf.

``Your names, please,'' the woman said.  ``And no funny business.''

They both gave their first names.

``Yes, hmm,'' she said.  ``My associates will be here in a moment.  Till then, do tell me where I can find the recording.''

It was Jesse's turn to be anxious.  The woman's statement had removed any doubt about her intentions, but there were still plenty of questions that Jesse wanted to ask.

Sascha beat him to it: ``What's your name, ma'am?'' he asked.

She turned the gun toward Sascha and gave him an odd look.  ``My name is Beata.  You can call me that.  Now, back to business.  You have a recording that I want, and it will be best for you if you cooperate with me.  My associates will not be so friendly.''

``I don't know what you're talking about,'' Jesse said, making a defiant stand.  He knew that this wouldn't work, but he was at a loss for what else to do.  They needed to find a way out of this, and preferably before the goons arrived.

``No funny business,'' Beata repeated, training the gun toward Jesse.  ``I know that you have it.  You told a man in Warsaw that you have the recording.  Lying to me will not help you.  It would be most useful to me to have it.''

``It isn't here,'' Sascha said.  ``There's nothing here for you.''

Beata sighed, her face beginning to show impatience.  ``I will not have you wasting my time, gentlemen.  You will tell me where to find the recording, and I will send my men to get it.''

``I'm sorry, but that's not possible,'' Jesse said.  He was shaking now and tried to will his legs to be still.

``And why is that?'' Beata asked.  Her voice was flat, and her annoyance at these obstacles was apparent.

``The recording is locked in a secure vault at a local university,'' Jesse lied.

``Then you will be taking us there,'' Beata said.  ``When my men arrive, you will tell us where to drive, and then you will unlock the vault.''

Jesse said nothing.  There was no way that Beata's plan would work.  The vault was locked, and Jesse had no key or passcode to open it.  Beata and her goons might remember how to manipulate the computerized scheduling system to defeat the door security, but the recording wasn't in the vault anyway.  Jesse was more interested in buying a bit of time, though.  He was confident that Sascha would think of a way out of this, but they had been dealt a difficult hand.  Rushing Beata and trying to overpower her ran the risk of someone getting shot, and she was expecting reinforcements any moment.  There was no easy escape from the apartment except through the front door.

A moment passed in which the three of them glared at one another.  Suddenly, there was a loud buzzing noise that seemed to come from the ceiling.  After cringing and inhaling sharply, Beata regained her composure.  She backed toward the door, keeping her gun level and pointed toward Jesse and Sascha.  She glanced at the panel on the wall that controlled the intercom system.  With her free hand, she pressed the button that unlocked the front door.  They all heard the door slam shut at the bottom of the stairs.  Heavy footsteps shook the floor.

When the door opened, three heavy-set men were standing on the other side.  Beata said something to the men in a foreign language.  The men nodded.  ``The gentlemen were just telling me,'' she said to them in English, turning back toward Jesse and Sascha, ``that the recording is in a vault at a university nearby.  They will be taking us there, and they will unlock the vault.''

``But first,'' she went on as she turned back to her men, ``I want you to search the apartment.  These men have not been entirely honest with me in our short time together.  It may be here, and I want to be certain.''

The three men walked past Beata, Jesse, and Sascha, and began looking around the apartment.  It was an easy task since Jesse had only brought basic provisions, expecting that his stay would be short.  They heard the men lift a few pieces of furniture, presumably looking under them for any hidden treasures.  Next, they searched the bathroom and dismantled Jesse's bed.  Jesse did not relish the thought that he would need to reassemble his apartment when this was all over.  The thought was fleeting, though.  The men had returned to the room empty handed.

``Nothing,'' one of them said in heavily accented English.

``Well,'' Beata said, ``then we will go to the university.  After you, gentlemen.''  She used the gun to gesture toward the door.

Jesse took the first step toward the door.  The next moment was pure pandemonium.  Jesse and Sascha heard the glass of the front window break.  Immediately, one of the burly men fell to the ground, his dark shirt soaked in a spreading liquid.  Another of the men clutched his arm and staggered backward, falling onto the floor after his third step.  The third man pulled a gun from underneath his coat and pointed it at Sascha.  He began stuttering nervously in a language that neither Sascha nor Jesse could understand.  He ducked behind a cabinet, trying to get out of view of the window.

Beata had moved so that her back was against the door that led to the stairway.  She could not see the window from there, but she was safe from further gunfire.

Sascha had hit the deck as soon as the shot rang out.  He recognized the tactic, but he couldn't be sure of the shooter's allegiance.  Was it a police force that had been watching the apartment?  Was it a shot intended for Jesse or Sascha that had taken a bad bounce?

Sascha barely had time to wonder about the situation before he heard more heavy footsteps on the stairway.  A look of fear come over Beata's face, and she scurried across the room to take up a position next to her remaining comrade.  They both aimed their guns at the door.  Jesse and Sascha, sensing that more violence was on the way, crawled out of the room.

When the door apartment door yawned open, nobody was there.  ``Who is there?'' Beata called, a clear note of nervousness in her voice.  ``We will shoot!''  Neither Beata nor her fellow gunman saw the small mirror that rounded the edge of the doorframe.  An instant later, two men in black clothing and thick vests burst around the corner with automatic guns in their hands.  Beata and her comrade fell before they could return fire.  Another three men and one woman in dark clothing came through the door behind the first two.  They quickly spread through the apartment, systematically checking each room.  When they found Sascha and Jesse, they ordered them to their knees and kept their guns trained on them.

``Check them,'' one of the men said.  Rough hands worked each of them over, looking for any sort of weapon.  A moment later, satisfied, the checkers stepped back.  Sascha felt relief despite the aggressive treatment.  S.W.A.T. teams had to be thorough, and they had no way of knowing what he and Jesse might try to do.  At any rate, he knew that they were safe.  They also had a very good case that could be used to root out any informants within the NYPD.

The women stepped forward.  ``Mr. Winter,'' she said, looking at Jesse, ``my name is Jenna.  We know that the gunmen were looking for a recording of a conversation.  Did they find it?''

``No, they didn't,'' Jesse said.  He felt oddly relieved to be able to speak about the recording openly.  It had been a difficult secret to keep, especially after the story of his success in Warsaw had hit the presses.  ``We told them that the recording was at the university in a vault, and they wanted us to take them there.  And then everything went to hell in here.''

``All right,'' the woman said.  ``Very good.  We're going to clean up a bit in here, and then you're going to take us there.  By force, if necessary.''  She turned to the rest of her entourage and gave a few terse commands in a foreign language.  They sprung into motion, several of them tending to Beata and the dead gunmen.  She spoke English with a flawless New York accent, and this had led Jesse to believe that she was a member of the local police.  It was only when she spoke to the men in a foreign tongue that he suspected something was amiss.

Jesse sat frozen in place, his mind racing.  He turned to Sascha, who had a look of bafflement on his face.  ``They're not the police?'' Jesse asked quietly.  ``Who are they?''

Sascha shook his head.  ``I think we've just been caught in the crossfire,'' he said.

``You're going to feel the wrath of the party this time, Mr. Winter,'' a woman's voice said.  It was the same woman who had spoken a moment before.  She approached them slowly, looking pleased with her position of power.

``You don't know me,'' she said, ``but you've met my cousin Marcel.  I was a student in your program.  You might recognize my name.  I'm Jenna Sternoza, and my research project was in Madrid.  Being a student here was very good for me.''  She paused meaningfully.  ``And it was good for my family.''

``When you came to our country and demanded the release of your spies,'' she went on, ``Marcel had no choice but to accept.  He had no way to know whether you were lying.  Since then, it's come to our attention that you were, in fact, telling the truth.  Naturally, the fact that a recording exists that could damage my family's reputation and cause us to lose the trust of the people of Warsaw is bad for business.  That's why you'll be taking us to the university and giving us every copy that you've made.  If you cooperate, then neither of you will be harmed, and you'll be free to go about your petty and spying-obsessed lives.  If you try to deceive us, then your fate will be much like our friends here.''  She gestured toward the bodies that were being searched by her men.

``Wait a minute,'' Sascha said, breaking his silence.  ``If you're with the party, then who are they?''  He nodded toward the bodies.

``Another interested party.  They are associates of the elderly woman who spoke to my uncle.  The woman whose voice is on the recording.  If they managed to get a copy of it, they'd do what you haven't had the guts to do.  They would sell the story to every yellow rag across Europe.  It would make them rich, and it would be the end of my family's prosperity.''

Jenna pulled a small handgun from the holster on her right hip and pulled the hammer back.  ``Let's make sure that doesn't happen, all right?  Get up.''

Jesse and Sascha both got to their feet.  They had been on their knees long enough that their legs ached as they rose.  ``The woman who showed up last night,'' Sascha said.  ``Black coat.  That was you.''

Jenne gave an amused half smile.  ``You've got a great memory for faces,'' she said, her voice laced with sarcasm.  ``But apparently you don't have the patience to handle a stake-out.''  Turning her back to them, she barked another set of terse orders to her men.

``Let's hit it,'' she said in English.

Sascha was trying to work out how he could quietly set off on alarm at the office once they arrived.  There was no system in place for alerting the police of an armed robbery or assault.  The university wasn't a bank, and the board of directors had decided that the money should be spent elsewhere.  If there was no way to trigger an alarm, then they might have to cut their losses and turn over the recording---wherever, and whatever, it happened to be.  Sascha doubted that the recording was in the vault.  It would have been trivially easy for Jesse to put it in there, but a digital storage device of any sort would be suspicious, and every student file was reviewed at least once per year.  However, leading this group of thugs to the wrong location would be as good as suicide.  They weren't going to put up with any tomfoolery.  Jesse was leading them to the right building, and that's where he and Sascha would have to make their stand.

The drive to CUNY was an uncomfortable one, but Jesse felt relieved to be out of the apartment.  The sight of the dead bodies made him nauseous, and the broken windows had left the place unbearably cold.  The police hadn't arrived to investigate the shattered glass, which led Jesse to wonder whether his neighborhood had suddenly been abandoned.

On their way out, Jenna's men had led Jesse and Sascha down the stairs and to the street.  All of them kept their weapons hidden, including a sniper who joined them from a post across the street.  He was the one who guaranteed an easy sweep of the apartment by disabling two of Beata's men when the moment was right.  His vantage point would have provided an excellent view of the building and part of the street.  In the game of spy versus spy, Jenna's crew was far superior.

Jesse sat squeezed between Sascha on his left and one of Jenna's gunmen on his right.  The other gunmen were in the seat behind him.  They were all tucked into a conversion van that had deeply tinted windows and a sour smell that reminded Jesse of spoiled milk.  The driver was taking odd routes, and eventually Jesse realized that he was avoiding high-traffic areas.  The van would look suspicous with the tinting job, and any search of the vehicle would turn up conflicting information and a number of weapons.

Jesse sat in the front passenger seat.  ``So Mr. Winter,'' she said as she surveyed the street, ``how many copies of this private conversation have you made?''

``No copies,'' Jesse said.  ``The only one that I have is the one that I was given.''

``Hmm,'' Jenna said.  ``You should always back up your stuff.  Especially the important things.  Very amateurish.  Anyway, you say that you were given that copy.  How did that happen?  What business is it of yours to have it?''

Jesse was suddenly aware that the man beside him was glaring at him, waiting to hear the response.  ``I got it as part of a briefing before my trip,'' he managed to say.  ``I don't know the source.''

``Listen, mister journalism-is-my-life,'' Jenna said, dropping all attempts at pleasantry.  ``I know the routine.  I've sat through your classes and got your joke of a degree.  You're supposed to protect your sources.  I get it.  But this isn't about state secrets or Berlusconi's latest scandal.  This is about you telling me what I want to know or dying right here in this van.  Your choice.''  She pulled her revolver out of its holster and made a show of checking its chamber.

``He's telling the truth,'' Sascha interjected.  ``There are people at CUNY who handle intelligence matters and interact with the feds.  We're not cleared for that stuff.  Jesse would have been given a thick stack of stuff to read, and maybe a few digital files to look through.  That's how it works.''

Jesse gave Sascha a look that communicated his gratitude.  Sascha gave a slight nod and then looked back at Jenna.

Jenna appeared to contemplate the sub-optimal response to her demands.  ``It's right there,'' she said, pointing out the windshield toward a large parking structure.  The driver nodded.  She quietly holstered her handgun and pulled her coat over it.  ``Now listen to me,'' she went on, ``this is how we'll do this.  Mr. Greene, you'll stay in the van.  Don't protest.  I'm not interested in your lip right now.  Mr. Winter, you're coming with us.''  She turned to the driver.  ``If you don't hear from me within ten minutes, then you have permission to shoot to kill.  For both of them.  Understood?''

Everyone stared at her.  The gunmen mumbled a response that sounded affirmative.

``Let's go,'' Jenna said and threw open her door.

Jesse followed the gunman to his right as he climbed out of the van.  One of the men from the back seat got out as well.  He closed the door behind him.  The driver had also gotten out and joined them.

``You'll lead the way,'' Jenna said to Jesse.  ``And don't waste my time.  You heard my instructions back there.''

Jesse took a deep breath, looked back at Sascha's dark shape in the van, and walked toward the building whose sign read \texttt{SHECKELL}.

It was still early enough that no one was at work, draining Jesse's hope that he could communicate his distress to someone he knew.  He led the entourage through the entrance to the building and walked toward the north elevator.  They were two floors above the floor that housed the vault.  Jesse punched the button labeled \texttt{B2} and waited for the doors to close.  He wished that Sascha were here.  There was much more space available to them now that they were out of the apartment, and Jesse knew that Sascha would be able to manufacture an escape.  \emph{Jenna probably knows just how capable he is, too}, he thought.  \emph{I'm on my own this time}.  He still had no idea what he would do when they got to the door of the vault.  He had no key, no passcode, and no intention of opening the door.

The slow elevator finally arrived at \texttt{B2}.  After it settled, the doors opened to reveal a dim hallway that led away to the right.  Jesse stepped out of the elevator and led the way toward the heavy steel door that was labeled only \texttt{RECORDS}.

When they arrived at the door, the gunmen formed a semi-circle around it, blocking Jesse's exit.  Jenna stepped forward, looking carefully at the door and the keypad beside it.  ``Punch in the code,'' she said.  ``Let's see it.''

``I don't have a code for this room,'' Jesse said.  ``We need someone from administration to open it.''

Jenna stepped closer to Jesse and grabbed the front of his shirt.  ``Listen, you little shit,'' she said.  ``I'm tired of your lies.  This building will be full of people in less than an hour, and I'm going to have that recording.''  She released Jesse's shirt and shoved him backward.  He stumbled into one of the gunmen, who shoved him back toward Jenna.  He managed to get his footing again.

``For someone so smart, you really don't seem to get it,'' Jenna said, putting her hands on her hips and pacing.  Then she gave a scream of rage, grabbing hold of the door and jerking the handle back and forth.  She fell backwards and landed with a hard thud as the door yawned open.

The gunmen murmered something to one another.  Jenna lifted herself onto her elbows and gazed at the door.  Her rage from the moment before had vanished and was replaced by a look of confusion and victory.  ``Well, well,'' she said.  ``I guess your stalling tactics are finished.''

Jesse scratched his head absently and took a step back from the swinging door.  His anxiety made everything seem to go by in slow motion.  The door to the vault was now open, and Jenna was gloating about it.  \emph{The security system doesn't turn off at night}, he thought.  Then the more pressing problem hit him: Jenna had thought that he had already been stalling for time, but the recording wasn't even in the vault.  He would have a few minutes to work with while she perused the student files to find the folders for the Polish research project.  After that, though, he would be facing another tongue lashing---or worse.

``Get in there,'' Jenna said to Jesse, grabbing his shoulder and pushing him through the open vault door.  ``Find me the folder.  NOW.''

Within the student records vault was a collection of towering steel shelving units.  For anyone familiar with library catalogues, it was easy to navigate.  Each shelving unit was labeled with the range of student last names that could be found on its shelves, and within each shelf the records were sorted alphabetically.

Jesse located the shelf labeled \texttt{R}.  ``The recording is in Sam Rzeznik's file, so it will be here somewhere.  It'll take me a few minutes to track it down.''

``Make it fast,'' Jenna snapped.  She uttered something to one of her men who nodded and pulled out his mobile phone.

\chapter{}

In the tinted van, Sascha felt himself begin to sweat.  One of the men in the van had just gotten a text message, and it had been more than ten minutes since Jesse left.  He knew that Jesse was still alive, but it was impossible to know what was happening.  If Jesse had managed to get into the vault, then he would be trying to buy some time.  The building would start filling up within thirty minutes, and a group of armed men in the vault would look highly suspicious.  But Jenna would know that, too.

Off in the distance, a police car pulled alongside the Sheckell building.  Aware that the men were watching him, Sascha pretended to look stretch his frame and look around.  He did his best to look bored.  A tall uniformed officer got out of the patrol car, and two men dressed in black shirts and black jeans emerged from the back seat.  They conversed for a moment, and then the two men in jeans walked toward a side entrance to the building.  They had purpose in their movement, and for a moment Sascha felt hopeful.  If he couldn't be there for Jesse, then maybe they could get him out of this jam.

\chapter{}

Jesse took his time, pulling one box after another from the shelf and looking through the records.  No one was watching over his shoulder, and he had looked through a few extra boxes to burn some time.  Occasionally he would pull a folder from a box and leaf through it, just in case one of the men happened to be looking at him.  He guessed that the file would be in the next box, but an idea had just occurred to him.

Jesse glanced around him, ready with the excuse that his neck and shoulders were getting sore.  No one challenged him.  He noted the location of each of his captors.  Two of the men were strolling away from him between two of the great shelving units.  Jenna had pulled a file and was looking through it.  The third man was studying a chart on the wall that showed the layout of the building.  \emph{I am the captain of my soul}, Jesse thought.

He turned on his heel, hoping that his first step would find traction.  It did, and he propelled himself toward the open door.  By the time he reached it, the men had realized what he was doing, and two of them had moved a hand to their right hip.  Jesse grabbed the heavy door as he passed it, using its inertia to swing himself around behind it.  Planting both his feet and summoning all of his strength, he heaved the door closed.  The floor shook beneath his feet as the heavy steel locked into place.  Jesse ducked quickly to one side, aware that the glass window in the door might not be bulletproof.  He could hear Jenna screaming at her men on the other side of the door.

Jesse edged himself away from the door until he was safely out of sight, and then he began to walk toward the south elevator.  \emph{Now what?} he thought.  There was no courtesy phone on the current floor.  He would need to get outside but stay out of sight of Jenna's men in the van.  They would be looking for him.  Jenna was probably notifying them already.

\emph{Oh, shit}, he thought.  \emph{Sascha}.  Not only would the men be looking for him, but they still had Sascha in the van.  If they couldn't find Jesse, they might torture or threaten to kill Sascha in order to get him to disclose the location of the recording.

Jesse punched the button for the south elevator and glanced back toward the vault.  He thought that he heard a beep from the elevator at the other end of the long hallway, but he dismissed it.  Seconds later, the elevator door opened.  Jesse rushed forward and collided with a man wearing dark clothes.  The man put his hands on Jesse's shoulders to steady him and took a step backward.  It was Detective Forth.

``Detective, I---the vault---it---'' Jesse stammered.

``I know,'' Forth said.  ``We knew that something was wrong when your phone was off for so long, and the wreckage in your apartment was a dead giveaway.  Show me the vault.''

Jesse led Forth back down the long hallway to the vault door.  Forth glanced inside, ducking back when he saw the men with loaded weapons.  He raised an open hand in front of the door for a brief moment, and then began punching buttons on the security keypad.

``Detective, no!'' Jesse shouted.  ``They'll kill us!''  He backed away, ready to run for the stairwell.

Forth punched the \texttt{UNLOCK} button and looked sidelong at Jesse.  ``It's time to face the music, Mr. Winter.''  He pulled the large handle, and the heavy door yawned open again.  The red-faced gunmen and Jenna came out, glaring at Jesse.  Jenne stepped forward and shoved him.  ``Did you think that little prank was cute?'' she said, hardly interested in an answer.  ``I'll show you cute.''  She pulled her handgun from its holster and grabbed Jesse by his shirt, pushing him against the wall.  She pushed the gun into his cheek, and he could feel the cold steel compressing his nasal cavity.  He closed his eyes.

``Jenna,'' Forth said, ``we need him alive.  Don't make this complicated.''

``You think this was easy?'' she said, turning her head toward Forth.  ``Getting into the country without getting harassed by your know-nothing agents was hard enough.  Getting to this point was hard enough.  It's time to end this, and I'll have blood on my hands if that's what it takes to vindicate my family.  The people \emph{need us}.  I know my cause.''

She turned back to Jesse.  ``All right.  Any more bullshit and I'll pull the trigger.  Got it?''

Jesse nodded weakly, his face beginning to burn from the pressure of the gun barrel.

``Where is the recording?  Is it in this building?''

``Yes.''

``Where?''

``It's---'' he faltered.

She pressed the gun into his face even harder, making the back of his head hurt and the pain in his cheek explode.

``It's upstairs,'' Jesse squeaked.  Jenna eased the pressure on the gun, but the pain hardly subsided.

``Let's go,'' she said.  ``Now.  If I see you wasting time, you're dead.''

She pushed him toward the north elevator, her gun aimed at his back.  Jesse walked at a brisk pace toward the closed elevator.  He had no idea what time it was, but he couldn't risk any more stalling.  If someone was going to save him, then it would happen with the cards that were on the table.

Jesse punched the button on the elevator and waited for it to arrive.  He would be taking them to the Office of Internal Affairs.  Sascha's office.  He had neven been in there without Sascha, but somehow he knew that his friend wouldn't mind, given the circumstances.

When the door opened, Jesse stood in shock.  There was a muffled metallic sound and a flash of light from the dim elevator, and he heard Jenna grunt behind him.  She collapsed in a heap, and her gun slid down the hallway out of reach.  He heard movement farther down the hallway, in the direction of the vault.  ``FREEZE,'' he heard a man's voice yell.  Jenna's gunmen wheeled around, panic in their eyes.

Forth, who had been standing on Jesse's right side, made no move.  His gun had never been unholstered, and he was in no position to win a draw.  Slowly, he raised his hands.  Behind them, Jenna was groaning.

Bradley Johnson stepped out of the elevator along with another man who Jesse didn't recognize.  Both were wearing bulletproof vests and carried automatic rifles.  ``Detective,'' Johnson said with a nonchalant air.

``Johnson,'' Forth said.  ``Web of lies.''

``Only your own.  Get on your knees.''

Forth sneered and opened his mouth to protest, but Johnson shouldered his weapon.  Forth slowly dropped to his knees and put his hands on his head.  Johnson nodded to the man beside him.  The man moved behind Forth and pulled a pair of heavy handcuffs from his belt.  For Jesse, the cuffs made a satisfying series of clicks as they closed around Forth's wrists.

Jesse remembered that his dear friend was still at the mercy of the goons outside, and his heart nearly stopped.  ``Sascha---he was in the---'' he started.

Johnson smiled and put a hand on Jesse's shoulder.  ``He's fine.  You can relax now.  We knew that we needed to hit these guys first, but we had a backup unit outside to sweep the vehicle.  They roughed him up a bit, but he'll be good as new in a few days.''

Jesse exhaled.  The combination of the day's events and his lack of sleep left him wondering when the other shoe would fall.  \emph{Has to be the fifth shoe by now}, he thought.

Outside, Sascha was sitting in the back of an ambulance.  When he saw Jesse, he jumped out of the ambulance and gave him a bear hug.

``I guess Johnson was right about you,'' Jesse said with a laugh.  ``How bad did they thump you?''

``Eh, it was nothing I haven't survived before.  Those guys were more bored than angry.  She didn't pay them enough.''

Jesse grinned.  ``Let's get something to eat,'' he said.  ``I have a feeling the feds are going to be processing this one for a while, and I'm not sitting through any more meetings on an empty stomach.''

``As long as it's not Thai,'' Sascha said, screwing up his face.  ``I can't get the smell of that stuff out of my nose.''

\chapter{}

A few days later, Sascha sat at his desk, sorting through a few hundred emails that had accumulated in his inbox.  He recognized the names of a few reporters who had contacted him about university issues that had leaked to the press.  There were a number of names that he also recognized, although in a different part of his brain.  The chief of a local television affiliate.  A New York state senator.  An actress who had handed him a cocktail at a holiday ball the year before and told him a mostly-ridiculous story about the way that her cat acted when she forgot to wear her hair in a ponytail.  Everyone seemed to know that he was involved in the recent events, which was odd.  The \emph{Times} had broken the story, but details were light.  The NYPD hadn't held a press conference, nor had it released a full statement to explain the involvement of some of its own in an international scandal.  No one was surprised by this, but the reporters were clearly aching for a story.

Sitting on a stool between Sascha's desk and the office door, his head tilted back against the wall, eyes closed, was Jesse.  He looked exhaused but peaceful.

Sascha pulled off his reading glasses and leaned back in his chair.  He looked at Jesse.  ``You're a model of calm,'' he said.  ``That's the most relaxed you've been in three months.''

``I wish my stomach would catch up,'' Jesse said without opening his eyes.  He had never been good at dealing with prolonged periods of stress, and the pain in his gut reminded him that anxiety had consequences.

``So I'm just wondering,'' Sascha began.  ``Wh---''

Jesse opened his eyes.  ``Look at Gumby,'' he said.

Sascha frowned.  He looked at the green rubber humanoid figure next to his computer screen.  He poked it, as he often did, and it responded by jiggling for several seconds.  Jesse had given it to him as a Christmas gift the year before.  They had a history of giving each other useless gifts that brought back youthful memories.  Gumby had been one of Sascha's favorite shows.

``I don't get it,'' Sascha said.  He felt dumb.  There was nothing there but rubber.

Jesse grinned.  ``Look closer.  Gumby was a \emph{brainy} guy.''

Sascha leaned forward, rolling his chair toward the desk.  He pulled the Gumby figure off of the steel post that lent it the ability to wobble.  He ran his fingers over its front and back surfaces.  Nothing.  It didn't feel heavy enough to contain any hidden material.  Something about the way that Jesse said \emph{brainy}, though, suggested that it wasn't just idle use of the word.  Sascha turned the Gumby over in his hand so that he was looking down at its head.  At first he saw nothing, but after staring for a moment he noticed an impossibly thin scar in the top of the figure's head.  Using two thumbs, he managed to pry it open slightly.  Inside, there was a glint of silver.

``USB port in the head,'' Jesse said.  ``Ever feel like you need one of those?''

``That's not funny, man,'' Sascha said.  ``We might as well have one, but I sure as hell don't \emph{need} one.  You put the goods on a USB stick?  And you left it on my desk?''

``Well, technically \emph{you} left it on your desk,'' Jesse said.  ``Nobody knew it was there, and that's exactly how it needed to be.''

Sascha worked the USB stick out of the figure, threw it at Jesse, and pressed Gumby's rubber head back into shape.  ``No more!'' he said, carefully fitting the figure back onto its metal rod.  He gave it a poke for good measure.

Jesse looked at the device in his hand.  The cause of three months of distress.  It had given him countless nights of broken sleep.  At least one disastrous catnap at work.  Nearly the loss of a good friend's life, if not his own.

Sascha was watching him.  ``Was it worth it?''

Jesse pursed his lips, turning the USB stick over in his hand.  ``Now that my life isn't being threatened, yeah.  It was.  But there were a few times, like yesterday, when I would have gladly traded the recording for our safety.  It's hard to be idealistic when an armed, raving band of lunatics---especially two of them---have you pinned in a corner.  This is the kind of stuff that they can't put on the brochures.  Not because it's hard to explain--the \emph{Times} will take care of that for us---but because it sounds gratuitously romantic, heroic, or whatever else.''

Sascha nodded.  ``It's the same in combat.  We don't promote it in the recruiting literature because it would glorify something that takes a tremendous toll on everyone involved, but it can be a defining moment for them too.  Not in the sense of validating their preconceived notions about war, but showing them how complicated and twisted the world can be.  We want to think that everything is fair, and that debate and compromise is how everything works.  It isn't.  It just isn't.  Sometimes there are things worth doing, and sometimes they suck.  People die.  But that doesn't make it less important to do the work.''

``I think I'm ready to retire after this one.  Do you think the university will let me take a pension at 32?''

Sascha's desk phone rang.  He shook his head and picked up the receiver.

``Greene,'' he said.  A pause.  ``Yes, sir.  Understood.''  A long pause.  ``Interesting.  Yes, he did.  Thank you very much.  Bye.''

Jesse raised his eyebrows as Sascha dropped the receiver into its cradle and looked at him.

``It was Johnson, the intern.  The head of the NYPD has been indicted on criminal espionage charges.  Something to do with a big crime ring in Europe.  Forth was just the fall guy.  Just the tip of the iceburg.  It's hard to say how far this is going to go before it's all aired out.  The reason for the delay was that the FBI needed to coordinate with Interpol to take care of some flight risks in Europe before the news broke.  It sounds like the capital-p-party is going have a real bellyache.''

Jesse was stunned.  He had hoped that the Forth would see prison time.  Based on what had happened in the previous 48 hours, tying Forth to Ingo and establishing that they were cooperating in organized crime would be easy.  Jesse had never met the head of the NYPD, though, and couldn't believe that this dust-up was going to have such broad reach.  Talk about bringing down an elephant.

``I can't imagine what this is going to mean.  Should we take a vacation and lie  low for a few days?'' he said.

``That's what Johnson suggested,'' Sascha said as he got to his feet.  ``Everyone is as bewildered as you are, and there's no way for the feds or Interpol to be sure that they've arrested all the important actors.  There are doubtless other agents like Ingo here in the states, and there may be standing orders to retaliate if their leaders run into trouble with the law.  You might get that retirement wish sooner than you think.  Time to start thinking about book ideas.''

Jesse pulled Sascha's coat and his own off of the coat rack.  ``I'm content with a few nights of sleep,'' he said.  ``Then we can talk about book writing and press interviews.''

As the two men left the office, they passed a group of students who had gathered around a television set that was mounted on the wall in the hallway.  On the screen, an FBI spokesman was reading the list of indictments and taking questions from the assembled press.  The headline at the bottom read: ``CUNY investigation uncovers corruption and international crime ring.  NYPD chief, detectives implicated.''

``Just a minute,'' Jesse said as he walked with Sascha toward the elevator.  He ducked into the men's room and pulled the USB stick from his pocket.  One of the only remaining copies in existence.  He snapped the eletronic stick in half, dropped the pieces into the first toilet and kicked the lever on the side.  With a certain satisfaction, he watched them bubble to the bottom of the bowl and then disappear.

\end{document}
