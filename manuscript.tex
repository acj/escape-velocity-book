
"... escape velocity?  It's too much ..." the woman's voice had said.

Jesse stopped, wheeling around to find the voice, but it was lost in the groggy brook of sounds and suitcoats moving toward him, away from him, and breaking around him.  He had just been thinking about his job and remembering his mother's advice.  /When you're stuck but ambition pulls you up, use it, you're already at escape velocity/... .  A tall passerby bumped into him, sending a momentary pain through his neck and shoulders.  The stranger apologized and tipped his hat, quickly regaining his brisk walking pace.

/Nobody else's mother teaches life lessons with physics metaphores/, he thought, allowing himself a wry smile.  He looked across Bedford Avenue, took a deep breath, and stepped back into the stream of Brooklyners.


The City University of New York Office of Internal Affairs was cold this morning.  The calendar above Sascha Greene's desk would lead one to believe that summer raged on, but the brisk morning air left no doubt that late September was a better guess.

"Sascha, the top feed is ready," called a voice near the window.  The Office of Internal Affairs, despite the lofty title, was a cramped, stuffy office that contained five desks, eighteen computers and other office machines, and could comfortably fit three bodies, provided that the bodies were reasonably proportioned and were not given to clumsiness.  Sascha sat facing the north wall, preferring not to be distracted by the constant movement of his colleagues.  Jill, the office technology specialist, sat near the windows where she could keep a watchful eye on the myriad devices that kept the room warm.  Sean, the interdepartmental liason (the External Internal Affairs Spokesman, Sascha called him), worked at a stand-up desk that was mounted on the west wall between the two oat doors.  The remainder of the 600 square foot space was a labyrinth of machines, cables clamped to the floor with black duct tape, stacks of official university records, and the occasional abandoned coffee mug.  The abundance of machines meant that one window propped open year-around.

"Got it."  Sascha stretched his shoulders and called up the day's list of business items that had been assembled by the administrative assistants on the floor above.  Amazing that these lists ever get out the door.  Sascha knew that despite the polished look of the top feed, it was the result of hours of messy negotiations, hearsay, and petty politics.  If the students only knew what they were paying for, he thought.

The top feed was a document that laid out, in gory and often career-damaging detail, each instance of academic dishonesty, professional misconduct, inter-departmental dispute, and plenty of other violations of laws or university policies.  The name of each involved party, whether accuser or accused, was listed in bold print alongside a description of the incident and various recommendations that the faculty senate or (in severe cases) the board of directors had made for resolving the issue at hand.  A copy was emailed to the president of the university, the ombudsman, the vice president of Internal Affairs (Sascha's superior), and the chief of university police.  Several delicate, high-profile cases had been exposed when details from the top feed had made their way into local newspapers, placing the university in an awkward situation.

Scanning through the list, Sascha spotted a familiar name.  He took a drink from his coffee mug, taking a moment to process the news.

"Jill, did you see that Winter is on the list again?" he said, his eyes still on the computer screen.

"Hadn't noticed.  What's the claim?" Jill asked.

"Espionage and destruction of university property," Sasched said, his voice laced with disbelief.

Sascha heard the steady rhythm of Jill's typing stop abruptly.  He turned to face her.  "This isn't his first time on the feed, but it sounds like it could be his last.  And I'd bet my left nut that he's innocent."

Jill ignored the comment.  "I'll admit that espionage is over the top, even for this group of suits.  Winter's first time was inconclusive, though, and you have to admit that it was fishy.

"Fishy or not, the guy has a good moral compass.  He's innocent."

The rhythmic typing resumed, and Sascha turned back to his computer.  He scrawled a note on his legal pad, downed the last of his coffee, and left the office.

								* * *
								
Jesse swallowed hard, wondering how he had gotten here.  The university's prosecutor had just painted him as a screwoff that, even though he should never have been hired, /most definitely/ should not be allowed to handle anything more than a press release for an obscure bit of undergraduate research.  Jesse had admired the bastard's poise and confidence as he delivered one gently stretched fact after another, building to a rousing conclusion that even Jesse found convincing but didn't agree with reality.  /Never let the facts get in the way/, Jesse thought to himself.

Three weeks earlier, Jesse had just finished writing a report for his superiors that evaluated the success of a new program at CUNY that paired journalism students with investigative reporters from major news organizations and private investigative agencies.  William Tweedie, the current president of CUNY, had made it his personal mission to restore investigative journalism to its rightful place as a tool that feeds the beating heart of democracy.  In addition to the partnership with private agencies, the vision stated, students would conduct an internship overseas in an area of strategic interest to the United States.  Military zones and several unfriendly nations were understandable excluded, but this did nothing to blunt the flood of criticism that the Tweedie, and the university as a whole, endured in the months that followed.

One prominent liberal blog bemoaned the arrogance and gall of a university president who would put his students at risk in foreign political landscapes where the local newspapers were as likely to carry a story about a reporter who had been jailed and executed as a spy as they were to discuss news about the London Stock Exchange.  Conservative hawks were just as eager to pounce, declaring investigative journalism a dead art that shouldn't be put on life support.  The interference of reporters in military and diplomatic affairs, they went on, had a detrimental effect on combat effectiveness and encouraged commanders and politicians to withdraw from public inquiry and debate for fear of being mischaracterized in the press.

Tweedie refused to pull the plug on the program that had made him a household name.  Perhaps by taking public figures to task and embracing modern technology as a means of tracking and eliminating corruption, he argued, the young public can wrest control of its future from the cash-heavy organizations that have a key vested interest in keeping everyone in the dark.  It took a toll on him, which was evidenced most clearly by the private security detail that he hired to keep a few eyes on his family's small house.  His efforts to promote investigative journalism and to encourage students to take up the fight inspired many, especially students in France and elsewhere in the United States, to form impromptu groups of vigilanti reporters.  There were nearly reports of students being arrested or harrassed by police after they had been found trespassing or had been accused of stalking.  While the actions of these student reporters departed from what Tweedie had intended, it told him something important: the will to peel back the layers of secrecy and deceit that crippled the public's understanding of major events, long thought to be lost on the young generations of the 21st century, was alive and well.

At the time of his first trial, Jesse Winter was 29.  A graduate of University of Colorado at Boulder, he held a dual degree in public policy and mathematics, with a concentration in information technology policy.  He had also taken a minor in poetry.  Jesse had served as the president of the campus debate club, had a senior internship at the Electronic Frontier Foundation, and wrote a technology column for a Boulder newspaper.

Jesse joined CUNY fresh out of college.  He had been offered positions at the Independent newspaper in the United Kingdom and a handful of private firms in Germany and the northeastern United States.  Making the decision to stay in the U.S. had been difficult, but the job paid well enough that Jesse could afford to travel and satisfy his interest in foreign politics.  His official title was Special Deputy to the Dean, working in the Office of the Ombudsman.  While this was misleading--his duties frequently involved international travel to undisclosed locations and conferring with individuals that the university could not publicly recognize as partners--much of his work was in the vein of settling disputes.

When William Tweedie had announced his pet program for reinvigorating investigative journalism, he had personally notified Jesse that his job description would be changing.  If he felt comfortable with the plan, then he was welcome to stay.  Jesse was intrigued by the promise of hush-hush work, additional travel, and doing his part to promote an old component of journalism that he had found lacking.  He signed a non-disclosure agreement, agreed to the revised terms, and came to work the next week to find that his desk and computer had been removed.  In its place was a note from the dean and a plane ticket without a marked destination that left from a terminal at JFK that Jesse had  never heard of.  "This should be educational," he said, turning on his heel and heading toward the elevator.

"You're going to eastern Europe," the dean had told him.  "The flight plan has been filed, and your pilot knows what to do.  Here's some background for you to read over while you're over the Atlantic."  He handed Jesse a thin tablet computer whose only marking was a serial number etched into the upper right corner.

"What's the scoop?" Jesse asked, running his hand over the tablet.

"You'll find everything on there," said the dean.  "This project is the president's baby, and you probably know more than I do.  When you land, use the switch on the side of the tablet to securely erase its flash memory.  None of the information on it is illegal, but it's sensitive and we don't want you getting tied up in a rumble with immigration."

"All right," Jesse said.  He noted the location of the secure erasure switch and looked back at the dean.  "Should I pay an extra month's rent, or is this a visit for tea?"

The dean laughed.  "See you in two weeks, son."

					* * *

The flight was uneventful.  The Astra Jets 1125 was cozy and quiet, carrying only three men.  Jesse had met the pilot, Lars Millsen, at a recent faculty meeting.  Jesse's nervous finger tapping caught the attention of Milssen on one occasion, prompting him to whisper something to his first officer.  A moment later, he was sitting next to Jesse in the front row of the cabin and placing a hot drink on each of their tray tables.  Jesse instinctively steadied the mug, but Milssen put him at ease.

"Don't worry, the bottoms are magnetic," Milssen said, gently tipping his own mug from side to side to show the resistance of the magnet.  "One of the perks of these corporate birds."

Jesse smiled and took a drink, trying to ease the knot in his stomach.  Green tea.  One of his favorites.  Jesse had already slogged through the 65 pages of minutiae related to the trip, undoubtedly written by the same smiling administrative assistants who handled all of the university's laundry.  He was bound for Warsaw, Poland, a place that he had visited twice for pleasure.  The third visit would be a tense one.

"Ready for duty?" Milssen asked with a smile.

"Not a chance," Jesse said, "but I'll do what I can.  These kids really stepped in it."

"The bits of news that have made it to my ear haven't been good.  Be careful, and we'll see you Thursday evening.  Give me a call when you're packed up.  We'll be starting our descent in about twenty."

Milssen handed Jesse a black prepaid mobile phone, shook his hand, and walked back to the cockpit.

Jesse knew that tomorrow, Wednesday, would be a tense affair.  He kept his nerves hidden, but his bowels never got the memo.  /I won't be doing this until I retire, unless they let me go at 32/, he thought, looking out the tiny window at the hills of central Europe.

The kids had /really/ stepped in it.  One student, the son of Polish immigrants, had enrolled in the new investigative journalism program at CUNY with a personal mission to prove that his father's failed bid for a seat on the Warsaw City Council in 1986.  The family was convinced that an opponent had floated a rumor about his father's marital fidelity shortly before the election, and that another political rival had abused her position of power at town meetings as a soapbox for making inaccurate statements about the family's intentions.  The student, Sam Rzeznik, had recruited two of his classmates to join him in Poland for a month to investigate the case.

The true story, however, had only come to light after the team of students was arrested by Interpol for breaking into the offices of a local political party and installing recording devices.  The official application for funds had stated that the team would be investigating instances of polling irregularities at a handful of small voting precincts outside of Warsaw.  The low-key nature of the trip, and the delicate issue at hand, had won over the review board and brought praise from William Tweedie.  The team had been awarded a travel and research grant for $8,000, enough to keep them fed and mobile for 30 days.

It was the voice recorder in the men's room that led to their arrest.  The team had installed surveillance equipment in two conference rooms, the office of a senior staffer, and the men's room.  The argument for putting microphones near the toilets was simple: plenty of informal chit-chat and joking happens when people are washing their hands.  If they could soak up the official line from the board room and grab the raw, unalloyed truth from the loo, then the team had an odds-on chance of landing a good story.

A week after the team had installed the equipment and left the building without raising an alarm, a pipe in the ceiling above the men's room had burst, soaking the voice recorder and flooding two rooms.  The office hired a pair of plumbers to repair the pipes and assess the damage.  The workers had no reason to suspect foul play when they found a waterlogged voice recorder, and they threw it into a strap pile with everything else that had shown water damage.  When the office manager came around to get an assessment from the men and learned about the voice recorder, he immediately dialed the chief of police.  After pulling the tapes from the office's own video surveillance system, it took less than an hour to work out what had happened.

Legally, Sam Rzeznik needed a miracle.  The police and Polish immigration authorities were able to match the images on the video tapes to photos on the students' visa paperwork, making the investigation quick and decisive.  If his case went to trial, the prospects were grim.  The evidence against him was devastating.  His academic career would be over for violating the terms of his admission to the investigative journalism program.  Even if he was acquitted, his extended family who still lived in Poland could face harassment, physical threats, or worse.  The other students on the team were in a better position to negotiate a reduced sentence, but they would still face jail time if convicted.

									* * *
									
Jesse tipped his taxi driver and stepped out of the cab onto a busy street in Warsaw.  Opening his umbrella, he looked up at the massive stone columns that flanked the entrance to the historic courthouse.  To his right was a series of shops that included a barber shop, a Greenpeace branch office, and a small coffee shop that had a cupcake protruding from the wall above its traditional-looking iron door.  Checking his impression against the description written on the napkin in his hand, he convinced himself that it was the right venue.  Jesse had no idea what to expect for the next two hours, but William Tweedie had sent him for a reason.

The cafe was busy but had a very European charm.  Jesse unbuttoned his overcoat, walking slowly toward the rear of the cafe as his senses took in the environment.  At the bar that lined the left side of the cafe, a woman was animatedly telling the bartender something about a bracelet that she was wearing.  Next to her, a bearded man in a suit turned to stare at Jesse as he walked past.  A waitress brushed against Jesse's arm, startling him and prompting terse apology.  On the other side of the cafe was a set of staggered four-seat tables, and along the back wall was a pair of high tables with two seats each.  At one of the high tables was a young-looking man with wiry hair pulled back into a ponytail who was reading a newspaper.  He was dressed in an expensive dress shirt and had draped his sport jacket over the back of his chair.  As Jesse approached, the man glanced up from his newspaper and looked him over.

"Jesse Winter?" the man said.  His voice was clear despite an accent that Jesse couldn't place, but the accent didn't concern him as much as the lack of friendliness.  Jesse reminded himself that he shouldn't expect to make friends today.  /Time to go to work/.

"Pleased to meet you.  Mr...?" Jesse said, offering his hand.

The man stood up, shaking Jesse's hand.  "Jasna.  Marcel Jasna.  You can call me Marcel.  May I call you Jesse?"

"Sure," Jesse said.

"Shall we sit, then?" said Marcel, gesturing to the empty chair across the small table.

Jesse pulled off his overcoat, draping it over the back of his chair, and sat down.  Immediately, a waiter was beside him to take his drink preference.  Recalling that Marcel had a glass of scotch on the table, he ordered a gin and tonic.  Stretching his shoulders, Jesse turned back to Marcel and offered a disarming smile.

"Well, Jesse, it's no secret why we come here today," Marcel said simply.  "The manager of my brother's political office caught your students in his building.  It was clear to us that they were spying on our meetings and attempting to sabotage our campaign.  We have video and audio evidence from the night when they entered the building and left recording devices.  This is, as you say, an open and shut case, don't you think?"

Jesse opened his mouth to respond, but Marcel had paused only for rhetorical effect.  He went on: "My brother has no interest in ruining the lives of young people.  But he finds your program that encourages students to become foreign spies for the United States to be distasteful and shameful.  If students go to jail for a time in order to bring the proper shame on your program, very few Polish citizens here will pay attention.  Your program is unpopular in your own country, yes?"

This time Marcel stopped to wait for a response, but Jesse sensed a trap.  "The program enjoys strong support from many quarters," he offered.

"Yes, there are always those who see public servants as corrupt by definition.  Have you noticed that foreign politicians are always corrupt, but your government does not look at its own face in a mirror?"  Marcel took a drink from his glass, then cleared his throat.  "This is not so simple as you think, Jesse.  You are thinking of your students, their future, their families, your president, perhaps your job.  We understand that it is far deeper.  Each time our politicians are painted as corrupt, Poland suffers.  We have suffered under the heel of such injustice for most of a century."

Jesse shifted in his seat.  "Marcel, these students lied to us.  I understand your position on spying, and I know that from your vantage point we seem to be training a new crop of young spies.  You need to know that these students were not authorized to conduct themselves in this way.  They lied on the application, and they lied to the review committee."

"This does not change substance of the situation."

"That's correct.  But it changes the details of the situation that you and I care about.  We're not here to debate state philosophies on espionage.  This is about the welfare of a group of kids who, while they showed themselves to be overzealous, have passion and talent that /must/ be put to productive uses.  Do you see?"

Marcel responded immediately.  "With respect, the fate of the students is beside the point.  This is an issue of national and political sovereignty, and we intend to treat it as such.  I must admit that I do not think we will reach a settlement.  We must not appear weak in the face of interference from America, even if the face is that of a young spy ring."

The waiter returned with Jesse's drink, taking the unspoken hint that the men were not interested in ordering a meal.

"Our university is prepared to work with you, sir," Jesse continued.  We have no interest in inflaming the relationship between the U.S. and a friendly nation such as Poland.  We know that despite your feelings about espionage, it would not bode well for your campaign to be associated with the arrest of a Polish student."  Turning his palms upward in a conciliatory gesture, Jesse added, "What options remain on the table?"

Marcel leaned back in his seat and tilted his head to the right, clearly wondering what Jesse had been in Jesse's mind as he spoke those words.  It was a moment before he spoke again.  "You offer us money, Jesse?"

"We are prepared to cover the costs of the repairs to your brother's office, as well as any legal costs that you've absorbed.  Beyond that, well... this is a negotiation."

"Please excuse me for a moment, Jesse," Marcel said, rising to his feet and walking toward the back door of the cafe.

/Time for a conference during the powwow/, Jesse thought to himself.  He took the moment to enjoy a taste of the gin and tonic in front of him that had so far been ignored.  As the strong flavor filled his senses, Jesse talked himself through the next few minutes of his conversation.  The man in the other chair suspected that Jesse was out of chips in the negotiation.  Jesse was certain of that.  Jesse also knew that his own case was weak and that offering money for repairs and cooperation was small potatoes to these guys.  Someone who cares about national and political sovereignty wasn't going to let a group of foreign kids with an agenda go without making a point or making some demands to make the American politians squirm.  And Jesse was just a pawn in all of this.  He knew that Marcel felt no personal sympathy for him.

As long as the guy didn't pull a gun, Jesse could handle this.  Several years of dealing with stuffy, arrogant university administrators had given him a thick skin.  He took another drink.  The cafe had grown louder as the lunch crowd began to assemble.  Still, the footsteps on the hardwood floor were distinct.  The conference was over.

"My apologies, Jesse," Marcel said as he took his seat.  "My brother can be very detailed."

"Not to worry.  How is he doing?"

"Very busy these days.  This issue with the students has taken a lot of his time, and he is eager to wash his hands of them."

/Didn't expect that/, Jesse thought.  /But he's not finished/.

"However," Marcel said, leaning back in his seat and slowly turning his glass, "a detail has come to his attention today.  One of the students, Sam Rzeznik.  You know of him?"

Jesse nodded.

"Yes.  It seems that his family has a history.  A history that complicates things for us.  What do you know about this?"

"I know that Sam is of Polish descent.  He mentioned in his interview with our review board that his grandfather was a public servant, but no serious problems came up during the proceedings.  Everything that Sam told us was verified before he was allowed to start his research project."

That much was true, but Jesse wasn't revealing his hand just yet.  The political problems had not involved Sam's grandfather, who had been involved in municipal government for most of his life.  Sam's father, on the other hand, had found himself the victim of circumstances entirely beyond his control.  The review board that CUNY had convened to consider the applications submitted by the investigate journalism students had learned about Jesse's father's political past after conversing with Polish authorities, but the board had given its stamp of approval after reviewing the facts.

"Yes, his grandfather served in the city government with my grandfather.  Those were very good days for Poland, despite the problems of the world.  It was the next generation that began the divisive era of our politics that we still see today.  The dishonesty.  The posturing."  Marcel paused, looking pointedly at Jesse.  "And the /spying/.  It is this generation that Sam's father helped to assemble.

"My father in his third year on the city council when Sam's father declared himself a candidate for the same seat.  My father had worked for ten years to clean up this city, and to restore pride to Warsaw citizens.  He had done every task to match the letter of the law.  His contituents trusted him.  Sent him letters of thanks each year.  He knew the spirit of the people, and knew how to bring money and business to Warsaw.  He understood the real purpose of a public life."

Marcel took a deep breath, clearly incensed by the memories that he was bringing to mind.  "When Sam's father declared himself a candidate, he... he knew nothing of politics.  He knew nothing of campaigns.  His only public achievement was to manage the local library, and poorly at that.  The library was never lower on funds.  He began to attend city council meetings, asking impossible questions.  There was neven a more irritating man.  He knew nothing of the business of government, but he made it his mission to lecture everyone on what he called the 'proper order of the gathering'.

"It was known that he saw visited women other than his wife.  After he declared his candidacy, this secret became known to the public.  It is not clear how this information became known.  He never recovered from the stain on his reputation, and he lost the election.  It was soon after that his family left the country, and we see now that he reaches through his son in an attempt to blame us for his past mistakes."

Marcel paused to finish his scotch.  "There will be no deal, Jesse."

"Are you familiar with the phrase 'catch-22', Marcel?"

"Yes, of course.  It is a situation in which every available course of action carries negative consequences.  Why do you ask this?"

"Since you are familiar with your father's political story, you might remember an incident involving an old woman in June of 1986."

Marcel's face clouded, and he shifted in his seat.  "Yes," he allowed, "I am familiar."

"If I told you that we had an voice record of that conversation, would this change the substance of our discussion?"

Jesse could sense the man's tension, and he didn't admire the mental quandary that he was inflicting on someone he had just met.  He probably believed honestly that the popular account of his father's involvement in the story about the old woman was a deliberate smear.  It was also clear that he had not expected Jesse to have the ace, much less to throw it now.

Marcel considered Jesse for a long moment.  "You attempt to deceive me."

"It is not your claims, sir, but the way that you abuse the trust of the people with one hand while smoothing their feathers with the other that make me doubt you," Jesse said, making it clear that he was uttering a quotation.  She spoke with an east Warsaw accent.

Marcel stared at him.  He would have remembered that decades-old quote.  Just when Jesse was sure that he had collected his thoughts and had worked out a stinging train of logic, he took a deep breath and pulled a cash clip from his pocket.  He placed three Euro bills on the table before replacing the clip in his trowsers.  Then he met Jesse's eyes.

"Your students will meet you at the airport tomorrow, Mr. Winter."

Marcel was out of his seat and halfway to the door of the cafe before Jesse fully processed what had happened.  The students were absolved of their legal worries.  An impossible situation had been turned on its head.  But there was no way that this was the end of the story.  These guys didn't quit that easily.  Jesse's trump card had ended the contest, but he knew that he would hear from Marcel again before he retired.  /Unless I retire at 32/.


The next day at the airport, Jesse stepped out of his taxi and was ambushed by a phalanx of relieved CUNY students.  It was obvious that they had all been crying.  Jesse reassured them that the university was concerned about their behavior but had committed to do everything it its power to protect them from unfair litigation.  That would be much easier once they were home, and Jesse wasted no time in moving them through a series of security checkpoints and onto the private jet.

The students were asleep before the jet had taxied to the runway.  Lars and his first officer looked over the exhausted students before nodding to Jesse and settling into the cockpit.  The jet would need to refuel in Iceland, he was told, and they would be home by morning.  Then the real questions would begin.  The review board would be reconvened to look for any clues that someone knew more about Sam Rzeznik's family than what was discussed.  The press would be there to get the scoop on the students and take their stories.  The FBI and the State Department would want to know everything.

Jesse had a feeling that this would be the best sleep that he would have for a few days.

					* * *

In the nine months that followed the Warsaw trip, Jesse heard little about the case once the furor of the press moved on.  The much-anticipated renewed public backlash against the investigative journalism program never arrived.  The CUNY administration was visibly relieved, but they all had to wonder when the other shoe might fall.  Were Marcel and his brother working behind the scenes to put together a counter-spying team?  Was anyone from CUNY already being followed?  It seemed prudent to be watchful and to keep an eye on the brothers' campaign in Poland, but no one wanted to discuss the matter, much less do anything about it.

One Thursday morning, a uniformed university police officer knocked on the door of Jesse's office.

"Mr. Jesse Winters?" the officer asked, glancing at a sheet of paper in his hand and looking back at Jesse.

"Yes?" Jesse replied.

"I'm detective Morse from the CUNY police department, I'd like to ask you a few questions, if that's all right."

"Certainly.  Please come in."  /This should be good/, Jesse thought, feeling his pulse quicken.

The officer stood near the door, refusing Jesse's offer of a chair.  "Mr. Winter, one of the members of the board of directors has launched an investigation into your involvement in the Warsaw incident from June.  Are you familiar with that case?"

"Yes, I traveled to Poland in June to negotiate on behalf of the university.  What's this about?"

"Would you mind coming down to the station to talk about it?  We can be sure of our privacy there."

"Am I under arrest, detective?"

"No, sir.  We just want you to discuss a few things with us."

"This office is private.  Why not stay here?"  Jesse felt his old privacy hackles starting to rise.  /My office is certainly more private than an interrogation room/.

"With all due respect, Jesse, given your involvement in this case you should know better than to assume something like that."


Jesse's visit to the CUNY police department was short and terrifying.  He learned that one of the board members, Stewart Rheingold, was investigating him for professional misconduct related to the Warsaw negotiations.  It was improper and against the law, the claim went, that Jesse made implicit legal threats against foreign nationals in order to secure the release of students who had broken Polish law.  Rheingold had been a vocal opponent of the investigative journalism program from the outset, and his position on the board of directors made him privy to the internal operations of the program, including Jesse's job description and the details of his work.

Still, Rheingold was not in the loop regarding what happened in Warsaw.  Only Jesse's immediate superior, the Dean, had known what transpired.  The Dean was a trusted associate who knew the high stakes for that trip.  There had been other instances when reporters had been circling like sharks, looking for a juicy story of undercover deals that they could turn into an exclusive.  The Dean had been gracious but firm, refusing to share details or to implicate Jesse.  The information had leaked another way, but it would be days before Jesse understood how.

The detective was as helpful as he could have been without breaching protocol.  He knew that Rheingold had a personal axe to grind.  When Jesse asked what the next few weeks would bring as the case proceeded, he was told that the department wouldn't continue with the investigation unless Rheingold could produce more evidence than a hunch.  Detective Morse couldn't share the details of the charges that had been filed, but he made it clear that this would be an uphill battle for the plaintiff.


When the trial dates were finally set, the details of the case against Jesse were foggier than ever.  Due to the confidential nature of the case, the evidence submitted by the university's prosecutor was sealed in an envelope for the judge to view privately with both attorneys.  Jesse had hired an attorney from a local firm that specialized in privacy and first amendment litigation.  At 50, the guy had been around the block more than once, and he seemed the type to be cool under pressure.

The case against Jesse was surprisingly weak, but the university prosecutor had constructed an elaborate argument that had a chance of succeeding.  If he could convince the judge that Jesse had issued a threat, even an implied one, then Jesse could be found guilty of obstruction of justice.  His attorney seemed confident that the judge would reject the argument, but its plausibility made Jesse nervous.  If the key question was whether Jesse issued a threat, then he was in a difficult position.  He /had/ issued a threat, although it was not spoken.  Anyone who had been privy to the conversation in Warsaw would have understood that what Jesse said was a threat.  But the thing gnawing at Jesse's mind was how that information had made its way to the ear of one of the board members.  One of two things was true.  Either Rheingold had secured a copy of the documents on the tablet computer that Jesse had carried to Warsaw, or Marcel and his brother had made a contact in the United States.  The latter worried Jesse on a much deeper level than the former.


On the final day of the trial, Jesse sat alone while his attorney consulted with the judge and the prosecutor.  He found himself thinking of his parents, wondering what they were doing at that instant.  He had told them about the legal issues but reassured them that it was nothing serious.  The case had not gone to the papers.  They were probably enjoying a nice breakfast at home.  His mother would be heading outside to weed the garden, and his father would be walking his collie or preparing a lecture for his classes.  They wouldn't be pleased about the predicament that Jesse was in, but they would understand and support his decisions.  He knew that much.

When the judge came out to issue the verdict, Jesse's attorney gave him a reassuring smile.

"Will the defendant please rise?" the judge drawled.

Jesse stood up from his wooden chair, straightening his jacket.

"In the case of Rheingold versus Winter, we find the defendant not guilty on the charges of obstruction of justice.  Mr. Winter, you are free to leave."  After a brief pause, the judge cleared his throat and added: "Off the record, I think that this case was a poor use of the court's time.  Good day, gentlemen."

Amid some scattered applause and hushed chatter from those around him, Jesse slowly processed the verdict.  He was cleared of wrongdoing.  But the gnawing feeling that something was left unresolved did not budge.  The nature of the trial evidence was sketchy, and the judge seemed reluctant to broach the topic during a public hearing.  This had been Jesse's first experience of a trial, but he judged from the body language of the attorneys and the judge himself that the more experienced men found the proceedings unusual.

After a final debriefing with his attorney, Jesse left the courthouse and walked toward his car.  As he fumbled in his pocket for his keys, his head spinning and reliving moments from the trial, he felt someone sock him on the shoulder.  Instinctively stepping away from the man who had appeared beside him, Jesse slowly recognized a familiar face.

"Good show, Jesse," Sascha Greene said with a wink.

"Hey, Sascha."  Jesse smiled and heaved a sigh of relief.  "Were you in there?  I didn't see you."

"Yeah, I wanted to see what the university had up its sleeve.  I guess we saw that logical gymnastics is their main game.  How do you feel about it?"

Jesse shook his head.  "I don't know.  Obviously, I'm happy that they didn't end my career.  I'm happy that the judge had a sense of humor.  I learned a lot about the politics of our board of directors."

"But...?"

"Well, something doesn't add up.  Rheingold shouldn't have known anything about what happened in Warsaw.  All the press knows is that the university negotiated a deal and kept the students out of hot water.  And that was the official story from the university.  Haw did he know?"

Sascha nodded, looking over his shoulder for a moment.  "Let's get some lunch.  My treat"

Jesse shrugged.  "All right.  I'm starving."


"You look like you haven't slept since Warsaw," Sascha said, dipping a shrimp in his cup of ranch salad dressing.

"You'd be surprised how close you are to the truth."

"Have you ever met Stewart Rheingold?"

"I've seen him at a few board meetings.  He likes to talk."

"Unfortunately for you, he likes to act too.  The guy is a modern day Joe McCarthy.  His big issue is national autonomy.  If anyone mentions the global economy, he starts frothing at the mouth.  I saw him take a senior professor to task because of the tie he was wearing.  Something about the supply chain that was involved and the China's currency tactics.  Anyway, it's probably nothing personal.  He saw an opportunity to teach us all a lesson and make an example of the school that's supposed to be under his supervision, and he pounced.

The part that really has you wondering, thought, is how he sweet talked his way into a copy of the materials that you and your boss thought were private."

Jesse looked up from his plate, studying Sascha's face.  "That's right.  Although that's only one possibility.  The guys in Warsaw can't be happy about how things went down.  They might have tipped off Rheingold about what was said."

"It's possible," Sascha said, "but let me tell you something.  Yesterday when I was leaving work, I overheard an interesting conversation in the elevator.  One of the administrative assistants for the Department of Financial Aid was telling a her co-worker about a document that she had retyped and converted into digitally signed and encrypted format.  Not too many documents get that treatment, so it piqued my interest.  She mentioned that the topic was a trip to eastern Europe.  She also knew -- and this is the key -- that the university was planning to exercise an unorthodox strategy to secure the cooperation of the Polish guys.  Something about the students' parents.  She didn't give details, but it sounded complicated and legally questionable.

"That much by itself isn't interesting.  The administrative assistants know a hell of a lot about what happens on campus, but everyone knows that.  Here's the rub, Jesse.  The woman in the elevator is good friends with Rheingold's wife.  They vacation together each year.  When the paperwork for your trip got processed and the students were released, it would have been easy for the administrative assistants to put the pieces together.  The trip could have come up in casual conversation, and Rheingold would have been very keen to hear the details."

Jesse nodded, letting the new information percolate.  "We'd better hope that there are no more lawsuits like mine.  Eventually someone will find a team of lawyers that can bring down the whole program.  They made a reasonable against me, and I feel like the judge let me go on a technicality."

"Hardly," Sascha said though a bite of his sandwich.  "The judge made a reasoned decision.  Any evidence that you made a threat would be hearsay or circumstantial.  He couldn't convict you on that.  Remember, Jesse, you did the right thing.  Those guys picked their strategy twenty years ago, and you know how it turned out.  They've been in damage control mode ever since."

"Thanks Sascha, but the fact that they're in damage control isn't comforting.  What if they still see us as a threat?  Taking me down could be revenge, or it could be a simple matter of neutralizing a threat."

"If they wanted to neutralize the threat, they would have gone after your bosses.  You're a talented guy, Jesse, and the university will miss you when you move on.  But in a more global sense, and to these guys in Poland, you're just another guy, and CUNY could replace you in an instant.  It's the program, and the fact that we admitted a bunch of students whose families have a policital axe to grind, that are the real threat.  I hear you, man, but I think your worry is misplaced."

"It's good to hear that from someone I trust.  I've been trying to convince myself that I was just being paranoid.  The trial has completely consumed me lately, and I think it's made me a little unbalanced.  Getting back to work will do me a world of good."

"Get to it, then," Sascha said, finishing his last bite of pickle and dropping a few bills on the table.  Let me know if you need anything.  And I mean /anything/."  He gave Jesse a significant look.

Jesse nodded.  "If I hear footsteps, you know I'll be in touch."

										* * *
										
Sascha closed the office door behind him and headed toward the north elevator.  He jabbed the call button and took a moment to look out the window.  Outside, a group of students clad in bright red and yellow shirts were playing a game of touch football.  The sun was shining, giving the false impression that it was a balmy summer day.  /Jesse is probably wishing that it were summer again/, Sascha thought.  Could Jesse have been right?  He couldn't imagine that the university would try him again for something related to his trip to Poland, especially after the last trial.

Espionage and destruction of property.  Two loosely defined but very serious charges for a young professional.  Sascha knew that the university was capable of eating its own.  That had always bothered him.  Some schools had a superb legal posture and would go to bat for members of their communities to the greatest extent that they could.  But not CUNY.  When students got in trouble for sharing files on their computers, the university would forward the Cease and Desist notice directly to the student who was using the network address that was implicated in the document.  This was poor form for the university, since a student could be falsely accused if someone happened to be the victim of a virus or worm, or if someone had used a weak password to protect his wireless network.  Guilt, intent, and liability were very complex matters to sort out in a place as technologically dense and active as a university campus.

In this case, though, Sascha thought that it was bad business for CUNY not to step up and defend Jesse.  He had been cleared of wrongdoing in the Warsaw trial.  It had become clear afterward that William Rheingold was motivated by political and personal concerns, and not because of any moral obligation.  Jesse had dealt with many other concerns in the meantime, but Sascha hadn't heard any more gossip from the press or the ladies upstairs.

The elevator arrived, and Sascha stepped in.  He punched the button for the fifth floor.  /Jesse will be in, and he needs to hear about this before the police come and start asking questions again/, Sascha thought.  He heard the chime indicating arrival at the fifth floor, stepped out, and froze.  The door to Jesse's unmarked office, which sounds an alarm if it remains open for more than one minute, stood ajar.  The lights were off, an odd thing at 10:30 on a Monday morning.

Sascha approached the office slowly, pulling his phone from his shirt pocket.  He leaned to his right, looking around the door.  He could only see the corner of the office, making him wish that the door had been reversed.  Sascha had been a U.S. Marine before college, and his tactical disadvantage on this side of the door made him uncomfortable.  Backing away and glancing at his phone, he dialed Jill's number.

"Jill Cumberland," the pleasant voice said.

"Jill, it's Sascha.  Is there maintenance in 503B scheduled for today?"

"503 beta?" she asked, sounding surprised.  "Isn't that the investigative journalism war room?"

"Yeah," Sascha replied, glancing down the hallway.  "Anything on the grid?"

"Two seconds here," Jill said.  Sascha could hear her fingers working her keyboard.  There was a long pause.  Then she spoke again, this time with slight concern.  "This is odd, Sascha.  There is nothing on the schedule, but someone added a ticket for 503B and then deleted it.  Is everything all right?  Do you want me to call security?"

"I'm not sure.  Don't call security just yet.  The door is ajar, and the lights are off.  I'm going to have a look around and make sure that everything is kosher."

"All right.  Be careful, Sascha.  I'll send someone down there if I don't hear from you in a few minutes."

"Give me ten," Sascha said, and quietly flipped his phone closed.  His pulse had quickened.  The information that Jill had given him, thought it probably seemed suspicous, would seem ultimately inconsequential to anyone else.  To Sascha, though, the news was unwelcome.  He knew the scheduling system very well, and he knew his building.  Someone had scheduled maintenance for the room where Jesse normally worked, and someone -- probably the same someone -- had cancelled the service ticket.  By itself, that could be attributed to a mistake by the person scheduling the ticket.  With the door ajar on a secured office that was normally occupied during the day, the situation changed.

The CUNY ticket scheduling system was complex once one looked behind the pleasant computer interface that the maintenance workers and administrative assistants used.  It synchronized its database with servers that handled service requests for the university at large, enabling reassignment of jobs when one maintenance crew was overloaded, as well as distributing the work of creating tickets.  Each building had to be inspected by the local fire chief at least once every three years, for example, and the university as a whole had to be fully reinspected every five years.  Requiring the staff of each building to coordinate with the fire chief would be time consuming for all parties, and so the administrative staff for CUNY negotiated the inspections for the entire campus.

Like most useful digital tools, the scheduling system cut both ways.  The flexibility and division of labor that was won by using distributed ticket creation also meant that tracking an errant ticket was nearly impossible.  There were hundreds of workstations across the campus, and many faculty and staff left their computers turned on and signed in when they left for the day.  The two minutes of waiting that they saved each morning left many sensitive resources at the mercy of anyone who didn't mind picking a lock, breaking a window, or scaling the wall of a building.  The deleted ticket would include the name of the person whose computer was used to create it, but that might be useless for tracking the real identity of the person at the keyboard.

There was one other thing on Sascha's mind as he slipped the phone back into his pocket.  The scheduling system was designed to interface with the door security system.  This was a convenience feature for maintenance workers who would become annoyed when the office that filed a maintenance ticket was protected by an elaborate security system that would reliably go off during their visit, leading to an automatic call to the police.  In order to save the university a few thousand dollars in police fines and lost productivity, the board of directors voted narrowly to install an override system and, regrettably, to give the scheduling system the ability to trigger it.

The political details of the override aside -- it had been a very messy, heated discussion on campus -- there was also the not-so-small matter of a bug in the scheduling system.  CUNY had hired a private software firm to develop the software for the scheduling system.  When it was decided that the scheduling system should interface with the door security system, the same firm had made the changes.  But since the change was made as an afterthought, five years after the software was developed, it was difficult for anyone to recall the picture of its design.  There were small details that were neglected, and periodically a consequence of those details would crop up.  For their part, the software firm had been helpful in troubleshooting and resolving difficulties as they arose.  This one, however, had gone unresolved for months.

When a new maintenance ticket was created, the ticket system send a secure message to the door security system that marked a room, floor, or building as a maintenance zone for a period of time -- normally the length of time specified on the ticket.  When the ticket expired, security was restored to its normal operation.  Several months back, however, Sascha had noticed a glitch.  After hours on a Thursday night, he had gone to the second floor to pull a file on a student who was under investigation for fabricating evidence for a research project.  This involved going into the records vault, which is a large space that is normally well-protected by security cameras, doors with card-key access and retinal scanners, and black-out times when the doors would not open.  On this Thursday night, Sascha found the door unlocked.  After calling security and waiting for an agent to arrive, he checked the vault and verified that everything was in order.  It didn't appear that anyone had been in the room that night.

After saying good night to the security detail, Sascha had gone back to his office to check the ticket system.  Maybe someone scheduled maintenance and forgotten to follow through, or maybe it was scheduled for the wrong room.  When Sascha pulled up the ticket system and checked the record for the vault, his blood ran cold.  There had been a ticket created for the vault so that a technician could install an upgraded ethernet switch as part of a campus-wide migration to a faster network architecture.  The next line in the record, though, showed that the ticket had been deleted several days later -- 2:18 p.m. on the same day that Sascha had gone to the vault.  The reason given for the deletion appeared to be that the technician had called in sick, and there was no one else available to do the work.  /If the ticket gets cancelled, the door security system still opens the room/, Sascha had thought, his mind reeling at the thought.  /This is almost as bad as the power to the raptor cages going out in /Jurassic Park//...

When Jill told him that the schedule showed a ticket that was created and immediately deleted, Sascha felt his heart start to hammer.  Seeing the door ajar was enough to activate his old Marine instincts.  Having backed away from the door to make his call, he now crept back toward it, a small canister of pepper spray in his left hand.  The only sound coming from the room as he moved silently along the wall was a rhythmic rustling of paper and the sound of an old dot-matrix printer.  Sascha reached the door and began slicing the pie, an old technique that he had perfected in basic training.  When a team of Marines needed to silently clear a room, they began with one man on each side of the door, guns drawn and shouldered.  This gave each man a sliver of the room to examine.  Next, they would turn slowly away from one another, increasing the portion of the room that they could see.  When the entire room had been examined from this position, they would signal to their comrades that it was safe to proceed.  Today, though, Sascha had to rely on his own training.  The wall gave him an advantage and a disadvantage.  The bulk of the office would be to his right, making it easy to slice the pie.  However, he would need to open the door in order to see the room.  If its hinges squeaked, he would be partially exposed, and the walls were not designed to stop or deflect gunfire.  /It's just an office/, he reminded himself.  This is a university in the U.S. of A.  With his right foot resting against the door frame, Sascha peered around the edge of the door.  There was an empty desk against the wall on his left, and a bookshelf on the far wall.  Above the desk was a steel matrix of time cards and pamphlets.  In the corner was a black, non-descript waste bin.

Sascha took a deep but quiet breath and tugged on the door.  It was a heavy door, but it gave way slightly at Sascha's touch.  No squeaks.  He still couldn't fit his body between the door and its frame, but pulling it open had given him a better view of the room.  It was much narrower and deeper than he expected.  There was another desk on the right wall, and there appeared to be a stack of dark coats draped over the office chair next to it.  Near the back of the chamber, Sascha saw a computer workstation with four monitors attached to it.  The computer was running, its screens aglow, but no one was using it.  /Odd/, Sascha thought and he scanned the room for movement.

Seeing nothing suspicious or out of place, Sascha opened the door enough that he could slip into the room.  When he did, he stopped short.  What he thought was a pile of coats was now clearly a man, and he was slumped over onto the desk in front of him.  Sascha approached closely, not immediately recognizing the man.  There was no sign of a struggle.  As Sascha reached out to put his hand on the man's shoulder, the dark mass inhaled sharply and sat up, turning his head toward Sascha.  Instinct taking over once again, Sascha backpedalled, raising his pepper spray into his line of vision.  "Who are you?" he ordered.

The man stared at him, clearly confused.  The low light on the room made him appear pale and green, almost alien-like.  "Sascha?" the man said, his voice rough from sleep.

Sascha found a light switch beside him and flipped it on, his pepper spray still aimed at the other man.  As the lights came on and his senses sharpened, Sascha dropped his pepper spray and rushed forward.  "Jesse.  Are you all right?  Did someone break in?"

"Are you kidding?  I was taking a nap, man.  I've worked three nights straight.  Are /you/ all right?  You look fit to kill."

Sascha exhaled, allowing himself a nervous chuckle.  "Boy, Jesse, do I have a story to tell you.  First, though, I'm going to lock the door."

Jesse frowned, looking past Sascha at the heavy steel door that stood open.  He rubbed his eyes, his brain not up to speed just yet.  "It was already locked.  It's always locked."

"Not this time.  I think I know why, too.  Grab your coat.  Let's take a quick walk."

Sascha pulled out his phone, dialled, and told Jill that everything was cool, at least for now.  Then he walked over to the computer terminal, loaded the program that controlled the security for the room, and pushed "Reset".  Immediately, he heard a click and a dull thud from the far end of the office as the door lock engaged.


Jesse felt like his legs were made of cement.  Sascha was the fitter man of the two of them, and he was walking at a pace that reflected the speed of his brain at that moment.  They were walking between a line of tall hedges that led from the building where they both worked to the CUNY cafeteria.  "We're not going for lunch," Sascha said.  "We're just going to find a place that's noisy and can give us same privacy."

Jesse puffed his cheeks and willed his legs to keep up.  The long hours at the office had been taking a toll on his fitness, but this was simply unnecessary.  "Sascha, why do we need privacy?  Was someone offended that I was asleep at my desk?"

Sascha stopped suddenly, giving Jesse a hard look.  "Check your pockets.  And your collar."

Jesse looked momentarily puzzled.  "Wha-" he began.  Then understanding appeared on his face.  Jesse checked his breast pocket, then his pants pockets.  Nothing.  He turned up his collar, running a thumb along the crease.  As he passed the back of his neck, his thumb caught something hard and plastic.  He glanced up at Sascha, his eyes wide.

"Show me", Sascha whispered.

Jesse turned around, flipping his collar open so that Sascha could see the device.  Sascha put his face very close to Jesse's back, trying to work out how to remove the bug without breaking it.  After a moment, he put his mouth next to Jesse's ear.

"Put your collar down and follow me," he whispered.

The two men walked into the student cafeteria.  They looked distinctly out of place in their jackets and slacks, but the students paid them no attention.  Sascha gestured toward the kitchen.  Jesse followed him through the swinging doors.

"May I take your jacket, sir?" Sascha said, putting on his best impression of a waiter.

"Certainly."

Sascha took the jacket from Jesse and turned on the sink.  He opened the collar, exposing the bug, and quickly submerged it in the water that had collected.  After the water had time to soak into the bug, he pried it from the fabric.  Sascha shook his head, turned the sink off, and handed the jacket back to Jesse.  Then he yanked the stopper out of the drain, sending the bug into the bowels of the university.

"After you, good sir," Sascha said with a self-satisfied grin.

Jesse allowed himself to laugh before giving Sascha a fist bump and walking back into the cafeteria.

"All right," Jesse said, settling into a booth.  "I'm awake now.  What the hell is going on?  I was asleep, and then I was bugged, and now I'm in the student cafeteria at a table with a short leg."

"I know, it's been a whirlwind.  This won't make sense unless I give you some background."

Sascha told Jesse about the problem with the vault's security.  He went into detail about the scheduling system.  As soon as he mentioned that the security doors unlocked despite the fact that the event had been cancelled, Jesse groaned and hung his head.  Jesse had debugged enough code in his life to recognize a programming error when he saw one.  This wasn't the type of error that would crash the system, but in some ways it was worse than that.  This was a logic error.  The programmers who had designed the module that was responsible for informing the security system that a ticket had been cancelled had neglected to wire up that bit of code properly.

"How long have you known about this bug?" Jesse asked.

"Months.  I reported it the day that the vault was unlocked, but nobody seems to know when it will be fixed.  Some of the programmers are on vacation, our people are busy with other things, and it only happens on rare occasions.  You know the drill."

"Yeah.  I bet it would get done faster if the president's office had been the one unlocked.  Who else knows about this?"

"I reported it to the director of I.T.  I have to imagine that his deputies are aware of it, but I don't know."

"Those guys are all underpaid for what they do, including the director.  It makes me nervous under the best of circumstances.  And now they have a huge security hole at their disposal until it gets fixed.  It's a ripe situation for someone to steal records, modify personnel files, or--"

"To bug a staff member who works on high-profile cases."

"I can't figure it out.  Who would want to spy on me?  We don't have any cases in the queue right now.  The most that they could learn is what the students in Brazil are learning about how the fish markets influence fuel production."

Jesse's thoughts turned back to the morning.  He had been doing paperwork for the students in Brazil and Venezuela when he fell asleep.  At that time, the door had been locked.  No one else had been around all morning.  Whoever came into the office had free reign.  Jesse had looked through the office after Sascha woke him up, and nothing looked out of place.  /Neither did the bug in my collar, though/, he thought.

"I wish I knew, Jesse.  At the very least, we need to look at the security tapes and figure out who was in your office.  In the meantime, let me make a suggestion.  Grab your papers from 503B and find another office for the week.  I'll take a look at the room assignment chart and see what's available."

Jesse sighed, looking out the window at the growing shadows on the lawn.

				
Sascha sat next to the security officer, staring at an array of computer monitors.  They had trawled through sixteen hours of closed-circuit TV footage, trying to find evidence that someone had entered Jesse's office.  Luckily for the hunters, there was little traffic on the fifth floor.  Unfortunately, their prey had eluded them up to this point.

"There!" Sascha called out, pointing a finger at labeled with the numeral 8.  "Who is that?"

Jesse stood behind the others, leaning forward and squinting at the grainy screen.  "Oh, that's Tammy from upstairs.  She brings me the documentation on new student cases.  I didn't see her today, though.  What's the timestamp?"

The security officer pushed a few buttons.  "Eleven fifty, sir.  Just before the top floor goes to lunch."

"Just after I fell asleep," Jesse said, sounding sheepish.  Then he suddenly perked up.  "But she would have noticed that the door security was turned off."

"Looks like she keyed in her passcode and opened the door," the officer said.  He rolled the tape forward by a few seconds.  "For a lot of folks it's a rote process.  They know that the numbers open the door, and the details are someone else's problem.  That door should have closed itself when she left, though."

Jesse had a brief flashback to the day when the heavy steel door was installed.  The technicians had pulled three large magnets out of a crate, mounting them inside the door frame.  They had spent several hours installing a new wiring block and upgrading the power supply to the office to support the new equipment.  "Don't pinch your fingers in this one," one of the workers had said with a nasty laugh.

"It's magnetic," Jesse said.  "And I'd bet you anything that it's tied into the security system."

The officer turned to face him, an odd expession on his face.  "They installed a magnetic door for a staff office?  Those are normally reserved for vaults or emergency weapons stores, things that need protection from determined criminals.  A heavy spring would have done just fine."

"I questioned the wisdom at the time, but the higher-ups were convinced that the extra protection was worthwhile," Jesse said.

Sascha heaved a sigh.  "So we know that Tammy was in the office to deliver paperwork and that the security door would have let her in without a passcode.  How long did she stay?"

The security officer rolled the tape forward again.  "Thirteen seconds, sir."

"Barely long enough to install any tracking equipment.  There's no way that she bugged Jesse."  Sascha turned to face Jesse.  "All right, let's think about this.  Did you leave your coat unattended anywhere in the past couple of days?"

Jesse looked toward the ceiling, rolling his mental tape backward and forward.  He had gone on a disastrous date on the previous Friday.  /Never go on a date when you haven't slept in 50 hours/, he concluded.  Over the weekend he had gone to his favorite coffee shop in Queens and had slept through a B-list film that was showing at a local theatre.  He took the bus to Queens each weekday morning and then walked to CUNY.  On this particular morning, he had purchased a copy of the /New York Times/ and read it on the bus.  The bus was crowded and stuffy, so he had taken off his jacket and, out of habit, draped it over his seat.  The newspaper had occupied him until his stop, leaving plenty of time for someone to mess with his jacket.  The collar would have been fully exposed, and the bustle of passengers always left him numb to small amounts of jostling.  "It must have been on the bus this morning," he said finally.  "I was reading about the Fed's latest argument for quantitative easing, and I had thrown my coat over the back of the seat.  There would have been ample time for someone to bug me."

"And ample time for the person to get away unnoticed," Sascha added, sounding momentarily defeated.  "This doesn't bode well.  I had hoped that we would find someone local, someone affiliated with CUNY.  This makes it a much bigger problem, potentially."

"Mr. Winter," the security officer said, "if you were indeed bugged on the bus, you will need to change your habits.  Tracking someone without being noticed is not an easy business, and whoever did this has probably been watching you for a while.  You don't need to move house unless you suspect that your safety is at risk, but I suggest that you drive to work for a while, or maybe take a cab.  If you routinely go for walks in your neighborhood, be sure to change up your routes."

"I think I need to move house," Jesse said.  He was suddenly feeling ill.


When they left the police station, it was full dark.  Nearly nine hours had passed since Sascha found Jesse asleep at his desk, his soft snoring being broadcast somewhere by an electronic bug on his collar.  It was unclear who had installed the bug, but the two friends had some idea of when, and where, it had been installed.  Unfortunately, there was no video surveillance on city buses, and Jesse had little recollection of anything from the bus ride.  Not enough to go on, as the detectives would say.

Sascha had offered to give Jesse a ride home, saving him the anguish of riding the bus again.  They climbed into Sascha's Volkswagon and made their way out of the parking structure.

"There was one other thing, Jesse.  Not that you need anything else to think about."

"What's that?"

Sascha turned on his high-beam lights as he turned onto at empty street.  "Your name came up on the top feed today.  That's why I came down to your office this morning."

Jesse looked at Sascha, searching his face for clues about the news that he was about to hear.  "Again?!  We haven't even processed any cases lately.  How can there be a case against me?"

"It's still early.  The case is being pursued by the Office of the Ombudsman, on behalf of a professor from the Department of Philosophy."

/A philosopher.  Great./  "I wish that people would file these cases against the program.  I've become the de facto lightning rod, and that's completely outside my job description."  He heaved a sigh, recalling the time lost and menta anguish of the first trial.  "What are the charges?"

"Espionage and destruction of university property."

Jesse felt ill again.  He had been cleared of wrongdoing in the first trial brought against him for espionage.  There had been no other noteworthy trips or incidents since then, not for Jesse or anyone else affiliated with the investigative journalism program.  The most exciting event in months had been Jesse's meeting with the U.S. ambassador to India to discuss possible partnerships with Indian universities.  That meeting hadn't even received a writeup in the school newspaper.

"I think you'll have better support from within CUNY this time," Sascha said, sensing Jesse's discomfort at the news.  "There's a precedent now, and the prosecutor won't be keen on repeating his last performance.  And that's assuming it even goes to trial.  I haven't heard any of the details or what the evidence is.  We'll know more in a few days."

The remainder of the ride to Jesse's apartment was silent, punctuated only by the occasional roar of a passing engine and the clicking of the car's turn signal.  Jesse thought about his neighborhood.  About the walks that he took each morning when he wasn't too exhausted to move.  About the lady on the corner whose dog reminded him of Ozzy Osbourne.  About the apartment that he had lived in since he moved to New York after college.  About the tree in the backyard that would be perfect for a treehouse for his kids someday.

It wasn't fair that he would need to move, but Jesse never thought that life owed him anything.  It's amazing that we're here at all, this species of creative primate.  There's no pre-determined order to things.  When you mixed together cause, effect, and complexity and unleashed them in a place governed by a set of physical laws, the result was pretty amazing.  A place full of turmoil and violence and uncertainty, but also of beauty and wonder and possibility.  The world that Jesse lived in was not one to fear, but he recognized that he still needed to protect himself.  Sometimes the wolves came to your door, and it wasn't always your fault.  It wasn't always your blood that they thirsted after, but the thirst often clouded their judgment.  Societies weren't small anymore, and the loosening of bonds between people in the same community made it easier for the wolves to find easy prey.  A random guy at a random university who held a low-profile job in an unpopular department wouldn't be missed.  It would be an easy meal, and one that would send chilling ripples throughout the program.  A message to the young spies in training.

Jesse thanked Sascha for the lift and walked up to his apartment.  He took a survey of the area as he walked and knew that Sascha would be doing the same as he watched Jesse enter the building.  All was quiet, and the night air seemed sweet in Jesse's nose.  He unlocked his door and went inside, flipping on the lights and dropping his keys on the countertop.  He looked around at his kitchen and living room.  This had been the eighth apartment that he toured, and it was perfect.  It was hard to imagine living anywhere else.

Jesse pulled his phone book from a drawer in the kitchen and looked at the listings for professional movers.  He wrote a few of the names and numbers on a scratch pad and put the book away.  It would wait until the morning.  He poured himself a glass of milk and sat in his easy chair in the living room.  The events of the day had made the whole of his life seem surreal.  /This morning I was on my way to an ordinary day at work/, Jesse thought.  /Now I'm under investigation for a second time, and someone finds me interesting enough to listen in on my conversations/.  Having finished his milk, he set the glass on an end table and leaned his head back on the chair, closing his eyes.


Jesse woke in the morning when his alarm clock began singing and squawking about the day's news.  He hadn't moved all night, and his body was quick to inform him that sleeping upright is an acquired habit.  His neck and shoulders ached.  Jesse stood up and walked to the front door to collect his copy of the /New York Times/ that was delivered each morning at some obscene hour.  The day's lead story was about political posturing between the governments of Afghanistan and Pakistan.  The investigative journalism program hadn't approved any requests for projects in those two nations because of security concerns, but Jesse knew that there were hundreds of interesting stories to follow.  He knew that they would graduate students who would go on to spend their entire careers in that region of the world.

The kitchen in Jesse's apartment was clean when one considered that a busy bachelor lived there by himself.  Two boxes of cereal sat on top of the refrigerator, and the counter top was clean and empty except for a coffee maker, a toaster, and a knife block.  A large collection of spices and teas were neatly tucked into a face-height cabinet above the sink.  During less busy times at the office, Jesse would be busy preparing a cooked breakfast and steeping a cup of white tea.  Today, though, a bowl of cereal would have to suffice.

After breakfast, Jesse brewed some strong coffee and absently read a few of the headlines in his newspaper.  The thing question of where he would do his work today occupied him like the flames of a roaring fire.  Sascha had offered to look at the available rooms, but Jesse thought that it might be wise to go off the grid and choose a place that would be less obvious and predictable.  A faculty lounge, or even the student cafeteria, would give a certain amount of peace and would enable Jesse to work incognito.  There was always the problem of overly-curious students or faculty looking over his shoulder, but that was a reasonable risk to take given the circumstances.  He decided that the student cafeteria would be his new office, and he would find a corner that afforded a good view of his surroundings.

The next step in the morning ritual was to shower and shave.  Jesse enjoyed these two steps more than usual.  The full night of quality sleep, his first in over a week, had left him in an upbeat mood.  /I can handle this.  Whatever it is, and whoever it is, I've got this/.  After getting himself cleaned up, he packed up his laptop computer, mobile phone, and his staff badge.  He locked the door and stepped out into the morning air, relieved not to find someone prowling around his apartment.  He walked down the wooden stairs that led to a short segment of sidewalk running between a small parking lot and the community garden where Jesse's neighbors grew potatoes, cabbage, and a number of leafy greens.

Jesse had never been much of a driver, and the morning traffic reminded him why he had chosen to take the bus.  He was on the road at 7:30 a.m., an hour and a quarter before traffic would be at its peak, but it was more than 40 minutes before he arrived at the east parking structure on the CUNY campus.  Sascha was the more cautious driver of the two of them -- a fact that always surprised Jesse -- and he had delivered Jesse to his apartment from the same parking structure in eleven minutes flat.  This created a complex set of feelings for Jesse on this particular morning, but mostly he was happy to arrive at work without any complications and without any tracking devices in his clothes.

The morning went by quickly.  Jesse had installed himself in a corner of the student cafeteria a few steps from both the men's room and an array of carafes of coffee.  He burned through a stack of paperwork that had accumulated the day before.  The steady chatter from the cafeteria provided a pleasant din that was as soothing as silence.  A few students recognized him from social events that he had chaired over the years, but no one bothered him.

Jesse was packing up to get some lunch when he noticed a new email in his inbox.  The subject heading read "Fwd: Escalation of service ticket related to door security in the Sheckell Building".  It was from Sascha, but the original author appeared to be an executive at the software firm that was hired to build the CUNY icket scheduling system.  The message had been cryptographically signed and encrypted both by Sascha and the original author, meaning two things.  First, the messages were authentic, since only Sascha and the gentleman from the software firm had copies of the keys that were used to sign and encrypt the emails.  Likewise, Jesse was the only one with a copy of his own key, meaning that only he could decrypt and read the message that Sascha had forwarded.  The second meaning of the signed and encrypted message was that it was important and sensitive.


PGP, or Pretty Good Privacy, was mainly used by just a few groups: curious computer enthusiasts, privacy wonks, diplomats dealing with sensitive information, and dissidents.  When someone wanted to send an electronic message to a contact, there was no guarantee that the message would arrive without being intercepted.  Much like tapping a phone line by splicing a few wires, the data that travelled acrass the internet could be seen by anyone along the path that the data took on its way to its destination.  This posed a serious problem for anyone wishing to have a private conversation.  In the Middle Ages, kings had solved this problem by inventing shift ciphers.  For example, they might rotate each letter one space in the alphabet so that /mouse/ became /npvtf/.  This was effective, but eventually everyone learned the cipher.  More sophisticated techniques were developed by mathematicians, especially those who were employed by well-funded militaries.

When computers arrived on the scene, though, the game changed.  A computer could check every popular shift cipher in a matter of microseconds, making it difficult to encrypt messages in a way that ensured their privacy despite the presence of many prying eyes.  PGP worked by asking each party in a secure conversation to generate two very long strings of numbers.  One was called a /private key/, and the other a /public key/.  Each person gives everyone else his public key and carefully protects his private key.  If Jesse wanted to tell Sascha that lunch would be in the cafeteria on Friday and wanted to be certain that his boss wouldn't read the message, then he would encrypt the message using Sascha's public key.  The magic of PGP lies in the next step.  Sascha's public and private keys were created as a pair.  When Sascha receives a message that was encrypted with his public key, then he uses his private key to decrypt and read it.  His private key is the only key in existence that can do so.  If Sascha wanted to confirm the information, then he could send a message back to Jesse by encrypting it with Jesse's public key.  Since Jesse is the only one with a copy of his private key, he is the only one able to read Sascha's return message.


Jesse typed in his passcode, unlocking his private key and allowing his email program to decrypt the message.

	-----BEGIN PGP SIGNED MESSAGE-----
	Hash: SHA1
	
	Jesse,
	
	It's amazing what you can achieve when you put your hand on the center of mass of a situation and give it a healthy bodyslam.  Give this a look.
	
	-- Sascha
	
	> Dear Mr. Greene,
	>
	> Thank you for writing.  I am very sorry about the difficulties with the
	> ticket system.  It is inexcusable that such a glitch would make it into
	> the production version of this software, and we will be taking immediate
	> steps to fix it and work with you to ensure that the changes are deployed
	> as soon as possible.  I will be investigating this matter personally and
	> have launched an internal investigation to determine how the problem
	> evaded the procedures that we use to identify security flaws in our 
	> products.
	>
	> I feel it necessary to inform you that we recently terminated an
	> employee's contract because he confessed to installing backdoors in
	> two of our flagship products that are targetted at large corporate
	> clients.  CUNY is not in the target audience for those products, and
	> neither of them is deployed at your campus.  I believe in transparency
	> even when it is uncomfortable, and I am simply trying to keep you informed
	> as we work our way through the details of this security breach.
	
	> Given the confidential nature of this issue, I expect that you will keep
	> the information in this email, and any subsequent emails, to yourself.  If
	> any tampering or foul play is discovered on your end or ours, we will work
	> with you to prosecute the individuals responsible to the fullest extent.
	>
	> Thank you for your business and cooperation.
	>
	> Sincerely,
	>
	> Theodore R. Ginsburg, Ph. D
	> CSO, Tall Oak Microsystems
	>
	>
	-----END PGP SIGNED MESSAGE-----

/Sascha, you rabble rouser/.  Jesse knew that there was a good story behind this email, but the details could wait.  Sascha did his best work when he was forced to make things happen on his own.  The slow pace of university politics and decision-making caused Sascha a lot of heartburn, and this was the type of situation that he thrived on.  Sascha's boss gave him a lot of latitude in his work, especially in circumstances where the system was failing.  With the safety of staff and students on the line, it was almost a blank check.  If Sascha needed to go over the heads of I.T., the dean's office, or even the board members, then that's just how things happened.  It would wait, though.  Jesse had subsisted on coffee since breakfast, and his body was ready to collect on its loan with interest.


The lunch options that were available at the CUNY cafeteria were nothing short of amazing.  When Jesse was an undergraduate in Colorado, his meal plan could net him a couple of pizza slices, a burger, or maybe a bit of stirfry.  If he wanted something that actually satisfied his nutritional needs, he would need to find time in the day to visit the local grocery or farmers' market and find a place to cook the food.  It was a subpar situation, and Jesse was glad to move into his first apartment where he had access to a kitchen that was larger than a linen closet.  The CUNY students, on the other hand, had it made.  They swiped their student cards at the door and walked into a plaza that offered no fewer than eight food stations, each offering a unique couisine.  If a student was in the mood for sushi, Thai food, Chicago-style pizza, vegetarian entrees, or a peanut butter sandwich, she only needed to wander to the correct counter and point a finger.  The hot food was prepared while she waited, and the ingredients were purchased fresh from area farmers as much as possible.  CUNY had its share of problems, but they always got the food right.

Jesse stood under the Thai Bear banner, waiting for a plate of pork satay.  He admired the skill of the brown-haired student who had taken his order and thrown it onto the large, circular cooking surface behind the counter.  A six-foot diameter vent mounted on the ceiling took care of the smoke and excess heat from the food as it cooked.  /I need one of those in my kitchen/, he thought.  /Maybe I can cook for her next time/.  Jesse had barely finished the thought when he saw someone approach on his left.  He saw a middle-aged man with a gray beard and a full head of white hair.  The man was wearing a flannel shirt and had his sleeves rolled up neatly at the elbow.

"Mr. Winter, I'm Dr. Ian Blair from the Department of Philosophy," the man said, offering his hand.

Jesse nodded, shaking Dr. Blair's hand.  /This must be the guy who's suing me/, Jesse thought.  "It's nice to meet you, Dr. Blair.  How are you doing today?"

"Fine, just fine.  Please call me Ian.  I was wondering if I could talk to you for a moment.  I have a table over by the window."  Ian motioned to a place near the main entrance to the cafeteria.

"Umm, sure.  I'll be right there.  My food will be ready in a couple of minutes."

"Of course."

The man walked back to his seat, picking up a magazine and picking at his food while he read.  Before Jesse could come up with a suitable excuse to get out of what promised to be an awkward conversation, his tray of food was pushed across the counter toward him.  He thanked the cashier by name and headed toward the table where Ian Blair sat, engrossed in a magazine article.

"What's the occasion, professor?" Jesse said, sitting down across from Ian.

"Ah, yes.  Mr. Winter, I... understand that one of my colleagues might be making some trouble for you.  We philosophers argue about a lot of inconsequential things, and sometimes our zeal for debate and ideology spills over into the real world.  I assume that you know what I'm referring to."

"I think so.  I learned yesterday that there was another suit being brought against me, and this time it was from a professor in your department.  The charges are serious.  Not your run-of-the-mill philosophy discussion."

"Yes, that's why I asked you to chat with me.  I think that the rabbit hole goes deeper than anyone is willing to admit.  My colleague is known to mingle with some folks from Europe and southeast Asia.  I find them shady, and I know I'm not the only one who does.  Nothing has ever been proven, but there is a lot of anecdotal evidence that could be used to piece together a case if it was necessary.  I don't know the details of your case for obvious reasons.  But you should know that this man doesn't do anything half-ass.  If he's after you, then you'll need all the allies you can gather."

"I appreciate the insight, Ian.  But why are you helping me like this?"

Ian looked up at Jesse and then out the window.  "He came after me once.  I didn't defend myself as well as I might have, and it cost me a year's salary and a series of publications in good venues.  My reputation took a big hit, and it consumed a lot of my time.  The department didn't give me any protection.  Everyone cowered like scared cats.  It's time for the culture to change, and I want to be part of the solution."

Jesse raised his glass of iced tea in a gesture of respect.  "Thanks.  Do you have a business card?  I'll be in touch."

									
The next week was remarkably routine.  Jesse kept moving his workstation in hopes of shaking off his secret admirers.  He never noticed anything out of place and never saw anyone watching or following him.  There were no bugs in his apartment, to the best of his knowledge.  His superiors had brought in a professional forensics and security crew to inspect every inch of his old office, looking for any unrecognized fingerprints, equipment, or disturbed ceiling tiles.  Nothing had turned up.

A security review of the ticket system had been conducted as well.  The records showed that the phantom ticket that defeated the door security system was created by a terminal in the basement of the Entomology Building.  The bughouse, as the Entomology Building was lovingly called by the students that spent their time there, was one of the oldest buildings on campus.  It was once the president's residence, and the university hadn't gotten around to installing surveillance equipment, making any effort to track the user of a computer in its basement a futile exercise.  The only camera in the vicinity was too far away to show any facial detail of any individual who was coming or going.

When the police chief had learned that Sascha had destroyed the bug by sending down a sink drain, he was furious.  Sascha's instinct had been to isolate and then eliminate the threat, and he had done so.  Unfortunately, this was no time for battlefield tactics.  It the police had access to the bug, then they had a fighting chance of tracking down a receiver and the person responsible for installing the equipment.  As it stood, they could do nothing but encourage Jesse to change his habits and keep a watchful eye for anything suspicious.

										* * *

Jesse had just finished a review of two projects that were conducted by teams of students in Taiwan and Slovenia.  The team in Slovenia had done an excellent job researching the Soviet influence on modern school curriculum and what the young generation of politicians were doing to establish a new blend of Slovenian culture and national pride.  William Tweedie planned to give them an award for their work.  "Every one of these glory stories," he had told Jesse in private, "is another gaping hole in the arguments of those asshats who tried to bug you, and the guys in Poland.  We're winning, Jesse.  We're winning."

For Jesse's entire life, he had been told that there were winning sides and losing sides.  Good and evil.  The right way and the wrong way.  He had often wondered what it felt like to be on the other side.  Weren't the evil guys just fellows who happened to come to a different conclusion; guys who thought that they, in turn, were on the right side and that everyone else was wrong?  When he went to college, Jesse learned a great deal about argument, debate, and premises.  A premise was an assumption, a leg that you used to prop up the argument that followed.  If you bet on the wrong premises, a skilled opponent would crush you.  Some folks would perform the most graceful and complicated verbal gymnastics in order to save face.  They were wrong, and they sensed it, but the perried and dodged their way out of the fight.  This was a favorite strategy of politicians.  They would never admit obvious, well-known truths that would make them vulnerable, and thus they would put forth a flowery stream of words that diverted the conversation from genocide in Africa to the charming notion that their grandmother's coffee shop sold the best cookies on Third Street, and that's why America is the place where every kid should grow up.

The problem with all of this, though, was that smart people could construct a reasonable-sounding argument for any old thing.  Unless it was a cut-and-dried subject, or perhaps a problem on which rigorous science could be brought to bear, it was probably too complicated to fully understand.  Is it better to follow the Keynesian school of economics, or is the Austrian school the One True Way Forward?  Can a mixed economy sustainably deliver the free market that we worship /and/ the services that we've come to enjoy?  Politicians wrangle about these ideas endlessly, and each will break down your door to tell you that he's found the One True Way Forward.

The only answer for Jesse was to drop a moral anchor made up of the few things that you really cared about and let the chips fall where they may.  Those few things are what makes a person, and it's a miserable man who tries to live in a way that's out of sync with them.

Some people, left to their own devices, have no problem causing injury to other people who they will never meet.  The ability to separate ourselves from the suffering in the world is a vital survival tool, but it also gives rise to indifference.  When the world of each person was small, interactions with unfamiliar people usually led to violence or avoidance.  Even when the world grew larger and more industrious, people dealt locally.  They saw the impact of their decisions, and they felt the repercussions of their mistakes.  When the world grew large enough that individuals could move freely from one place to another, and to begin a new life in a new place without the knowledge of any previous acquaintance, the conditions were right for organizations of individuals to pride themselves on growing wealthy and powerful on the misfortune of millions.  Without the ability to look their victims in the eye, their evolutionary sense of altruism that had developed over billions of years in the forests and open plains was snuffed out, lacking the stimulus that it needed in order to guide them.

Someone had to watch the watchers.  To watch the centers of power.  Not only was it likely that those in power would be tempted to allow its scope and potency to creep and grow, but history had delivered the unalloyed verdict that it was inevitable.  One societal power structure after another had rendered its version of justice and its vision for progress, and one after another had tumbled.  We have learned many lessons, and modern power tends to live longer and hide its bad deeds more effectively.  As power learned how to act in public, though, its watchers needed to learn to read between the lines.  They needed to learn its whims, its ambitions, and the patterns in its expression.  By staying one step ahead, the watchers could give the power an immune system.  A means of keeping it honest.  A means of giving its hand a reason to follow its mouth.

Jesse felt strongly that investigative journalism was the key antidote, or at least one of them, for the world's political struggles.  Without a press that was willing to dig in, ask questions that make politicians incredibly uncomfortable, and be willing to rock the boat, there was no check on power.  

The same can be said about the day-to-day decisions of everyday citizens.  The ordinary person on the street had neither the means nor the motivation to learn whether the merchant who sold him his lunch paid a fair price for the raw ingredients.  Or whether the merchant bought from suppliers who he knew to be ethical.  Were the men who harvested the beans for his coffee that morning slaves, or were they honest men working for other honest men who made a living for themselves?  The separation between people that led to abuses was not the exclusive province of the rich and powerful.  It was not only the merchants and the owners of land and resources.  Each decision during the day of the average man has consequences.  This is the result of the division of labor that enabled society to make vast strides in production, health, and quality of life.

When a man or woman picks up a piece of fruit in a market and savors its flavor, that transaction reinforces the chain of events and exchanges that carried the fruit to that place.  There are billions upon billions of such exchanges that happen each day, and the conscientious man demands accountability in the chains that he encounters.  His desire for ethical expression, the chance to make the world a more perfect place, gives him pause.  It is investigative journalism that meets this demand.


"When will I hear more about the charges?" Jesse asked.  Jesse and Sascha were sitting alone in the Office of Internal Affairs.  It was the end of a long day for both of them.  Jesse had formally presented the students who traveled Slovenia for two months their award from the president.  During the photo session that followed, one of them had mentioned that she had found her calling in life.  She would be traveling to Europe after graduation to work for Amnesty International on human rights cases.

"Probably after the news about the Slovenian crew dies down," Sascha replied, stretching his legs and his head on his arm.  It would be bad P.R. for a faculty member to throw down the gauntlet on the same day that the program had given a prestigious award for a successful trip.  Even the critics can't say anything bad about this one."

"Except that the students should have left the work to the Slovenians."

"Ah, the old 'They're taking our jobs' argument, applied in reverse!  Maybe this will light a fire under some universities over there to do some honest investigation.  With the exception of Fisk, hardly anybody knows what journalism is anymore."

"I saw Fisk speak at M.I.T. once.  I remember thinking that he would make an excellent peace broker, because he knows the state of affairs so well that nobody could feed him any bullshit.  But then I realized that brutal honesty and incisive critique are out of bounds when you're dealing with diplomats.  They're trained in soft skills and polite chat, not in history."

Sascha laughed.  "Don't worry, Jesse.  We've got the next group of young journalists on the blocks right now, and our professors cite Fisk all the time.  In a few years, our biggest problem will be how to keep the students on our waiting list from tearing our doors off their hinges."

"Here's hoping," Jesse said.  "I'm going to head out, man.  Thanks for the chat."

"Any time.  Hey, let me know--"

Sascha's desk phone rang, interrupting his train of thought.  He picked up the receiver and said, "Greene".  He frowned, listening to the voice in the earpiece.  "Yeah, he's here."  Looking up at Jesse, he shrugged, looking puzzled.  "Sure thing.  I'll send him down.  Bye."

"I guess you'll be finding out about those charges tonight," Sascha said.

"They want to talk to me /now/?" Jesse said, sounding incredulous.

"You got it.  Room 290, ten minutes.  Bring your badge."

/Bring your badge/.  Jesse knew what that meant, and his heart sank.  When a police officer was under investigation, his gun and shield were taken until his name was cleared.  Around here, your badge was revoked and you were placed on administrative leave.  This was Jesse's equivalent of being neutered.

										* * *
										
Room 290 was cold and clean, a room that was designed to impress or intimidate.  Which effect took precedence depended on the lighting.  On this occasion, every light in the room was burning.  There were three conference tables arranged in a U-shape, with a single seat placed at the top of the U.  There were small drink knapkins laid on the tables in front of each seat, and karafes of fresh water had been deposited at even intervals.  Jesse counted ten fifteen seats.  /Full tribunal/, he thought, still puzzling over the choice of time and location.

"Mr. Winter, may I take your coat?" a voice said from Jesse's left.  He turned, nodding to the well-dressed woman whose face he recognized, though he couldn't place it.

"Yes, thank you."  Jesse slid off his overcoat and handed it to the woman.

"You're welcome," she said, pulling a hanger from the coat rack on the wall.  "I'm Barbara Nash.  I'll be the note taker for this meeting."

Jesse smiled, trying to hide his anxiety.  "I remember you from the Warsaw trial."

"With a bit of luck, this meeting will be less stressful than the trial was."

"Do you know why this was arranged so hastily?"

"This is just a preliminary inquiry.  It was arranged more than a week ago by the Office of the Ombudsman.  Normally the defendant isn't present, but the dean decided that you should be here, in light of the unusual circumstances surrounding the case."